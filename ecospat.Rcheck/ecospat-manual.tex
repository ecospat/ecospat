\nonstopmode{}
\documentclass[a4paper]{book}
\usepackage[times,inconsolata,hyper]{Rd}
\usepackage{makeidx}
\usepackage[utf8,latin1]{inputenc}
% \usepackage{graphicx} % @USE GRAPHICX@
\makeindex{}
\begin{document}
\chapter*{}
\begin{center}
{\textbf{\huge Package `ecospat'}}
\par\bigskip{\large \today}
\end{center}
\begin{description}
\raggedright{}
\item[Version]\AsIs{2.2.1}
\item[Date]\AsIs{2018-06-013}
\item[Title]\AsIs{Spatial Ecology Miscellaneous Methods}
\item[Author]\AsIs{Olivier Broennimann [cre, aut],
Valeria Di Cola [cre, aut],
Blaise Petitpierre [ctb],
Frank Breiner [ctb],
Manuela D`Amen [ctb],
Christophe Randin [ctb],
Robin Engler [ctb],
Wim Hordijk [ctb],
Julien Pottier [ctb],
Mirko Di Febbraro [ctb],
Loic Pellissier [ctb],
Dorothea Pio [ctb],
Ruben Garcia Mateo [ctb],
Anne Dubuis [ctb],
Daniel Scherrer [ctb],
Luigi Maiorano [ctb],
Achilleas Psomas [ctb],
Charlotte Ndiribe [ctb],
Nicolas Salamin [ctb],
Niklaus Zimmermann [ctb],
Antoine Guisan [aut]}
\item[Maintainer]\AsIs{Olivier Broennimann }\email{olivier.broennimann@unil.ch}\AsIs{}
\item[VignetteBuilder]\AsIs{knitr}
\item[Depends]\AsIs{ade4 (>= 1.6-2), ape (>= 3.2), gbm (>= 2.1.1), sp (>= 1.0-15)}
\item[Imports]\AsIs{adehabitatHR (>= 0.4.11), adehabitatMA (>= 0.3.8), biomod2 (>=
3.1-64), dismo (>= 0.9-3), ecodist (>= 1.2.9), maptools (>=
0.8-39), randomForest (>= 4.6-7), spatstat (>= 1.37-0), raster
(>= 2.5-8), rms (>= 4.5-0), MigClim (>= 1.6), gtools (>=
3.4.1), PresenceAbsence (>= 1.1.9), methods (>= 3.1.1),
doParallel (>= 1.0.10), foreach (>= 1.4.3), iterators (>=
1.0.8), parallel, classInt (>= 0.1-23), vegan (>= 2.4-1),
poibin (>= 1.3), snowfall (>= 1.61)}
\item[Suggests]\AsIs{rgdal (>= 1.2-15), rJava (>= 0.9-6), XML (>= 3.98-1.1), knitr
(>= 1.14)}
\item[LazyData]\AsIs{true}
\item[URL]\AsIs{}\url{http://www.unil.ch/ecospat/home/menuguid/ecospat-resources/tools.html}\AsIs{}
\item[Description]\AsIs{Collection of R functions and data sets for the support of spatial ecology analyses with a focus on pre-, core and post- modelling analyses of species distribution, niche quantification and community assembly. Written by current and former members and collaborators of the ecospat group of Antoine Guisan, Department of Ecology and Evolution (DEE) & Institute of Earth Surface Dynamics (IDYST), University of Lausanne, Switzerland.}
\item[License]\AsIs{GPL}
\item[BugReports]\AsIs{}\url{https://github.com/vdicolab/ecospat}\AsIs{}
\end{description}
\Rdcontents{\R{} topics documented:}
\inputencoding{utf8}
\HeaderA{ecospat-package}{Spatial Ecology Miscellaneous Methods}{ecospat.Rdash.package}
\aliasA{ecospat}{ecospat-package}{ecospat}
\keyword{package}{ecospat-package}
%
\begin{Description}\relax
Collection of methods, utilities and data sets for the support of spatial ecology analyses with a focus on pre-, core and post- modelling analyses of species distribution, niche quantification and community assembly. Specifically, 

\bold{-Pre-modelling:}

Spatial autocorrelation --> \code{ecospat.mantel.correlogram};

Variable selection --> \code{ecospat.npred};

Climate Analalogy --> \code{ecospat.climan, ecospat.mess} and \code{ecospat.plot.mess};

Phylogenetic diversity measures --> \code{ecospat.calculate.pd};

Biotic Interactions --> \code{ecospat.co-occurrences} and \code{ecospat.Cscore};

Minimum Dispersal routes --> \code{ecospat.mdr};

Niche Quantification --> \code{ecospat.grid.clim.dyn, ecospat.niche.equivalency.test,} 

\code{ecospat.niche.similarity.test, ecospat.plot.niche, ecospat.plot.niche.dyn,} 

\code{ecospat.plot.contrib, ecospat.niche.overlap, ecospat.plot.overlap.test,} 

\code{ecospat.niche.dyn.index} and \code{ecospat.shift.centroids};

Data Preparation --> \code{ecospat.caleval, ecospat.cor.plot, ecospat.makeDataFrame,} 

\code{ecospat.occ.desaggregation, ecospat.rand.pseudoabsences, ecospat.rcls.grd,} 

\code{ecospat.recstrat\_prop, ecospat.recstrat\_regl} and \code{ecospat.sample.envar};

\bold{-Core Niche Modelling:}

Model evaluation --> \code{ecospat.cv.glm, ecospat.permut.glm, ecospat.cv.gbm,} 

\code{ecospat.cv.me, ecospat.cv.rf, ecospat.boyce, ecospat.CommunityEval,} 

\code{ecospat.cohen.kappa, ecospat.max.kappa, ecospat.max.tss, ecospat.meva.table,} 

\code{ecospat.plot.kappa, ecospat.plot.tss} and \code{ ecospat.adj.D2.glm};

Spatial predictions and projections --> \code{ecospat.ESM.Modeling,} 

\code{ecospat.ESM.EnsembleModeling, ecospat.ESM.Projection, ecospat.ESM.EnsembleProjection,} 

\code{ecospat.SESAM.prr, ecospat.migclim, ecospat.binary.model, ecospat.Epred}

and \code{ecospat.mpa};

Variable Importance --> \code{ecospat.maxentvarimport};

\bold{-Post Modelling:}

Variance Partition --> \code{ecospat.varpart};

Spatial predictions of species assemblages --> \code{ecospat.cons\_Cscore};

Range size quantification --> \code{ecospat.rangesize} and 

\code{ecospat.occupied.patch};

The \code{ecospat} package was written by current and former members and collaborators of the ecospat group of Antoine Guisan, Department of Ecology and Evolution (DEE) \& Institute of Earth Surface Dynamics (IDYST), University of Lausanne, Switzerland. 

\end{Description}
%
\begin{Details}\relax

\Tabular{ll}{
Package: & ecospat\\{}
Type: & Package\\{}
Version: & 2.2.0\\{}
Date: & 2017-11-22\\{}
License: & GPL \\{}
}

\end{Details}
%
\begin{Author}\relax
Olivier Broennimann [aut],
Valeria Di Cola [cre, aut],
Blaise Petitpierre [ctb],
Frank Breiner [ctb],
Manuela D`Amen [ctb],
Christophe Randin [ctb],
Robin Engler [ctb],
Wim Hordijk [ctb],
Julien Pottier [ctb],
Mirko Di Febbraro [ctb],
Loic Pellissier [ctb],
Dorothea Pio [ctb],
Ruben Garcia Mateo [ctb],
Anne Dubuis [ctb],
Daniel Scherrer [ctb],
Luigi Maiorano [ctb],
Achilleas Psomas [ctb],
Charlotte Ndiribe [ctb]
Nicolas Salamin [ctb],
Niklaus Zimmermann [ctb],
Antoine Guisan [aut]

\end{Author}
\inputencoding{utf8}
\HeaderA{ecospat.adj.D2.glm}{Calculate An Adjusted D2}{ecospat.adj.D2.glm}
%
\begin{Description}\relax
This function is used for calculating an adjusted D2 from a calibrated GLM object
\end{Description}
%
\begin{Usage}
\begin{verbatim}
    ecospat.adj.D2.glm(glm.obj)
\end{verbatim}
\end{Usage}
%
\begin{Arguments}
\begin{ldescription}
\item[\code{glm.obj}] Any calibrated GLM object with a binomial error distribution
\end{ldescription}
\end{Arguments}
%
\begin{Details}\relax
This function takes a calibrated GLM object with a binomial error distribution and returns an evaluation of the model fit.
The measure of the fit of the models is expressed as the percentage of explained deviance adjusted by the number of degrees
of freedom used (similar to the adjusted-R2 in the case of Least-Square regression; see Weisberg 1980) and is called the adjusted-D2
(see guisan and Zimmermann 2000 for details on its calculation).
\end{Details}
%
\begin{Value}
Returns an adjusted D square value (proportion of deviance accounted for by the model).
\end{Value}
%
\begin{Author}\relax
Christophe Randin \email{christophe.randin@unibas.ch} and Antoine Guisan \email{antoine.guisan@unil.ch}
\end{Author}
%
\begin{References}\relax
Weisberg, S. 1980. Applied linear regression. Wiley.

Guisan, A., S.B. Weiss and A.D. Weiss. 1999. GLM versus CCA spatial modeling of plant species distribution. \emph{Plant Ecology}, \bold{143}, 107-122.

Guisan, A. and N.E. Zimmermann. 2000. Predictive habitat distribution models in ecology. \emph{Ecol. Model.}, \bold{135}, 147-186.
\end{References}
%
\begin{Examples}
\begin{ExampleCode}

glm.obj<-glm(Achillea_millefolium~ddeg+mind+srad+slp+topo, 
family = binomial, data=ecospat.testData)

ecospat.adj.D2.glm(glm.obj)

\end{ExampleCode}
\end{Examples}
\inputencoding{utf8}
\HeaderA{ecospat.binary.model}{Generate Binary Models}{ecospat.binary.model}
%
\begin{Description}\relax
Generate a binary map from a continuous model prediction.
\end{Description}
%
\begin{Usage}
\begin{verbatim}
ecospat.binary.model (Pred, Threshold)
\end{verbatim}
\end{Usage}
%
\begin{Arguments}
\begin{ldescription}
\item[\code{Pred}] RasterLayer predicted suitabilities from a SDM prediction.
\item[\code{Threshold}] A threshold to convert continous maps into binary maps (e.g. the output of the function \code{ecospat.mpa}() or use the \code{optimal.thresholds} from PresenceAbsence R package.
\end{ldescription}
\end{Arguments}
%
\begin{Details}\relax
This function generates a binary model prediction (presence/absence) from an original model applying a threshold. The threshold could be arbitrary, or be based on the maximum acceptable error of false negatives (i.e. percentage of the presence predicted as absences, omission error).
\end{Details}
%
\begin{Value}
The binary model prediction (presence/absence).
\end{Value}
%
\begin{Author}\relax
Ruben G. Mateo \email{rubeng.mateo@gmail.com} with contributions of Frank Breiner \email{frank.breiner@wsl.ch}
\end{Author}
%
\begin{References}\relax
Fielding, A.H. and J.F. Bell. 1997. A review of methods for the assessment of prediction errors in conservation presence/absence models. \emph{Environmental Conservation}, \bold{24}: 38-49.

Engler, R., A Guisan and L. Rechsteiner. 2004. An improved approach for predicting the distribution of rare and endangered species from occurrence and pseudo-absence data. \emph{Journal of Applied Ecology}, \bold{41}, 263-274.

Liu, C., Berry, P. M., Dawson, T. P. and R. G. Pearson. 2005. Selecting thresholds of occurrence in the prediction of species distributions. \emph{Ecography}, \bold{28}, 385-393.

Jimenez-Valverde, A. and J.M.Lobo. 2007. Threshold criteria for conversion of probability of species presence to either-or presence-absence. \emph{Acta oecologica}, \bold{31}, 361-369.

Liu, C., White, M. and G. Newell. 2013. Selecting thresholds for the prediction of species occurrence with presence-only data. \emph{J. Biogeogr.}, \bold{40}, 778-789.

Freeman, E.A. and G.G. Moisen. 2008. A comparison of the performance of threshold criteria for binary classification in terms of predicted prevalence and kappa. \emph{Ecological Modelling}, \bold{217}, 48-58.
\end{References}
%
\begin{SeeAlso}\relax
\code{\LinkA{ecospat.mpa}{ecospat.mpa}}, \code{\LinkA{optimal.thresholds}{optimal.thresholds}}
\end{SeeAlso}
%
\begin{Examples}
\begin{ExampleCode}
library(dismo)


# only run if the maxent.jar file is available, in the right folder
jar <- paste(system.file(package="dismo"), "/java/maxent.jar", sep='')

# checking if maxent can be run (normally not part of your script)
file.exists(jar)
require(rJava)

# get predictor variables
fnames <- list.files(path=paste(system.file(package="dismo"), '/ex', sep=''), 
                     pattern='grd', full.names=TRUE )
predictors <- stack(fnames)


# file with presence points
occurence <- paste(system.file(package="dismo"), '/ex/bradypus.csv', sep='')
occ <- read.table(occurence, header=TRUE, sep=',')[,-1]
colnames(occ) <- c("x","y")

# fit model, biome is a categorical variable
me <- maxent(predictors, occ, factors='biome')

# predict to entire dataset
pred <- predict(me, predictors) 

plot(pred)
points(occ)


# use MPA to convert suitability to binary map (90% of occurrences encompass by binary map)
mpa.cutoff <- ecospat.mpa(pred,occ)

pred.bin.mpa <- ecospat.binary.model(pred,mpa.cutoff)

plot(pred.bin.mpa)
points(occ)
\end{ExampleCode}
\end{Examples}
\inputencoding{utf8}
\HeaderA{ecospat.boyce}{Calculate Boyce Index}{ecospat.boyce}
%
\begin{Description}\relax
Calculate the Boyce index as in Hirzel et al. (2006). The Boyce index is used to assess model performance.
\end{Description}
%
\begin{Usage}
\begin{verbatim}
ecospat.boyce (fit, obs, nclass=0, window.w="default", res=100, PEplot = TRUE)
\end{verbatim}
\end{Usage}
%
\begin{Arguments}
\begin{ldescription}
\item[\code{fit}] A vector or Raster-Layer containing the predicted suitability values
\item[\code{obs}] A vector containing the predicted suitability values or xy-coordinates (if "fit" is a Raster-Layer) of the validation points (presence records)
\item[\code{nclass}] The number of classes or vector with class thresholds. If \code{nclass=0}, the Boyce index is calculated with a moving window (see next parameters)
\item[\code{window.w}] The width of the moving window (by default 1/10 of the suitability range)
\item[\code{res}] The resolution of the moving window (by default 100 focals)
\item[\code{PEplot}] If true, plot the predicted to expected ratio along the suitability class
\end{ldescription}
\end{Arguments}
%
\begin{Details}\relax
The Boyce index only requires presences and measures how much model predictions differ from random distribution of the observed presences across the prediction gradients (Boyce et al. 2002). It is thus the most appropriate metric in the case of presence-only models. It is continuous and varies between -1 and +1. Positive values indicate a model which present predictions are consistent with the distribution of presences in the evaluation dataset, values close to zero mean that the model is not different from a random model, negative values indicate counter predictions, i.e., predicting poor quality areas where presences are more frequent (Hirzel et al. 2006).
\end{Details}
%
\begin{Value}
Returns the predicted-to-expected ratio for each class-interval: F.ratio

Returns the Boyce index value: Spearman.cor

Creates a graphical plot of the predicted to expected ratio along the suitability class
\end{Value}
%
\begin{Author}\relax
Blaise Petitpierre \email{bpetitpierre@gmail.com} and Frank Breiner \email{frank.breiner@unil.ch}
\end{Author}
%
\begin{References}\relax
Boyce, M.S., P.R. Vernier, S.E. Nielsen and F.K.A. Schmiegelow. 2002. Evaluating resource selection functions. \emph{Ecol. Model.}, \bold{157}, 281-300.

Hirzel, A.H., G. Le Lay, V. Helfer, C. Randin and A. Guisan. 2006. Evaluating the ability of habitat suitability models to predict species presences. \emph{Ecol. Model.}, \bold{199}, 142-152.
\end{References}
%
\begin{Examples}
\begin{ExampleCode}
obs <- (ecospat.testData$glm_Saxifraga_oppositifolia
[which(ecospat.testData$Saxifraga_oppositifolia==1)])

ecospat.boyce (fit = ecospat.testData$glm_Saxifraga_oppositifolia , obs, nclass=0, 
window.w="default", res=100, PEplot = TRUE)
\end{ExampleCode}
\end{Examples}
\inputencoding{utf8}
\HeaderA{ecospat.calculate.pd}{Calculate Phylogenetic Diversity Measures}{ecospat.calculate.pd}
%
\begin{Description}\relax
Calculate all phylogenetic diversity measures listed in Schweiger et al., 2008 (see full reference below).
\end{Description}
%
\begin{Usage}
\begin{verbatim}
ecospat.calculate.pd (tree, data, method="spanning", type="clade", root=FALSE, 
average=FALSE, verbose=TRUE)
\end{verbatim}
\end{Usage}
%
\begin{Arguments}
\begin{ldescription}
\item[\code{tree}] The phylogenetic tree
\item[\code{data}] A presence or absence (binary) matrix for each species (columns) in each location or grid cell (rows)
\item[\code{method}] The method to use. Options are "pairwise", "topology", and "spanning". Default is "spanning".
\item[\code{type}] Phylogenetic measure from those listed in Schweiger et al 2008. Options are "Q", "P", "W", "clade", "species", "J", "F", "AvTD","TTD", "Dd". Default is "clade".
\item[\code{root}] Phylogenetic diversity can either be rooted or unrooted. Details in Schweiger et al 2008. Default is FALSE.
\item[\code{average}] Phylogenetic diversity can either be averaged or not averaged. Details in Schweiger et al 2008. Default is FALSE.
\item[\code{verbose}] Boolean indicating whether to print progress output during calculation. Default is TRUE.
\end{ldescription}
\end{Arguments}
%
\begin{Details}\relax
Given a phylogenetic tree and a presence/absence matrix this script calculates phylogenetic diversity of a group of species across a given set of grid cells or locations. The library "ape" is required to read the tree in R. Command is "read.tree" or "read.nexus".
Options of type:
"P" is a normalized mearure of "Q".
"clade" is "PDnode" when root= FALSE, and is "PDroot" ehn root =TRUE.
"species" is "AvPD".
\end{Details}
%
\begin{Value}
This function returns a list of phylogenetic diversity values for each of the grid cells in the presence/absence matrix
\end{Value}
%
\begin{Author}\relax
Nicolas Salamin \email{nicolas.salamin@unil.ch} and Dorothea Pio \email{Dorothea.Pio@fauna-flora.org}
\end{Author}
%
\begin{References}\relax
Schweiger, O., S. Klotz, W. Durka and I. Kuhn. 2008. A comparative test of phylogenetic diversity indices. \emph{Oecologia}, \bold{157}, 485-495.

Pio, D.V., O. Broennimann, T.G. Barraclough, G. Reeves, A.G. Rebelo, W. Thuiller, A. Guisan and N. Salamin. 2011. Spatial predictions of phylogenetic diversity in conservation decision making. \emph{Conservation Biology}, \bold{25}, 1229-1239.

Pio, D.V., R. Engler, H.P. Linder, A. Monadjem, F.P.D. Cotterill, P.J. Taylor, M.C. Schoeman, B.W. Price, M.H. Villet, G. Eick, N. Salamin and A. Guisan. 2014. Climate change effects on animal and plant phylogenetic diversity in southern Africa. \emph{Global Change Biology}, \bold{20}, 1538-1549.
\end{References}
%
\begin{Examples}
\begin{ExampleCode}
fpath <- system.file("extdata", "ecospat.testTree.tre", package="ecospat")
tree <-read.tree(fpath)
data <- ecospat.testData[9:52] 

pd <- ecospat.calculate.pd(tree, data, method = "spanning", type = "species", root = FALSE, 
average = FALSE, verbose = TRUE )

plot(pd)
\end{ExampleCode}
\end{Examples}
\inputencoding{utf8}
\HeaderA{ecospat.caleval}{Calibration And Evaluation Dataset}{ecospat.caleval}
%
\begin{Description}\relax
Generate an evaluation and calibration dataset with a desired ratio of disaggregation.
\end{Description}
%
\begin{Usage}
\begin{verbatim}
ecospat.caleval (data, xy, row.num=1:nrow(data), nrep=1, ratio=0.7, 
disaggregate=0, pseudoabs=0, npres=0, replace=FALSE)
\end{verbatim}
\end{Usage}
%
\begin{Arguments}
\begin{ldescription}
\item[\code{data}] A vector with presence-absence (0-1) data for one species.
\item[\code{xy}] The x and y coordinates of the projection dataset.
\item[\code{row.num}] Row original number
\item[\code{nrep}] Number of repetitions
\item[\code{ratio}] Ratio of disaggregation
\item[\code{disaggregate}] Minimum distance of disaggregation (has to be in the same scale as xy)
\item[\code{pseudoabs}] Number of pseudoabsences
\item[\code{npres}] To select a smaller number of presences from the dataset to be subsetted. The maximum number is the total number of presences
\item[\code{replace}] F to replace de pseudoabsences
\end{ldescription}
\end{Arguments}
%
\begin{Details}\relax
This functions generates two list, one with the calibration or training dataset and other list with the evaluation or testing dataset disaggregated with a minimum distance.
\end{Details}
%
\begin{Value}
list("eval"=eval,"cal"=cal))
\end{Value}
%
\begin{Author}\relax
Blaise Petitpierre \email{bpetitpierre@gmail.com}
\end{Author}
%
\begin{Examples}
\begin{ExampleCode}
data <- ecospat.testData
caleval <- ecospat.caleval (data = ecospat.testData[53], xy = data[2:3], row.num = 1:nrow(data), 
nrep = 2, ratio = 0.7, disaggregate = 0.2, pseudoabs = 100, npres = 10, replace = FALSE)
caleval
\end{ExampleCode}
\end{Examples}
\inputencoding{utf8}
\HeaderA{ecospat.CCV.communityEvaluation.bin}{Calculates a range of community evaluation metrics based on different thresholding techniques.}{ecospat.CCV.communityEvaluation.bin}
\keyword{\textbackslash{}textasciitilde{}kwd1}{ecospat.CCV.communityEvaluation.bin}
\keyword{\textbackslash{}textasciitilde{}kwd2}{ecospat.CCV.communityEvaluation.bin}
%
\begin{Description}\relax
The function uses the output of \code{\LinkA{ecospat.CCV.modeling}{ecospat.CCV.modeling}} to calculate a range of community evaluation metrics based on a selection of thresholding techniques both for the calibration data and independent evaluation data.
\end{Description}
%
\begin{Usage}
\begin{verbatim}
ecospat.CCV.communityEvaluation.bin(ccv.modeling.data,
                                    thresholds= c('MAX.KAPPA', 'MAX.ROC','PS_SDM'),
                                    community.metrics=c('SR.deviation','Sorensen'),
                                    parallel=TRUE,
                                    cpus=4,
                                    fix.threshold=0.5,
                                    MCE=5,
                                    MEM=NULL)
\end{verbatim}
\end{Usage}
%
\begin{Arguments}
\begin{ldescription}
\item[\code{ccv.modeling.data}] a \code{'ccv.modeling.data'} object returned by \code{\LinkA{ecospat.CCV.modeling}{ecospat.CCV.modeling}}
\item[\code{thresholds}] a selection of thresholds (\code{'FIXED', 'MAX.KAPPA', 'MAX.ACCURACY', 'MAX.TSS', 'SENS\_SPEC', 'MAX.ROC', 'OBS.PREVALENCE', 'AVG.PROBABILITY', 'MCE', 'PS\_SDM, MEM'}) to be calculated and applied for the model evaluation.
\item[\code{community.metrics}] a selection of community evaluation metrics (\code{'SR.deviation', 'community.AUC', 'community.overprediction' ,'community.underprediction',' community.accuracy', 'community.sensitivity', 'community.specificity', 'community.kappa', 'community.tss', 'Sorensen', 'Jaccard', 'Simpson'}) to be calculated for each seleted thresholding technique.
\item[\code{parallel}] should parallel computing be allowed (\code{TRUE/FALSE})
\item[\code{cpus}] number of cpus to use in parallel computing
\item[\code{fix.threshold}] fixed threshold to be used. Only gets used if thresholding technique \code{FIXED} is selected.
\item[\code{MCE}] maximum omission error (\%) allowed for the thresholding. Only gets used if thresholding technique \code{MCE} is selected.
\item[\code{MEM}] a vetor with the species richness prediction of a MEM for each site. Only needed if \code{MEM} is selected.

\end{ldescription}
\end{Arguments}
%
\begin{Details}\relax
The function uses the probability output of the \code{\LinkA{ecospat.CCV.modeling}{ecospat.CCV.modeling}} function and creates binary maps based on the selected thresholding methods. These binary maps are then used to calculate the selected community evaluation metrics both for the calibration and evaluation data of each modeling run.
\end{Details}
%
\begin{Value}
\begin{ldescription}
\item[\code{DataSplitTable}] a matrix with \code{TRUE/FALSE} for each model run (\code{TRUE}=Calibration point, \code{FALSE}=Evaluation point)
\item[\code{CommunityEvaluationMetrics.CalibrationSites}] a 4-dimensional array containing the community evaluation metrics for the calibartion sites of each run (\code{NA} means that the site was used for evaluation)
\item[\code{CommunityEvaluationMetrics.EvaluationSites}] a 4-dimensional array containing the community evaluation metrics for the evaluation sites of each run (\code{NA} means that the site was used for calibaration)
\item[\code{PA.allSites}] a 4-dimensional array of the binary prediction for all sites and runs under the different thresholding appraoches.
\end{ldescription}
\end{Value}
%
\begin{Author}\relax
Daniel Scherrer <daniel.j.a.scherrer@gmail.com>
\end{Author}
%
\begin{References}\relax
Scherrer, D., D'Amen, M., Mateo, M.R.G., Fernandes, R.F. \& Guisan , A. (2018) How to best threshold and validate stacked species assemblages? Community optimisation might hold the answer. Methods in Ecology and Evolution, in review
\end{References}
%
\begin{SeeAlso}\relax
\code{\LinkA{ecospat.CCV.createDataSplitTable}{ecospat.CCV.createDataSplitTable}}; \code{\LinkA{ecospat.CCV.communityEvaluation.prob}{ecospat.CCV.communityEvaluation.prob}}
\end{SeeAlso}
%
\begin{Examples}
\begin{ExampleCode}
#Loading species occurence data and remove empty communities
testData <- ecospat.testData[,c(24,34,43,45,48,53,55:58,60:63,65:66,68:71)]
sp.data <- testData[which(rowSums(testData)>0), sort(colnames(testData))]

#Loading environmental data
env.data <- ecospat.testData[which(rowSums(testData)>0),4:8]

#Coordinates for all sites
xy <- ecospat.testData[which(rowSums(testData)>0),2:3]

#Running all the models for all species
myCCV.Models <- ecospat.CCV.modeling(sp.data = sp.data,
                                     env.data = env.data,
                                     xy = xy,
                                     NbRunEval = 5,
                                     minNbPredictors = 10,
                                     VarImport = 3)
                                     
#Thresholding all the predictions and calculating the community evaluation metrics
myCCV.communityEvaluation.bin <- ecospat.CCV.communityEvaluation.bin(
      ccv.modeling.data = myCCV.Models, 
      thresholds = c('MAX.KAPPA', 'MAX.ROC','PS_SDM'),
      community.metrics= c('SR.deviation','Sorensen'),
      parallel = TRUE,
      cpus = 4)
\end{ExampleCode}
\end{Examples}
\inputencoding{utf8}
\HeaderA{ecospat.CCV.communityEvaluation.prob}{Evaluates community predictions directly on the probabilities (i.e., threshold independent)}{ecospat.CCV.communityEvaluation.prob}
\keyword{\textbackslash{}textasciitilde{}kwd1}{ecospat.CCV.communityEvaluation.prob}
\keyword{\textbackslash{}textasciitilde{}kwd2}{ecospat.CCV.communityEvaluation.prob}
%
\begin{Description}\relax
This function generates a number of community evaluation metrics directly based on the probability returned by the individual models. Instead of thresholding the predictions (\code{\LinkA{ecospat.CCV.communityEvaluation.bin}{ecospat.CCV.communityEvaluation.bin}} this function directly uses the probability and compares its outcome to null models or average expectations.)
\end{Description}
%
\begin{Usage}
\begin{verbatim}
ecospat.CCV.communityEvaluation.prob(
      ccv.modeling.data,
      community.metrics=c('SR.deviation','community.AUC','probabilistic.Sorensen'),
      se.th=0.01,
      parallel = TRUE,
      cpus = 4)
\end{verbatim}
\end{Usage}
%
\begin{Arguments}
\begin{ldescription}
\item[\code{ccv.modeling.data}] a \code{'ccv.modeling.data'} object returned by \code{\LinkA{ecospat.CCV.modeling}{ecospat.CCV.modeling}}
\item[\code{community.metrics}] a selection of community metrics to calculate (\code{'SR.deviation', 'community.AUC', 'probabilistic.Sorensen', 'probabilistic.Jaccard', 'probabilistic.Simpson'})
\item[\code{se.th}] the desired precission for the community metrics (standard error of the mean)
\item[\code{parallel}] should parallel computing be allowed (\code{TRUE/FALSE})
\item[\code{cpus}] number of cpus to use in parallel computing
\end{ldescription}
\end{Arguments}
%
\begin{Value}
\begin{ldescription}
\item[\code{DataSplitTable}] a matrix with \code{TRUE/FALSE} for each model run (\code{TRUE}=Calibration point, \code{FALSE}=Evaluation point)
\item[\code{CommunityEvaluationMetrics.CalibrationSites}] a 3-dimensional array containing the community evaluation metrics for the calibartion sites of each run (\code{NA} means that the site was used for evaluation)
\item[\code{CommunityEvaluationMetrics.EvaluationSites}] a 3-dimensional array containing the community evaluation metrics for the evaluation sites of each run (\code{NA} means that the site was used for calibaration)
\end{ldescription}
\end{Value}
%
\begin{Note}\relax
If the community evaluation metric \code{'SR.deviation'} is selected the returned tables will have the following columns: 
\begin{itemize}

\item \code{SR.obs} = observed species richness, 
\item \code{SR.mean} = the predicted species richness (based on the probabilities assuming poission binomial distribution), 
\item \code{SR.dev} = the deviation of observed and predicted species richness, 
\item \code{SR.sd} = the standard deviation of the predicted species richness (based on the probabilities assuming poission binomial distribution), 
\item \code{SR.prob} = the probability that the observed species richness falls within the predicted species richness (based on the probabilities assuming poission binomial distribution), 
\item \code{SR.imp.05} = improvement of species richness prediction over null-model 0.5, 
\item \code{SR.imp.average.SR} = improvement of species richness prediction over null-model average.SR and 
\item \code{SR.imp.prevalence} = improvement of species richness prediction over null-model prevalence.

\end{itemize}


If the community evalation metric \code{community.AUC} is selected the returned tables will have the following colums: 
\begin{itemize}

\item \code{Community.AUC} = The AUC of ROC of a given site (in this case the ROC plot is community sensitiviy [percentage species predicted corretly present] vs 1 - community specificity [percentage of species predicted correctly absent])

\end{itemize}


If any of the other community evaluation metrics (\code{'probabilistic.Sorensen', 'probabilistic.Jaccard', 'probabilistic.Simpson'}) is selected the returned tables will have the follwing colums:
\begin{itemize}

\item \code{METRIC.mean} = The average Sorensen/Jaccard/Simpson based on a number of random draws of the probabilities.
\item \code{METRIC.sd} = The standard deviation of Sorensen/Jaccard/Simpson based on a number of random draws of the probabilities.
\item \code{METRIC.CI} = The 95\% confidence intervall of the average Sorensen/Jaccard/Simpson based on the standard deviation and number of draws. Should normally be <= \code{se.th}.
\item \code{nb.it} = number of draws used to estimate all the parameters. The draws stop as soon as the desired precission (\code{se.th}) is reached or the limit of allowed iterations (default=10'000).
\item composition.imp.05 = improvement of species compostion prediction over the null-model 0.5.
\item composition.imp.average.SR = improvement of the species composition prediction over the null-model average.SR.
\item composition.imp.prevalence = improvement of the species composition prediction over the null-model prevalence.

\end{itemize}


For detailed descriptions of the null models see Scherrer et al. .....
\end{Note}
%
\begin{Author}\relax
Daniel Scherrer <daniel.j.a.scherrer@gmail.com>
\end{Author}
%
\begin{SeeAlso}\relax
\code{\LinkA{ecospat.CCV.createDataSplitTable}{ecospat.CCV.createDataSplitTable}}; \code{\LinkA{ecospat.CCV.communityEvaluation.bin}{ecospat.CCV.communityEvaluation.bin}};
\end{SeeAlso}
%
\begin{Examples}
\begin{ExampleCode}
#Loading species occurence data and remove empty communities
testData <- ecospat.testData[,c(24,34,43,45,48,53,55:58,60:63,65:66,68:71)]
sp.data <- testData[which(rowSums(testData)>0), sort(colnames(testData))]

#Loading environmental data
env.data <- ecospat.testData[which(rowSums(testData)>0),4:8]

#Coordinates for all sites
xy <- ecospat.testData[which(rowSums(testData)>0),2:3]

#Running all the models for all species
myCCV.Models <- ecospat.CCV.modeling(sp.data = sp.data,
                                     env.data = env.data,
                                     xy = xy,
                                     NbRunEval = 5,
                                     minNbPredictors = 10,
                                     VarImport = 3)
                                     
#Calculating the probabilistic community metrics
myCCV.communityEvaluation.prob <- ecospat.CCV.communityEvaluation.prob(
      ccv.modeling.data = myCCV.Models,
      community.metrics = c('SR.deviation','community.AUC','probabilistic.Sorensen'),
      se.th = 0.02, 
      parallel = TRUE,
      cpus = 4)
\end{ExampleCode}
\end{Examples}
\inputencoding{utf8}
\HeaderA{ecospat.CCV.createDataSplitTable}{Creates a DataSplitTable for usage in ecospat.ccv.modeling.}{ecospat.CCV.createDataSplitTable}
\keyword{\textbackslash{}textasciitilde{}kwd1}{ecospat.CCV.createDataSplitTable}
\keyword{\textbackslash{}textasciitilde{}kwd2}{ecospat.CCV.createDataSplitTable}
%
\begin{Description}\relax
Creates a DataSplitTable with calibration and evaluation data either for cross-validation or repeated split sampling at the community level (i.e., across all species).

\end{Description}
%
\begin{Usage}
\begin{verbatim}
ecospat.CCV.createDataSplitTable(NbRunEval, 
                                 DataSplit,
                                 validation.method,
                                 NbSites,
                                 sp.data=NULL,
                                 minNbPresences=NULL,
                                 minNbAbsences=NULL,
                                 maxNbTry=1000)
\end{verbatim}
\end{Usage}
%
\begin{Arguments}
\begin{ldescription}
\item[\code{NbRunEval}] number of cross-validation or split sample runs
\item[\code{DataSplit}] proportion (\%) of sites used for model calibration
\item[\code{validation.method}] the type of \code{DataSplitTable} that should be created. Must be either \code{'cross-validation'} or \code{'split-sample'}
\item[\code{NbSites}] number of total sites available. Is ignored if sp.data is provided.
\item[\code{sp.data}] a data.frame where the rows are sites and the columns are species (values 1,0)
\item[\code{minNbPresences}] the desired minimum number of Presences required in each run
\item[\code{minNbAbsences}] the desired minimum number of Absences required in each run
\item[\code{maxNbTry}] number of random tries allowed to create a fitting DataSplitTable
\end{ldescription}
\end{Arguments}
%
\begin{Details}\relax
If a \code{sp.data} data.frame with species presences and absences is provided the function tries to create a \code{DataSplitTable} which ensures that the maximum possible number of species can be modelled (according to the specified minimum presences and absences.)
\end{Details}
%
\begin{Value}
\begin{ldescription}
\item[\code{DataSplitTable}] a matrix with \code{TRUE/FALSE} for each model run (\code{TRUE}=Calibration point, \code{FALSE}=Evaluation point)
\end{ldescription}





\end{Value}
%
\begin{Author}\relax
Daniel Scherrer <daniel.j.a.scherrer@gmail.com>

\end{Author}
%
\begin{SeeAlso}\relax
\code{\LinkA{ecospat.CCV.modeling}{ecospat.CCV.modeling}}

\end{SeeAlso}
%
\begin{Examples}
\begin{ExampleCode}
#Creating a DataSplitTable for 200 sites, 25 runs with an 
#80/20 calibration/evaluation cross-validation

DataSplitTable <- ecospat.CCV.createDataSplitTable(NbSites = 200, 
                                                   NbRunEval=25, 
                                                   DataSplit=80, 
                                                   validation.method='cross-validation')
                                                   
#Loading species occurence data and remove empty communities
testData <- ecospat.testData[,c(24,34,43,45,48,53,55:58,60:63,65:66,68:71)]
sp.data <- testData[which(rowSums(testData)>0), sort(colnames(testData))]

#Creating a DataSplitTable based on species data directly
DataSplitTable <- ecospat.CCV.createDataSplitTable(NbRunEval = 20,
                                                   DataSplit = 70,
                                                   validation.method = "cross-validation",
                                                   NbSites = NULL,
                                                   sp.data = sp.data, 
                                                   minNbPresence = 15, 
                                                   minNbAbsences = 15, 
                                                   maxNbTry = 250)
\end{ExampleCode}
\end{Examples}
\inputencoding{utf8}
\HeaderA{ecospat.CCV.modeling}{Runs indivudual species distribuion models with SDMs or ESMs}{ecospat.CCV.modeling}
\keyword{\textbackslash{}textasciitilde{}kwd1}{ecospat.CCV.modeling}
\keyword{\textbackslash{}textasciitilde{}kwd2}{ecospat.CCV.modeling}
%
\begin{Description}\relax
Creates probabilistic prediction for all species based on SDMs or ESMs and returns their evaluation metrics and variable importances.

\end{Description}
%
\begin{Usage}
\begin{verbatim}
ecospat.CCV.modeling(sp.data, 
                     env.data, 
                     xy,
                     DataSplitTable=NULL,
                     DataSplit = 70, 
                     NbRunEval = 25,
                     minNbPredictors =5,
                     validation.method = "cross-validation",
                     models.sdm = c("GLM","RF"), 
                     models.esm = "CTA", 
                     modeling.options.sdm = NULL, 
                     modeling.options.esm = NULL, 
                     ensemble.metric = "AUC", 
                     ESM = "YES",
                     parallel = TRUE, 
                     cpus = 4,
                     VarImport = 10,
                     modeling.id = as.character(format(Sys.time(), \code{"
}

\arguments{
  \item{sp.data}{a data.frame where the rows are sites and the columns are species (values 1,0)}
  \item{env.data}{either a data.frame where rows are sites and colums are environmental variables or a raster stack of the envrionmental variables}
  \item{xy}{two column data.frame with X and Y coordinates of the sites (most be same coordinate system as \code{env.data})}
  \item{DataSplitTable}{a table providing \code{TRUE/FALSE} to indicate what points are used for calibration and evaluation. As returned by \code{\LinkA{ecospat.CCV.createDataSplitTable}{ecospat.CCV.createDataSplitTable}}}
  \item{DataSplit}{percentage of dataset observations retained for the model training  (only needed if no \code{DataSplitTable} provided)}
  \item{NbRunEval}{number of cross-validatio/split sample runs (only needed if no \code{DataSplitTable} provided)}
  \item{minNbPredictors}{minimum number of occurences [min(presences/Absences] per predicotors needed to calibrate the models}
  \item{validation.method}{either "cross-validation" or "split-sample" used to validate the communtiy predictions (only needed if no \code{DataSplitTable} provided)}
  \item{models.sdm}{modeling techniques used for the normal SDMs. Vector of models names choosen among \code{'GLM', 'GBM', 'GAM', 'CTA', 'ANN', 'SRE', 'FDA', 'MARS', 'RF', 'MAXENT.Phillips' and 'MAXENT.Tsuruoka'}}
  \item{models.esm}{modeling techniques used for the ESMs. Vector of models names choosen among \code{'GLM', 'GBM', 'GAM', 'CTA', 'ANN', 'SRE', 'FDA', 'MARS', 'RF', 'MAXENT.Phillips' and 'MAXENT.Tsuruoka'}}
  \item{modeling.options.sdm}{modeling options for the normal SDMs. \code{"BIOMOD.models.options"}" object returned by \code{\LinkA{BIOMOD\_ModelingOptions}{BIOMOD.Rul.ModelingOptions}}}
  \item{modeling.options.esm}{modeling options for the ESMs. \code{"BIOMOD.models.options"} object returned by \code{\LinkA{BIOMOD\_ModelingOptions}{BIOMOD.Rul.ModelingOptions}}}
  \item{ensemble.metric}{evaluation score used to weight single models to build ensembles: \code{'AUC', 'Kappa' or 'TSS'}}
  \item{ESM}{either \code{'YES'} (ESMs allowed), \code{'NO'} (ESMs not allowed) or \code{'ALL'} (ESMs used in any case)}
  \item{parallel}{should parallel computing be allowed (\code{TRUE/FALSE})}
  \item{cpus}{number of cpus to use in parallel computing}
  \item{VarImport}{number of permutation runs to evaluate variable importance}
  \item{modeling.id}{character, the ID (=name) of modeling procedure. A random number by default}
\end{verbatim}
\end{Usage}
%
\begin{Details}\relax
The basic idea of the community cross-validation (CCV) is to use the same data (sites) for the model calibration/evaluation of all species. This ensures that there is "independent" cross-validation/split-sample data available not only at the individual species level but also at the community level. This is key to allow an unbiased estimation of the ability to predict species assemblages (Scherrer et al. 2018). 
The output of the ecospat.CCV.modeling function can then be used to evaluate the species assemblage predictions with the \code{\LinkA{ecospat.CCV.communityEvaluation.bin}{ecospat.CCV.communityEvaluation.bin}} or \code{\LinkA{ecospat.CCV.communityEvaluation.prob}{ecospat.CCV.communityEvaluation.prob}} functions.
\end{Details}
%
\begin{Value}
\begin{ldescription}
\item[\code{modelling.id}] character, the ID (=name) of modeling procedure
\item[\code{output.files}] vector with the names of the files written to the hard drive
\item[\code{speciesData.calibration}] a 3-dimensional array of presence/absence data of all species for the calibration plots used for each run
\item[\code{speciesData.evaluation}] a 3-dimensional array of presence/absence data of all species for the evaluation plots used for each run
\item[\code{speciesData.full}] a data.frame of presence/absence data of all species (same as \code{sp.data} input)
\item[\code{DataSplitTable}] a matrix with \code{TRUE/FALSE} for each model run (\code{TRUE}=Calibration point, \code{FALSE}=Evaluation point)
\item[\code{singleSpecies.ensembleEvaluationScore}] a 3-dimensional array of single species evaluation metrics (\code{'Max.KAPPA', 'Max.TSS', 'AUC of ROC'})
\item[\code{singleSpecies.ensembleVariableImportance}] a 3-dimensional array of single species variable importance for all predictors
\item[\code{singleSpecies.calibrationSites.ensemblePredictions}] a 3-dimensional array of the predictions for each species and run at the calibration sites
\item[\code{singleSpecies.evaluationSites.ensemblePredictions}] a 3-dimensional array of the predictions for each species and run at the evaluation sites
\end{ldescription}
\end{Value}
%
\begin{Author}\relax
Daniel Scherrer <daniel.j.a.scherrer@gmail.com>
\end{Author}
%
\begin{References}\relax
Scherrer, D., D'Amen, M., Mateo, M.R.G., Fernandes, R.F. \& Guisan , A. (2018) How to best threshold and validate stacked species assemblages? Community optimisation might hold the answer. Methods in Ecology and Evolution, in review
\end{References}
%
\begin{SeeAlso}\relax
\code{\LinkA{ecospat.CCV.createDataSplitTable}{ecospat.CCV.createDataSplitTable}}; \code{\LinkA{ecospat.CCV.communityEvaluation.bin}{ecospat.CCV.communityEvaluation.bin}}; \code{\LinkA{ecospat.CCV.communityEvaluation.prob}{ecospat.CCV.communityEvaluation.prob}}
\end{SeeAlso}
%
\begin{Examples}
\begin{ExampleCode}
#Loading species occurence data and remove empty communities
testData <- ecospat.testData[,c(24,34,43,45,48,53,55:58,60:63,65:66,68:71)]
sp.data <- testData[which(rowSums(testData)>0), sort(colnames(testData))]

#Loading environmental data
env.data <- ecospat.testData[which(rowSums(testData)>0),4:8]

#Coordinates for all sites
xy <- ecospat.testData[which(rowSums(testData)>0),2:3]

#Running all the models for all species
myCCV.Models <- ecospat.CCV.modeling(sp.data = sp.data,
                                     env.data = env.data,
                                     xy = xy,
                                     NbRunEval = 5,
                                     minNbPredictors = 10,
                                     VarImport = 3)
\end{ExampleCode}
\end{Examples}
\inputencoding{utf8}
\HeaderA{ecospat.climan}{A climate analogy setection tool for the modeling of species distributions}{ecospat.climan}
%
\begin{Description}\relax
Assess climate analogy between a projection extent (p) and a reference extent (ref, used in general as the background to calibrate SDMs)
\end{Description}
%
\begin{Usage}
\begin{verbatim}
ecospat.climan (ref, p)
\end{verbatim}
\end{Usage}
%
\begin{Arguments}
\begin{ldescription}
\item[\code{ref}] A dataframe with the value of the variables (i.e columns) for each point of the reference exent.
\item[\code{p}] A dataframe with the value of the variables (i.e columns) for each point of the projection exent.

\end{ldescription}
\end{Arguments}
%
\begin{Value}
Returns a vector. Values below 0 are novel conditions at the univariate level (similar to the MESS), values between 0 and 1 are analog and values above 1 are novel covariate condtions. 
For more information see Mesgeran et al. (2014)
\end{Value}
%
\begin{Author}\relax
Blaise Petitpierre \email{bpetitpierre@gmail.com}
\end{Author}
%
\begin{References}\relax
Mesgaran, M.B., R.D. Cousens and B.L. Webber. 2014. Here be dragons: a tool for quantifying novelty due to covariate range and correlation change when projecting species distribution models. \emph{Diversity \& Distributions}, \bold{20}, 1147-1159.
\end{References}
%
\begin{Examples}
\begin{ExampleCode}
x <- ecospat.testData[c(4:8)]
p<- x[1:90,] #A projection dataset.
ref<- x[91:300,] #A reference dataset
ecospat.climan(ref,p)

\end{ExampleCode}
\end{Examples}
\inputencoding{utf8}
\HeaderA{ecospat.cohen.kappa}{Cohen's Kappa}{ecospat.cohen.kappa}
%
\begin{Description}\relax
Calculates Cohen's kappa and variance estimates, within a 95 percent confidence interval.
\end{Description}
%
\begin{Usage}
\begin{verbatim}
    ecospat.cohen.kappa(xtab)
\end{verbatim}
\end{Usage}
%
\begin{Arguments}
\begin{ldescription}
\item[\code{xtab}] 
A symmetric agreement table.


\end{ldescription}
\end{Arguments}
%
\begin{Details}\relax
The argument xtab is a contingency table.
xtab <- table(Pred >= th, Sp.occ)
\end{Details}
%
\begin{Value}
A list with elements 'kap', 'vark', 'totn' and 'ci' is returned. 
'kap' is the cohen's kappa, 'vark' is the variance estimate within a 95 percent confidence interval, 'totn' is the number of plots and 'ci' is the confidence interval. 
\end{Value}
%
\begin{Author}\relax
Christophe Randin \email{christophe.randin@wsl.ch} with contributions of Niklaus. E. Zimmermann \email{niklaus.zimmermann@wsl.ch} and Valeria Di Cola \email{valeria.dicola@unil.ch}
\end{Author}
%
\begin{References}\relax
Bishop, Y.M.M., S.E. Fienberg and P.W. Holland. 1975. Discrete multivariate analysis: Theory and Practice. Cambridge, MA: MIT Press. pp. 395-397.

Pearce, J. and S. Ferrier. 2000. Evaluating the predictive performance of habitat models developed using logistic regression. \emph{Ecol. Model.}, \bold{133}, 225-245.
\end{References}
%
\begin{SeeAlso}\relax
\code{\LinkA{ecospat.meva.table}{ecospat.meva.table}}, \code{\LinkA{ecospat.max.tss}{ecospat.max.tss}}, \code{\LinkA{ecospat.plot.tss}{ecospat.plot.tss}}, \code{\LinkA{ecospat.plot.kappa}{ecospat.plot.kappa}}, \code{\LinkA{ecospat.max.kappa}{ecospat.max.kappa}}
\end{SeeAlso}
%
\begin{Examples}
\begin{ExampleCode}
Pred <- ecospat.testData$glm_Agrostis_capillaris
Sp.occ <- ecospat.testData$Agrostis_capillaris
th <- 0.39 # threshold
xtab <- table(Pred >= th, Sp.occ)

ecospat.cohen.kappa(xtab)
\end{ExampleCode}
\end{Examples}
\inputencoding{utf8}
\HeaderA{ecospat.CommunityEval}{Community Evaluation}{ecospat.CommunityEval}
%
\begin{Description}\relax
Calculate several indices of accuracy of community predictions.
\end{Description}
%
\begin{Usage}
\begin{verbatim}
ecospat.CommunityEval (eval, pred, proba, ntir)
\end{verbatim}
\end{Usage}
%
\begin{Arguments}
\begin{ldescription}
\item[\code{eval}] A matrix of observed presence-absence (ideally independent from the dataset used to fit species distribution models) of the species with n rows for the sites and s columns for the species.
\item[\code{pred}] A matrix of predictions for the s species in the n sites. Should have the same dimension as eval.
\item[\code{proba}] Logical variable indicating whether the prediction matrix contains presences-absences (FALSE) or probabilities (TRUE).
\item[\code{ntir}] Number of trials of presence-absence predictions if pred is a probability matrix.
\end{ldescription}
\end{Arguments}
%
\begin{Details}\relax
This function calculates several indices of accuracy of community predictions based on stacked predictions of species ditribution models. In case proba is set to FALSE the function returns one value per index and per site. In case proba is set to TRUE the function generates presences-absences based on the predicted probabilities and returns one value per index, per site and per trial.
\end{Details}
%
\begin{Value}
A list of evaluation metrics calculated for each site (+ each trial if proba is set to TRUE):

deviance.rich.pred: the deviation of the predicted species richness to the observed

overprediction: the proportion of species predicted as present but not observed among the species predicted as present

underprediction: the proportion of species predicted as absent but observed among the species observed as present

prediction.success: the proportion of species correctly predicted as present or absent

sensitivity: the proportion of species correctly predicted as present among the species observed as present

specificity : the proportion of species correctly predicted as absent among the species observed as absent

kappa: the proportion of specific agreement

TSS: sensitivity+specificity-1

similarity: the similarity of community composition between the observation and the prediction. The calculation is based on the Sorenses index.

Jaccard: this index is a widely used metric of community similarity.
\end{Value}
%
\begin{Author}\relax
Julien Pottier \email{julien.pottier@clermont.inra.fr} 

with contribution of Daniel Scherrer \email{daniel.scherrer@unil.ch}, Anne Dubuis \email{anne.dubuis@gmail.com} 
and Manuela D'Amen \email{manuela.damen@unil.ch}
\end{Author}
%
\begin{References}\relax
Pottier, J., A. Dubuis, L. Pellissier, L. Maiorano, L. Rossier, C.F. Randin, P. Vittoz and A. Guisan. 2013. The accuracy of plant assemblage prediction from species distribution models varies along environmental gradients. \emph{Global Ecology and Biogeography}, \bold{22}, 52-63.
\end{References}
%
\begin{Examples}
\begin{ExampleCode}
## Not run: 
eval <- Data[c(53,62,58,70,61,66,65,71,69,43,63,56,68,57,55,60,54,67,59,64)]
pred <- Data[c(73:92)]

ecospat.CommunityEval (eval, pred, proba=TRUE, ntir=10)

## End(Not run)
\end{ExampleCode}
\end{Examples}
\inputencoding{utf8}
\HeaderA{ecospat.cons\_Cscore}{Constrained Co-Occurrence Analysis.}{ecospat.cons.Rul.Cscore}
%
\begin{Description}\relax
Co-occurrence Analysis \& Environmentally Constrained Null Models. The function tests for non-random patterns of species co-occurrence in a presence-absence matrix. It calculates the C-score index for the whole community and for each species pair. An environmental constraint is applied during the generation of the null communities.
\end{Description}
%
\begin{Usage}
\begin{verbatim}
ecospat.cons_Cscore(presence,pred,nperm,outpath)
\end{verbatim}
\end{Usage}
%
\begin{Arguments}
\begin{ldescription}
\item[\code{presence}] A presence-absence dataframe for each species (columns) in each location or grid cell (rows) Column names (species names) and row names (sampling plots).
\item[\code{pred}] A dataframe object with SDM predictions. Column names (species names SDM) and row names (sampling plots).
\item[\code{nperm}] The number of permutation in the null model.
\item[\code{outpath}] Path to specify where to save the results.

\end{ldescription}
\end{Arguments}
%
\begin{Details}\relax
An environmentally constrained approach to null models will provide a more robust evaluation of species associations by facilitating the distinction between mutually exclusive processes that may shape species distributions and community assembly.
The format required for input databases: a plots (rows) x species (columns) matrix. Input matrices should have column names (species names) and row names (sampling plots).
NOTE: a SES that is greater than 2 or less than -2 is statistically significant with a tail probability of less than 0.05 (Gotelli \& McCabe 2002 - Ecology)
\end{Details}
%
\begin{Value}
Returns the C-score index for the observed community (ObsCscoreTot), the mean of C-score for the simulated communities (SimCscoreTot), p.value (PValTot) and standardized effect size (SES.Tot). It also saves a table in the specified path where the same 
metrics are calculated for each species pair (only the table with species pairs with significant p.values is saved in this version). 
\end{Value}
%
\begin{Author}\relax
Anne Dubuis \email{anne.dubuis@gmail.com} and Manuela D`Amen \email{manuela.damen@unil.ch}
\end{Author}
%
\begin{References}\relax
Gotelli, N.J. and D.J. McCabe. 2002. Species co-occurrence: a meta-analysis of JM Diamond`s assembly rules model. \emph{Ecology}, \bold{83}, 2091-2096.

Peres-Neto, P.R., J.D. Olden and D.A. Jackson. 2001. Environmentally constrained null models: site suitability as occupancy criterion. \emph{Oikos}, \bold{93}, 110-120.

\end{References}
%
\begin{Examples}
\begin{ExampleCode}
## Not run: 
presence <- ecospat.testData[c(53,62,58,70,61,66,65,71,69,43,63,56,68,57,55,60,54,67,59,64)]
pred <- ecospat.testData[c(73:92)]
nperm <- 10000
outpath <- getwd()
ecospat.cons_Cscore(presence, pred, nperm, outpath)

## End(Not run)
\end{ExampleCode}
\end{Examples}
\inputencoding{utf8}
\HeaderA{ecospat.cor.plot}{Correlation Plot}{ecospat.cor.plot}
%
\begin{Description}\relax
A scatter plot of matrices, with bivariate scatter plots below the diagonal, histograms on the diagonal, and the Pearson correlation above the diagonal. Useful for descriptive statistics of small data sets (better with less than 10 variables).
\end{Description}
%
\begin{Usage}
\begin{verbatim}
ecospat.cor.plot(data)
\end{verbatim}
\end{Usage}
%
\begin{Arguments}
\begin{ldescription}
\item[\code{data}] A dataframe object with environmental variables.
\end{ldescription}
\end{Arguments}
%
\begin{Details}\relax
Adapted from the pairs help page. Uses panel.cor, and panel.hist, all taken from the help pages for pairs. It is a simplifies version of \code{ pairs.panels}() function of the package \code{psych}.
\end{Details}
%
\begin{Value}
A scatter plot matrix is drawn in the graphic window. The lower off diagonal draws scatter plots, the diagonal histograms, the upper off diagonal reports the Pearson correlation.
\end{Value}
%
\begin{Author}\relax
Adjusted by L. Mathys, 2006, modified by N.E. Zimmermann
\end{Author}
%
\begin{Examples}
\begin{ExampleCode}
data <- ecospat.testData[,4:8]
ecospat.cor.plot(data)
\end{ExampleCode}
\end{Examples}
\inputencoding{utf8}
\HeaderA{ecospat.co\_occurrences}{Species Co-Occurrences}{ecospat.co.Rul.occurrences}
%
\begin{Description}\relax
Calculate an index of species co-occurrences.
\end{Description}
%
\begin{Usage}
\begin{verbatim}
ecospat.co_occurrences (data)
\end{verbatim}
\end{Usage}
%
\begin{Arguments}
\begin{ldescription}
\item[\code{data}] A presence-absence matrix for each species (columns) in each location or grid cell (rows) or a matrix with predicted suitability values. 
\end{ldescription}
\end{Arguments}
%
\begin{Details}\relax
Computes an index of co-occurrences ranging from 0 (never co-occurring) to 1 (always co-occuring).
\end{Details}
%
\begin{Value}
The species co-occurrence matrix and box-plot of the co-occurrence indices
\end{Value}
%
\begin{Author}\relax
Loic Pellissier \email{loic.pellissier@unifr.ch}
\end{Author}
%
\begin{References}\relax
Pellissier, L., K.A. Brathen, J. Pottier, C.F. Randin, P. Vittoz, A. Dubuis, N.G. Yoccoz, T. Alm, N.E. Zimmermann and A. Guisan. 2010. Species distribution models reveal apparent competitive and facilitative effects of a dominant species on the distribution of tundra plants. \emph{Ecography}, \bold{33}, 1004-1014.

Guisan, A. and N. Zimmermann. 2000. Predictive habitat distribution models in ecology. \emph{Ecological Modelling}, \bold{135}:147-186
\end{References}
%
\begin{Examples}
\begin{ExampleCode}
## Not run: 
matrix <- ecospat.testData[c(9:16,54:57)]
ecospat.co_occurrences (data=matrix)

## End(Not run)
\end{ExampleCode}
\end{Examples}
\inputencoding{utf8}
\HeaderA{ecospat.Cscore}{Pairwise co-occurrence Analysis with calculation of the C-score index.}{ecospat.Cscore}
%
\begin{Description}\relax
The function tests for nonrandom patterns of species co-occurrence in a presence-absence matrix. It calculates the C-score index for the whole community and for each species pair. Null communities have column sum fixed.
\end{Description}
%
\begin{Usage}
\begin{verbatim}
ecospat.Cscore (data, nperm, outpath)
\end{verbatim}
\end{Usage}
%
\begin{Arguments}
\begin{ldescription}
\item[\code{data}] 
A presence-absence dataframe for each species (columns) in each location or grid cell (rows). Column names (species names) and row names (sampling plots).

\item[\code{nperm}] 
The number of permutation in the null model.


\item[\code{outpath}] 
Path to specify where to save the results.


\end{ldescription}
\end{Arguments}
%
\begin{Details}\relax
This function allows to apply a pairwise null model analysis (Gotelli and Ulrich 2010) to a presence-absence community matrix to determine which species associations are significant across the study area. The strength of associations is quantified by the C-score index (Stone and Roberts 1990) and a 'fixed-equiprobable' null model algorithm is applied.
The format required for input databases: a plots (rows) x species (columns) matrix. Input matrices should have column names (species names) and row names (sampling plots). 
NOTE: a SES that is greater than 2 or less than -2 is statistically significant with a tail probability of less than 0.05 (Gotelli \& McCabe 2002). 

\end{Details}
%
\begin{Value}
The function returns the C-score index for the observed community (ObsCscoreTot), p.value (PValTot) and standardized effect size (SES.Tot). It saves also a table in the working directory where the same metrics are calculated for each species pair (only the table with species pairs with significant p-values is saved in this version)
\end{Value}
%
\begin{Author}\relax
Christophe Randin \email{	christophe.randin@wsl.ch} and Manuela D'Amen <manuela.damen@msn.com>
\end{Author}
%
\begin{References}\relax
Gotelli, N.J. and D.J. McCabe. 2002. Species co-occurrence: a meta-analysis of JM Diamond's
assembly rules model. \emph{Ecology}, \bold{83}, 2091-2096.

Gotelli, N.J. and W. Ulrich. 2010. The empirical Bayes approach as a tool to identify non-random species associations. \emph{Oecologia}, \bold{162}, 463-477

Stone, L. and A. Roberts, A. 1990. The checkerboard score and species distributions. \emph{Oecologia}, \bold{85}, 74-79

\end{References}
%
\begin{SeeAlso}\relax
\code{\LinkA{ecospat.co\_occurrences}{ecospat.co.Rul.occurrences}} and \code{\LinkA{ecospat.cons\_Cscore}{ecospat.cons.Rul.Cscore}}
\end{SeeAlso}
%
\begin{Examples}
\begin{ExampleCode}
## Not run: 
data<- ecospat.testData[c(53,62,58,70,61,66,65,71,69,43,63,56,68,57,55,60,54,67,59,64)]
nperm <- 10000
outpath <- getwd()
ecospat.Cscore(data, nperm, outpath)


## End(Not run)
\end{ExampleCode}
\end{Examples}
\inputencoding{utf8}
\HeaderA{ecospat.cv.example}{Cross Validation Example Function}{ecospat.cv.example}
%
\begin{Description}\relax
Run the cross validation functions on an example data set.
\end{Description}
%
\begin{Usage}
\begin{verbatim}
ecospat.cv.example ()
\end{verbatim}
\end{Usage}
%
\begin{Details}\relax
This function takes an example data set, calibrates it for various models, and then runs the cross validation functions on the results. Mainly to show how to use the cross validation functions.
\end{Details}
%
\begin{Author}\relax
Christophe Randin \email{	christophe.randin@wsl.ch} and Antoine Guisan \email{antoine.guisan@unil.ch}
\end{Author}
%
\begin{Examples}
\begin{ExampleCode}
## Not run:  
ecospat.cv.example ()

## End(Not run)
\end{ExampleCode}
\end{Examples}
\inputencoding{utf8}
\HeaderA{ecospat.cv.gbm}{GBM Cross Validation}{ecospat.cv.gbm}
%
\begin{Description}\relax
K-fold and leave-one-out cross validation for GBM.
\end{Description}
%
\begin{Usage}
\begin{verbatim}
ecospat.cv.gbm (gbm.obj, data.cv, K=10, cv.lim=10, jack.knife=FALSE)
\end{verbatim}
\end{Usage}
%
\begin{Arguments}
\begin{ldescription}
\item[\code{gbm.obj}] A calibrated GBM object with a binomial error distribution. Attention: users have to tune model input parameters according to their study!
\item[\code{data.cv}] A dataframe object containing the calibration data set with the same names for response and predictor variables.
\item[\code{K}] Number of folds. 10 is recommended; 5 for small data sets.
\item[\code{cv.lim}] Minimum number of presences required to perform the K-fold cross-validation.
\item[\code{jack.knife}] If TRUE, then the leave-one-out / jacknife cross-validation is performed instead of the 10-fold cross-validation.
\end{ldescription}
\end{Arguments}
%
\begin{Details}\relax
This function takes a calibrated GBM object with a binomial error distribution and returns predictions from a stratified 10-fold cross-validation or a leave-one-out / jack-knived cross-validation. Stratified means that the original prevalence of the presences and absences in the full dataset is conserved in each fold.
\end{Details}
%
\begin{Value}
Returns a dataframe with the observations (obs) and the corresponding predictions by cross-validation or jacknife.
\end{Value}
%
\begin{Author}\relax
Christophe Randin \email{christophe.randin@unibas.ch} and Antoine Guisan \email{antoine.guisan@unil.ch}
\end{Author}
%
\begin{References}\relax
Randin, C.F., T. Dirnbock, S. Dullinger, N.E. Zimmermann, M. Zappa and A. Guisan. 2006. Are niche-based species distribution models transferable in space? \emph{Journal of Biogeography}, \bold{33}, 1689-1703.


Pearman, P.B., C.F. Randin, O. Broennimann, P. Vittoz, W.O. van der Knaap, R. Engler, G. Le Lay, N.E. Zimmermann and A. Guisan. 2008. Prediction of plant species distributions across six millennia. \emph{Ecology Letters}, \bold{11}, 357-369.
\end{References}
%
\begin{Examples}
\begin{ExampleCode}
## Not run: 
gbm <- ecospat.cv.gbm (gbm.obj= get ("gbm.Agrostis_capillaris", envir=ecospat.env), 
ecospat.testData, K=10, cv.lim=10, jack.knife=FALSE)

## End(Not run)
\end{ExampleCode}
\end{Examples}
\inputencoding{utf8}
\HeaderA{ecospat.cv.glm}{GLM Cross Validation}{ecospat.cv.glm}
%
\begin{Description}\relax
K-fold and leave-one-out cross validation for GLM.
\end{Description}
%
\begin{Usage}
\begin{verbatim}
ecospat.cv.glm (glm.obj, K=10, cv.lim=10, jack.knife=FALSE)
\end{verbatim}
\end{Usage}
%
\begin{Arguments}
\begin{ldescription}
\item[\code{glm.obj}] Any calibrated GLM object with a binomial error distribution.
\item[\code{K}] Number of folds. 10 is recommended; 5 for small data sets.
\item[\code{cv.lim}] Minimum number of presences required to perform the K-fold cross-validation.
\item[\code{jack.knife}] If TRUE, then the leave-one-out / jacknife cross-validation is performed instead of the 10-fold cross-validation.
\end{ldescription}
\end{Arguments}
%
\begin{Details}\relax
This function takes a calibrated GLM object with a binomial error distribution and returns predictions from a stratified 10-fold cross-validation or a leave-one-out / jack-knived cross-validation. Stratified means that the original prevalence of the presences and absences in the full dataset is conserved in each fold.
\end{Details}
%
\begin{Value}
Returns a dataframe with the observations (obs) and the corresponding predictions by cross-validation or jacknife.
\end{Value}
%
\begin{Author}\relax
Christophe Randin \email{christophe.randin@unibas.ch} and Antoine Guisan \email{antoine.guisan@unil.ch}
\end{Author}
%
\begin{References}\relax
Randin, C.F., T. Dirnbock, S. Dullinger, N.E. Zimmermann, M. Zappa and A. Guisan. 2006. Are niche-based species distribution models transferable in space? \emph{Journal of Biogeography}, \bold{33}, 1689-1703.

Pearman, P.B., C.F. Randin, O. Broennimann, P. Vittoz, W.O. van der Knaap, R. Engler, G. Le Lay, N.E. Zimmermann and A. Guisan. 2008. Prediction of plant species distributions across six millennia. \emph{Ecology Letters}, \bold{11}, 357-369.
\end{References}
%
\begin{Examples}
\begin{ExampleCode}
## Not run: 
glm <- ecospat.cv.glm (glm.obj = get ("glm.Agrostis_capillaris", envir=ecospat.env), 
K=10, cv.lim=10, jack.knife=FALSE)

## End(Not run)
\end{ExampleCode}
\end{Examples}
\inputencoding{utf8}
\HeaderA{ecospat.cv.me}{Maxent Cross Validation}{ecospat.cv.me}
%
\begin{Description}\relax
K-fold and leave-one-out cross validation for Maxent.
\end{Description}
%
\begin{Usage}
\begin{verbatim}
ecospat.cv.me (data.cv.me, name.sp, names.pred, K=10, cv.lim=10, jack.knife=FALSE)
\end{verbatim}
\end{Usage}
%
\begin{Arguments}
\begin{ldescription}
\item[\code{data.cv.me}] A dataframe object containing the calibration data set of a Maxent object to validate with the same names for response and predictor variables.
\item[\code{name.sp}] Name of the species / response variable.
\item[\code{names.pred}] Names of the predicting variables.
\item[\code{K}] Number of folds. 10 is recommended; 5 for small data sets.
\item[\code{cv.lim}] Minimum number of presences required to perform the K-fold cross-validation.
\item[\code{jack.knife}] If TRUE, then the leave-one-out / jacknife cross-validation is performed instead of the 10-fold cross-validation.
\end{ldescription}
\end{Arguments}
%
\begin{Details}\relax
This function takes a calibrated Maxent object with a binomial error distribution and returns predictions from a stratified 10-fold cross-validation or a leave-one-out / jack-knived cross-validation. Stratified means that the original prevalence of the presences and absences in the full dataset is conserved in each fold.
\end{Details}
%
\begin{Value}
Returns a dataframe with the observations (obs) and the corresponding predictions by cross-validation or jacknife.
\end{Value}
%
\begin{Author}\relax
Christophe Randin \email{christophe.randin@unibas.ch} and Antoine Guisan \email{antoine.guisan@unil.ch}
\end{Author}
%
\begin{References}\relax
Randin, C.F., T. Dirnbock, S. Dullinger, N.E. Zimmermann, M. Zappa and A. Guisan. 2006. Are niche-based species distribution models transferable in space? \emph{Journal of Biogeography}, \bold{33}, 1689-1703.


Pearman, P.B., C.F. Randin, O. Broennimann, P. Vittoz, W.O. van der Knaap, R. Engler, G. Le Lay, N.E. Zimmermann and A. Guisan. 2008. Prediction of plant species distributions across six millennia. \emph{Ecology Letters}, \bold{11}, 357-369.
\end{References}
%
\begin{Examples}
\begin{ExampleCode}

## Not run: 
me <- ecospat.cv.me(ecospat.testData, names(ecospat.testData)[53], 
names(ecospat.testData)[4:8], K = 10, cv.lim = 10, jack.knife = FALSE)

## End(Not run)
\end{ExampleCode}
\end{Examples}
\inputencoding{utf8}
\HeaderA{ecospat.cv.rf}{RandomForest Cross Validation}{ecospat.cv.rf}
%
\begin{Description}\relax
K-fold and leave-one-out cross validation for randomForest.
\end{Description}
%
\begin{Usage}
\begin{verbatim}
ecospat.cv.rf (rf.obj, data.cv, K=10, cv.lim=10, jack.knife=FALSE)
\end{verbatim}
\end{Usage}
%
\begin{Arguments}
\begin{ldescription}
\item[\code{rf.obj}] Any calibrated randomForest object with a binomial error distribution.
\item[\code{data.cv}] A dataframe object containing the calibration data set with the same names for response and predictor variables.
\item[\code{K}] Number of folds. 10 is recommended; 5 for small data sets.
\item[\code{cv.lim}] Minimum number of presences required to perform the K-fold cross-validation.
\item[\code{jack.knife}] If TRUE, then the leave-one-out / jacknife cross-validation is performed instead of the 10-fold cross-validation.
\end{ldescription}
\end{Arguments}
%
\begin{Details}\relax
This function takes a calibrated randomForest object with a binomial error distribution and returns predictions from a stratified 10-fold cross-validation or a leave-one-out / jack-knived cross-validation. Stratified means that the original prevalence of the presences and absences in the full dataset is conserved in each fold.
\end{Details}
%
\begin{Value}
Returns a dataframe with the observations (obs) and the corresponding predictions by cross-validation or jacknife.
\end{Value}
%
\begin{Author}\relax
Christophe Randin \email{christophe.randin@wsl.ch} and Antoine Guisan \email{antoine.guisan@unil.ch}
\end{Author}
%
\begin{References}\relax
Randin, C.F., T. Dirnbock, S. Dullinger, N.E. Zimmermann, M. Zappa and A. Guisan. 2006. Are niche-based species distribution models transferable in space? \emph{Journal of Biogeography}, \bold{33}, 1689-1703.


Pearman, P.B., C.F. Randin, O. Broennimann, P. Vittoz, W.O. van der Knaap, R. Engler, G. Le Lay, N.E. Zimmermann and A. Guisan. 2008. Prediction of plant species distributions across six millennia. \emph{Ecology Letters}, \bold{11}, 357-369.
\end{References}
%
\begin{Examples}
\begin{ExampleCode}
## Not run: 
rf <- ecospat.cv.rf(get("rf.Agrostis_capillaris", envir = ecospat.env), 
ecospat.testData[, c(53, 4:8)], K = 10, cv.lim = 10, jack.knife = FALSE)

## End(Not run)
\end{ExampleCode}
\end{Examples}
\inputencoding{utf8}
\HeaderA{ecospat.env}{Package Environment}{ecospat.env}
%
\begin{Description}\relax
A package environment that is used to contain certain (local) variables and results, especially those in example functions and data sets.
\end{Description}
%
\begin{Examples}
\begin{ExampleCode}
ls(envir=ecospat.env)
\end{ExampleCode}
\end{Examples}
\inputencoding{utf8}
\HeaderA{ecospat.Epred}{Prediction Mean}{ecospat.Epred}
%
\begin{Description}\relax
Calculate the mean (or weighted mean) of several predictions.
\end{Description}
%
\begin{Usage}
\begin{verbatim}
ecospat.Epred (x, w=rep(1,ncol(x)), th=0)
\end{verbatim}
\end{Usage}
%
\begin{Arguments}
\begin{ldescription}
\item[\code{x}] A dataframe object with SDM predictions.
\item[\code{w}] Weight of the model, e.g. AUC. The default is 1.
\item[\code{th}] Threshold used to binarize.
\end{ldescription}
\end{Arguments}
%
\begin{Details}\relax
The Weighted Average consensus method utilizes pre-evaluation of the predictive performance of the single-models. In this approach, half (i.e. four) of the eight single-models with highest accuracy are selected first, and then a WA is calculated based on the pre-evaluated AUC of the single-models
\end{Details}
%
\begin{Value}
A weighted mean binary transformation of the models.
\end{Value}
%
\begin{Author}\relax
Blaise Petitpierre \email{bpetitpierre@gmail.com}
\end{Author}
%
\begin{References}\relax

Boyce, M.S.,  P.R. Vernier, S.E. Nielsen and  F.K.A. Schmiegelow. 2002. Evaluating resource selection functions. \emph{Ecol. Model.}, \bold{157}, 281-300.

Marmion, M., M. Parviainen, M. Luoto, R.K. Heikkinen andW.  Thuiller. 2009. Evaluation of consensus methods in predictive species distribution modelling.  \emph{Diversity and Distributions}, \bold{15}, 59-69.

\end{References}
%
\begin{Examples}
\begin{ExampleCode}
x <- ecospat.testData[c(92,96)]
mean <- ecospat.Epred (x, w=rep(1,ncol(x)), th=0.5)
\end{ExampleCode}
\end{Examples}
\inputencoding{utf8}
\HeaderA{ecospat.ESM.EnsembleModeling}{Ensamble of Small Models: Evaluates and Averages Simple Bivariate Models To ESMs}{ecospat.ESM.EnsembleModeling}
%
\begin{Description}\relax
This function evaluates and averages simple bivariate models by weighted means to Ensemble Small Models as in Lomba et al. 2010 and Breiner et al. 2015.

\end{Description}
%
\begin{Usage}
\begin{verbatim}
    ecospat.ESM.EnsembleModeling( ESM.modeling.output, 
                                  weighting.score, 
                                  threshold=NULL, 
                                  models)
\end{verbatim}
\end{Usage}
%
\begin{Arguments}
\begin{ldescription}
\item[\code{ESM.modeling.output}] a \code{list} returned by \code{\LinkA{ecospat.ESM.Modeling}{ecospat.ESM.Modeling}}

\item[\code{weighting.score}] an evaluation score used to weight single models to build ensembles:"AUC","TSS",

"Boyce","Kappa","SomersD" \#the evaluation methods used to evaluate ensemble models 

( see \code{"\LinkA{BIOMOD\_Modeling}{BIOMOD.Rul.Modeling}"}  \bold{models.eval.meth} section for more detailed informations )

\item[\code{threshold}] 
threshold value of an evaluation score to select the bivariate model(s) included for building the ESMs

\item[\code{models}] 
vector of models names choosen among 'GLM', 'GBM', 'GAM', 'CTA', 'ANN', 'SRE', 'FDA', 'MARS', 'RF','MAXENT.Phillips', "MAXENT.Tsuruoka" (same as in \code{biomod2})

\#a character vector (either 'all' or a sub-selection of model names) that defines the models kept for building the ensemble models (might be useful for removing some non-preferred models)


\end{ldescription}
\end{Arguments}
%
\begin{Details}\relax
The basic idea of ensemble of small models (ESMs) is to model a species distribution based on small, simple models, for example all possible bivariate models (i.e. models that contain only two predictors at a time out of a larger set of predictors), and then combine all possible bivariate models into an ensemble (Lomba et al. 2010; Breiner et al. 2015).

The ESM set of functions could be used to build ESMs using simple bivariate models which are averaged using weights based on model performances (e.g. AUC) according to Breiner et al. (2015). They provide full functionality of the approach described in Breiner et al. (2015).


\end{Details}
%
\begin{Value}

species:          species name
ESM.fit:          data.frame of the predicted values for the data used to build the models.
ESM.evaluations:  data.frame with evaluations scores for the ESMs
weights:          weighting scores used to weight the bivariate models to build the single ESM
weights.EF:       weighting scores used to weight the single ESM to build the ensemble of ESMs from different modelling techniques (only available if >1 modelling techniques were selected).
failed:           bivariate models which failed because they could not be calibrated.


A \code{"\LinkA{BIOMOD.EnsembleModeling.out}{BIOMOD.EnsembleModeling.out.Rdash.class}"}. This object will be later given to \code{\LinkA{ecospat.ESM.EnsembleProjection}{ecospat.ESM.EnsembleProjection}} if you want to make some projections of this ensemble-models.

\end{Value}
%
\begin{Author}\relax
Frank Breiner \email{frank.breiner@wsl.ch} 

with contributions of Olivier Broennimann \email{olivier.broennimann@unil.ch}

\end{Author}
%
\begin{References}\relax

Lomba, A., L. Pellissier, C.F. Randin, J. Vicente, F. Moreira, J. Honrado and A. Guisan. 2010. Overcoming the rare species modelling paradox: A novel hierarchical framework applied to an Iberian endemic plant. \emph{Biological Conservation}, \bold{143},2647-2657.
Breiner F.T., A. Guisan, A. Bergamini and M.P. Nobis. 2015. Overcoming limitations of modelling rare species by using ensembles of small models. \emph{Methods in Ecology and Evolution}, \bold{6},1210-1218.
Breiner F.T., Nobis M.P., Bergamini A., Guisan A. 2018. Optimizing ensembles of small models for predicting the distribution of species with few occurrences. \emph{Methods in Ecology and Evolution}. \LinkA{https://doi.org/10.1111/2041-210X.12957}{https://doi.org/10.1111/2041.Rdash.210X.12957}
\end{References}
%
\begin{SeeAlso}\relax
\code{\LinkA{ecospat.ESM.Modeling}{ecospat.ESM.Modeling}}, \code{\LinkA{ecospat.ESM.Projection}{ecospat.ESM.Projection}}, \code{\LinkA{ecospat.ESM.EnsembleProjection}{ecospat.ESM.EnsembleProjection}}

\code{\LinkA{BIOMOD\_Modeling}{BIOMOD.Rul.Modeling}}, \code{\LinkA{BIOMOD\_Projection}{BIOMOD.Rul.Projection}}

\end{SeeAlso}
%
\begin{Examples}
\begin{ExampleCode}
   ## Not run: 
# Loading test data
inv <- ecospat.testNiche.inv

# species occurrences
xy <- inv[,1:2]
sp_occ <- inv[11]

# env
current <- inv[3:10]



### Formating the data with the BIOMOD_FormatingData() function from the package biomod2

sp <- 1
myBiomodData <- BIOMOD_FormatingData( resp.var = as.numeric(sp_occ[,sp]),
                                      expl.var = current,
                                      resp.xy = xy,
                                      resp.name = colnames(sp_occ)[sp])


### Calibration of simple bivariate models
my.ESM <- ecospat.ESM.Modeling( data=myBiomodData,
                                models=c('GLM','RF'),
                                NbRunEval=2,
                                DataSplit=70,
                                weighting.score=c("AUC"),
                                parallel=FALSE)  


### Evaluation and average of simple bivariate models to ESMs
my.ESM_EF <- ecospat.ESM.EnsembleModeling(my.ESM,weighting.score=c("SomersD"),threshold=0)

### Projection of simple bivariate models into new space 
my.ESM_proj_current<-ecospat.ESM.Projection(ESM.modeling.output=my.ESM,
                                            new.env=current)

### Projection of calibrated ESMs into new space 
my.ESM_EFproj_current <- ecospat.ESM.EnsembleProjection(ESM.prediction.output=my.ESM_proj_current,
                                                        ESM.EnsembleModeling.output=my.ESM_EF)

## get the model performance of ESMs 
my.ESM_EF$ESM.evaluations
## get the weights of the single bivariate models used to build the ESMs
my.ESM_EF$weights

## End(Not run)
\end{ExampleCode}
\end{Examples}
\inputencoding{utf8}
\HeaderA{ecospat.ESM.EnsembleProjection}{Ensamble of Small Models: Projects Calibrated ESMs Into New Space Or Time.}{ecospat.ESM.EnsembleProjection}
%
\begin{Description}\relax
This function projects calibrated ESMs into new space or time.

\end{Description}
%
\begin{Usage}
\begin{verbatim}
    ecospat.ESM.EnsembleProjection( ESM.prediction.output, 
                                    ESM.EnsembleModeling.output,
                                    chosen.models = 'all')
\end{verbatim}
\end{Usage}
%
\begin{Arguments}
\begin{ldescription}
\item[\code{ESM.prediction.output}] a list object returned by \code{\LinkA{ecospat.ESM.Projection}{ecospat.ESM.Projection}}


\item[\code{ESM.EnsembleModeling.output}] 
a list object returned by \code{\LinkA{ecospat.ESM.EnsembleModeling}{ecospat.ESM.EnsembleModeling}}
\item[\code{chosen.models}] a character vector (either 'all' or a sub-selection of model names) to remove models from the ensemble (same as in \code{biomod2}). Default is 'all'.

\end{ldescription}
\end{Arguments}
%
\begin{Details}\relax
The basic idea of ensemble of small models (ESMs) is to model a species distribution based on small, simple models, for example all possible bivariate models (i.e. models that contain only two predictors at a time out of a larger set of predictors), and then combine all possible bivariate models into an ensemble (Lomba et al. 2010; Breiner et al. 2015).

The ESM set of functions could be used to build ESMs using simple bivariate models which are averaged using weights based on model performances (e.g. AUC) according to Breiner et al. (2015). They provide full functionality of the approach described in Breiner et al. (2015).

For further details please refer to \code{\LinkA{BIOMOD\_EnsembleForecasting}{BIOMOD.Rul.EnsembleForecasting}}.

\end{Details}
%
\begin{Value}
Returns the projections of ESMs for the selected single models and their ensemble (data frame or raster stack). ESM.projections `projection files' are saved on the hard drive projection folder. This files are either an \code{array} or a \code{RasterStack} depending the original projections data type.
Load these created files to plot and work with them.

\end{Value}
%
\begin{Author}\relax
Frank Breiner \email{frank.breiner@wsl.ch}

\end{Author}
%
\begin{References}\relax

Lomba, A., L. Pellissier, C.F. Randin, J. Vicente, F. Moreira, J. Honrado and A. Guisan. 2010. Overcoming the rare species modelling paradox: A novel hierarchical framework applied to an Iberian endemic plant. \emph{Biological Conservation}, \bold{143},2647-2657.
Breiner F.T., A. Guisan, A. Bergamini and M.P. Nobis. 2015. Overcoming limitations of modelling rare species by using ensembles of small models. \emph{Methods in Ecology and Evolution}, \bold{6},1210-1218.
Breiner F.T., Nobis M.P., Bergamini A., Guisan A. 2018. Optimizing ensembles of small models for predicting the distribution of species with few occurrences. \emph{Methods in Ecology and Evolution}. \LinkA{https://doi.org/10.1111/2041-210X.12957}{https://doi.org/10.1111/2041.Rdash.210X.12957}
\end{References}
%
\begin{SeeAlso}\relax
\code{\LinkA{ecospat.ESM.Modeling}{ecospat.ESM.Modeling}}, \code{\LinkA{ecospat.ESM.Projection}{ecospat.ESM.Projection}}, \code{\LinkA{ecospat.ESM.EnsembleModeling}{ecospat.ESM.EnsembleModeling}}

\code{\LinkA{BIOMOD\_Modeling}{BIOMOD.Rul.Modeling}}, \code{\LinkA{BIOMOD\_Projection}{BIOMOD.Rul.Projection}}, \code{\LinkA{BIOMOD\_EnsembleForecasting}{BIOMOD.Rul.EnsembleForecasting}}, \code{\LinkA{BIOMOD\_EnsembleModeling}{BIOMOD.Rul.EnsembleModeling}}

\end{SeeAlso}
%
\begin{Examples}
\begin{ExampleCode}
   ## Not run: 
# Loading test data for the niche dynamics analysis in the invaded range
inv <- ecospat.testNiche.inv

# species occurrences
xy <- inv[,1:2]
sp_occ <- inv[11]

# env
current <- inv[3:10]



### Formating the data with the BIOMOD_FormatingData() function form the package biomod2
setwd(path.wd)
t1 <- Sys.time()
sp <- 1
myBiomodData <- BIOMOD_FormatingData( resp.var = as.numeric(sp_occ[,sp]),
                                      expl.var = current,
                                      resp.xy = xy,
                                      resp.name = colnames(sp_occ)[sp])

myBiomodOption <- Print_Default_ModelingOptions()


### Calibration of simple bivariate models
my.ESM <- ecospat.ESM.Modeling( data=myBiomodData,
                                models=c('GLM','RF'),
                                models.options=myBiomodOption,
                                NbRunEval=2,
                                DataSplit=70,
                                weighting.score=c("AUC"),
                                parallel=FALSE)  


### Evaluation and average of simple bivariate models to ESMs
my.ESM_EF <- ecospat.ESM.EnsembleModeling(my.ESM,weighting.score=c("SomersD"),threshold=0)

### Projection of simple bivariate models into new space 
my.ESM_proj_current<-ecospat.ESM.Projection(ESM.modeling.output=my.ESM,
                                            new.env=current)

### Projection of calibrated ESMs into new space 
my.ESM_EFproj_current <- ecospat.ESM.EnsembleProjection(ESM.prediction.output=my.ESM_proj_current,
                                                        ESM.EnsembleModeling.output=my.ESM_EF)

## get the model performance of ESMs 
my.ESM_EF$ESM.evaluations
## get the weights of the single bivariate models used to build the ESMs
my.ESM_EF$weights

## End(Not run)
\end{ExampleCode}
\end{Examples}
\inputencoding{utf8}
\HeaderA{ecospat.ESM.Modeling}{Ensamble of Small Models: Calibration of Simple Bivariate Models}{ecospat.ESM.Modeling}
%
\begin{Description}\relax
This function calibrates simple bivariate models as in Lomba et al. 2010 and Breiner et al. 2015.
\end{Description}
%
\begin{Usage}
\begin{verbatim}
    ecospat.ESM.Modeling( data, 
                          NbRunEval, 
                          DataSplit, 
                          DataSplitTable, 
                          Prevalence,
                          weighting.score, 
                          models, 
                          tune,
                          modeling.id, 
                          models.options, 
                          which.biva, 
                          parallel, 
                          cleanup)
\end{verbatim}
\end{Usage}
%
\begin{Arguments}
\begin{ldescription}
\item[\code{data}] \code{BIOMOD.formated.data} object returned by \code{\LinkA{BIOMOD\_FormatingData}{BIOMOD.Rul.FormatingData}}

\item[\code{NbRunEval}] 
number of dataset splitting replicates for the model evaluation (same as in \code{biomod2})

\item[\code{DataSplit}] 
percentage of dataset observations retained for the model training (same as in \code{biomod2})

\item[\code{DataSplitTable}] 
a matrix, data.frame or a 3D array filled with TRUE/FALSE to specify which part of data must be used for models calibration (TRUE) and for models validation (FALSE). Each column corresponds to a 'RUN'. If filled, arguments NbRunEval and DataSplit  will be ignored.

\item[\code{Prevalence}] 
either NULL or a 0-1 numeric used to build 'weighted response weights'. In contrast to Biomod the default is 0.5 (weighting presences equally to the absences). If NULL each observation (presence or absence) has the same weight (independent of the number of presences and absences).

\item[\code{weighting.score}] 
evaluation score used to weight single models to build ensembles: 'AUC', 'SomersD' (2xAUC-1), 'Kappa', 'TSS' or 'Boyce'

\item[\code{models}] 
vector of models names choosen among 'GLM', 'GBM', 'GAM', 'CTA', 'ANN', 'SRE', 'FDA', 'MARS', 'RF','MAXENT.Phillips', 'MAXENT.Tsuruoka' (same as in \code{biomod2})

\item[\code{tune}] 
logical. if true model tuning will be used to estimate optimal parameters for the models (Default: False).
   
\item[\code{modeling.id}] 
character, the ID (=name) of modeling procedure. A random number by default.

\item[\code{models.options}] 
BIOMOD.models.options object returned by BIOMOD\_ModelingOptions (same as in \code{biomod2})

\item[\code{which.biva}] 
integer. which bivariate combinations should be used for modeling? Default: all

\item[\code{parallel}] 
logical. If TRUE, the parallel computing is enabled (highly recommended)

\item[\code{cleanup}] 
numeric. Calls removeTmpFiles() to delete all files from rasterOptions()\$tmpdir which are older than the given time (in hours). This might be necessary to prevent running over quota. No cleanup is used by default. 

\end{ldescription}
\end{Arguments}
%
\begin{Details}\relax
The basic idea of ensemble of small models (ESMs) is to model a species distribution based on small, simple models, for example all possible bivariate models (i.e. models that contain only two predictors at a time out of a larger set of predictors), and then combine all possible bivariate models into an ensemble (Lomba et al. 2010; Breiner et al. 2015).

The ESM set of functions could be used to build ESMs using simple bivariate models which are averaged using weights based on model performances (e.g. AUC) according to Breiner et al. (2015). They provide full functionality of the approach described in Breiner et al. (2015).

The argument \code{which.biva} allows to split model runs, e.g. if \code{which.biva} is 1:3, only the three first bivariate variable combinations will be modeled. This allows to run different biva splits on different computers. However, it is better not to use this option if all models are run on a single computer.
Default: running all biva models. 
NOTE: Make sure to give each of your biva runs a unique \code{modeling.id}.
\end{Details}
%
\begin{Value}
A BIOMOD.models.out object (same as in \code{biomod2})
See \code{"\LinkA{BIOMOD.models.out}{BIOMOD.models.out.Rdash.class}"} for details.

\end{Value}
%
\begin{Author}\relax
Frank Breiner \email{frank.breiner@wsl.ch} and Mirko Di Febbraro \email{mirkodifebbraro@gmail.com} with contributions of Olivier Broennimann \email{olivier.broennimann@unil.ch}
\end{Author}
%
\begin{References}\relax
Lomba, A., L. Pellissier, C.F. Randin, J. Vicente, F. Moreira, J. Honrado and A. Guisan. 2010. Overcoming the rare species modelling paradox: A novel hierarchical framework applied to an Iberian endemic plant. 
\emph{Biological Conservation}, \bold{143},2647-2657.
Breiner F.T., A. Guisan, A. Bergamini and M.P. Nobis. 2015. Overcoming limitations of modelling rare species by using ensembles of small models. \emph{Methods in Ecology and Evolution}, \bold{6},1210-1218.
Breiner F.T., Nobis M.P., Bergamini A., Guisan A. 2018. Optimizing ensembles of small models for predicting the distribution of species with few occurrences. \emph{Methods in Ecology and Evolution}. \LinkA{https://doi.org/10.1111/2041-210X.12957}{https://doi.org/10.1111/2041.Rdash.210X.12957}

\end{References}
%
\begin{SeeAlso}\relax
\code{\LinkA{ecospat.ESM.EnsembleModeling}{ecospat.ESM.EnsembleModeling}}, \code{\LinkA{ecospat.ESM.Projection}{ecospat.ESM.Projection}}, \code{\LinkA{ecospat.ESM.EnsembleProjection}{ecospat.ESM.EnsembleProjection}}

\code{\LinkA{BIOMOD\_FormatingData}{BIOMOD.Rul.FormatingData}}, \code{\LinkA{BIOMOD\_ModelingOptions}{BIOMOD.Rul.ModelingOptions}}, \code{\LinkA{BIOMOD\_Modeling}{BIOMOD.Rul.Modeling}},\code{\LinkA{BIOMOD\_Projection}{BIOMOD.Rul.Projection}}

\end{SeeAlso}
%
\begin{Examples}
\begin{ExampleCode}
   ## Not run: 
# Loading test data
inv <- ecospat.testNiche.inv

# species occurrences
xy <- inv[,1:2]
sp_occ <- inv[11]

# env
current <- inv[3:10]



### Formating the data with the BIOMOD_FormatingData() function from the package biomod2
sp <- 1
myBiomodData <- BIOMOD_FormatingData( resp.var = as.numeric(sp_occ[,sp]),
                                      expl.var = current,
                                      resp.xy = xy,
                                      resp.name = colnames(sp_occ)[sp])

### Calibration of simple bivariate models
my.ESM <- ecospat.ESM.Modeling( data=myBiomodData,
                                models=c('GLM','RF'),
                                NbRunEval=2,
                                DataSplit=70,
                                Prevalence=0.5
                                weighting.score=c("AUC"),
                                parallel=FALSE)  


### Evaluation and average of simple bivariate models to ESMs
my.ESM_EF <- ecospat.ESM.EnsembleModeling(my.ESM,weighting.score=c("SomersD"),threshold=0)

### Projection of simple bivariate models into new space 
my.ESM_proj_current<-ecospat.ESM.Projection(ESM.modeling.output=my.ESM,
                                            new.env=current)

### Projection of calibrated ESMs into new space 
my.ESM_EFproj_current <- ecospat.ESM.EnsembleProjection(ESM.prediction.output=my.ESM_proj_current,
                                                        ESM.EnsembleModeling.output=my.ESM_EF)

## get the model performance of ESMs 
my.ESM_EF$ESM.evaluations
## get the weights of the single bivariate models used to build the ESMs
my.ESM_EF$weights

## End(Not run)
\end{ExampleCode}
\end{Examples}
\inputencoding{utf8}
\HeaderA{ecospat.ESM.Projection}{Ensamble of Small Models: Projects Simple Bivariate Models Into New Space Or Time}{ecospat.ESM.Projection}
%
\begin{Description}\relax
This function projects simple bivariate models on new.env
\end{Description}
%
\begin{Usage}
\begin{verbatim}
    ecospat.ESM.Projection(ESM.modeling.output, 
                           new.env, 
                           parallel,
                           cleanup)
\end{verbatim}
\end{Usage}
%
\begin{Arguments}
\begin{ldescription}
\item[\code{ESM.modeling.output}] 
\code{list} object returned by \code{\LinkA{ecospat.ESM.Modeling}{ecospat.ESM.Modeling}}

\item[\code{new.env}] 
A set of explanatory variables onto which models will be projected. It could be a \code{data.frame}, a \code{matrix}, or a \code{rasterStack} object. Make sure the column names (\code{data.frame} or \code{matrix}) or layer Names (\code{rasterStack}) perfectly match with the names of variables used to build the models in the previous steps.

\item[\code{parallel}] 
Logical. If TRUE, the parallel computing is enabled
\item[\code{cleanup}] 
Numeric. Calls removeTmpFiles() to delete all files from rasterOptions()\$tmpdir which are older than the given time (in hours). This might be necessary to prevent running over quota. No cleanup is used by default

\end{ldescription}
\end{Arguments}
%
\begin{Details}\relax
The basic idea of ensemble of small models (ESMs) is to model a species distribution based on small, simple models, for example all possible bivariate models (i.e. models that contain only two predictors at a time out of a larger set of predictors), and then combine all possible bivariate models into an ensemble (Lomba et al. 2010; Breiner et al. 2015).

The ESM set of functions could be used to build ESMs using simple bivariate models which are averaged using weights based on model performances (e.g. AUC) accoring to Breiner et al (2015). They provide full functionality of the approach described in Breiner et al. (2015).

The name of \code{new.env} must be a regular expression (see ?regex)
\end{Details}
%
\begin{Value}
Returns the projections for all selected models (same as in \code{biomod2})
See \code{"\LinkA{BIOMOD.projection.out}{BIOMOD.projection.out.Rdash.class}"} for details.

\end{Value}
%
\begin{Author}\relax
 Frank Breiner \email{frank.breiner@wsl.ch} 

with contributions of Olivier Broennimann \email{olivier.broennimann@unil.ch}
\end{Author}
%
\begin{References}\relax

Lomba, A., L. Pellissier, C.F. Randin, J. Vicente, F. Moreira, J. Honrado and A. Guisan. 2010. Overcoming the rare species modelling paradox: A novel hierarchical framework applied to an Iberian endemic plant. \emph{Biological Conservation}, \bold{143},2647-2657.
Breiner F.T., A. Guisan, A. Bergamini and M.P. Nobis. 2015. Overcoming limitations of modelling rare species by using ensembles of small models. \emph{Methods in Ecology and Evolution}, \bold{6},1210-1218.
Breiner F.T., Nobis M.P., Bergamini A., Guisan A. 2018. Optimizing ensembles of small models for predicting the distribution of species with few occurrences. \emph{Methods in Ecology and Evolution}. \LinkA{https://doi.org/10.1111/2041-210X.12957}{https://doi.org/10.1111/2041.Rdash.210X.12957}

\end{References}
%
\begin{SeeAlso}\relax
\code{\LinkA{ecospat.ESM.EnsembleModeling}{ecospat.ESM.EnsembleModeling}}, \code{\LinkA{ecospat.ESM.Modeling}{ecospat.ESM.Modeling}}, \code{\LinkA{ecospat.ESM.EnsembleProjection}{ecospat.ESM.EnsembleProjection}}

\code{\LinkA{BIOMOD\_FormatingData}{BIOMOD.Rul.FormatingData}}, \code{\LinkA{BIOMOD\_ModelingOptions}{BIOMOD.Rul.ModelingOptions}}, \code{\LinkA{BIOMOD\_Modeling}{BIOMOD.Rul.Modeling}},\code{\LinkA{BIOMOD\_Projection}{BIOMOD.Rul.Projection}}

\end{SeeAlso}
%
\begin{Examples}
\begin{ExampleCode}
   ## Not run: 
# Loading test data
inv <- ecospat.testNiche.inv

# species occurrences
xy <- inv[,1:2]
sp_occ <- inv[11]

# env
current <- inv[3:10]



### Formating the data with the BIOMOD_FormatingData() function from the package biomod2
       
sp <- 1
myBiomodData <- BIOMOD_FormatingData( resp.var = as.numeric(sp_occ[,sp]),
                                      expl.var = current,
                                      resp.xy = xy,
                                      resp.name = colnames(sp_occ)[sp])

### Calibration of simple bivariate models
my.ESM <- ecospat.ESM.Modeling( data=myBiomodData,
                                models=c('GLM','RF'),
                                NbRunEval=2,
                                DataSplit=70,
                                weighting.score=c("AUC"),
                                parallel=FALSE)  


### Evaluation and average of simple bivariate models to ESMs
my.ESM_EF <- ecospat.ESM.EnsembleModeling(my.ESM,weighting.score=c("SomersD"),threshold=0)

### Projection of simple bivariate models into new space 
my.ESM_proj_current<-ecospat.ESM.Projection(ESM.modeling.output=my.ESM,
                                            new.env=current)

### Projection of calibrated ESMs into new space 
my.ESM_EFproj_current <- ecospat.ESM.EnsembleProjection(ESM.prediction.output=my.ESM_proj_current,
                                                        ESM.EnsembleModeling.output=my.ESM_EF)

## get the model performance of ESMs 
my.ESM_EF$ESM.evaluations
## get the weights of the single bivariate models used to build the ESMs
my.ESM_EF$weights

## End(Not run)
\end{ExampleCode}
\end{Examples}
\inputencoding{utf8}
\HeaderA{ecospat.grid.clim.dyn}{Dynamic Occurrence Densities Grid}{ecospat.grid.clim.dyn}
%
\begin{Description}\relax
Create a grid with occurrence densities along one or two gridded environmental gradients.
\end{Description}
%
\begin{Usage}
\begin{verbatim}
ecospat.grid.clim.dyn (glob, glob1, sp, R, th.sp, th.env, geomask)
\end{verbatim}
\end{Usage}
%
\begin{Arguments}
\begin{ldescription}
\item[\code{glob}] A two-column dataframe (or a vector) of the environmental values (in column) for background pixels of the whole study area (in row).
\item[\code{glob1}] A two-column dataframe (or a vector) of the environmental values (in column) for the background pixels of the species (in row).
\item[\code{sp}] A two-column dataframe (or a vector) of the environmental values (in column) for the occurrences of the species (in row).
\item[\code{R}] The resolution of the grid.
\item[\code{th.sp}] The quantile used to delimit a threshold to exclude low species density values.
\item[\code{th.env}] The quantile used to delimit a threshold to exclude low environmental density values of the study area.
\item[\code{geomask}] A geographical mask to delimit the background extent if the analysis takes place in the geographical space.

It can be a SpatialPolygon or a raster object. Note that the CRS should be the same as the one used for the points.


\end{ldescription}
\end{Arguments}
%
\begin{Details}\relax
Using the scores of an ordination (or SDM prediction), create a grid z of RxR pixels (or a vector of R pixels when using scores of dimension 1 or SDM predictions) with occurrence densities. Only scores of one, or two dimensions can be used.
\code{th.sp} is the quantile of the distribution of species density at occurrence sites. 
For example, if \code{th.sp} is set to 0.05, the the species niche is drawn by including 95 percent of the species occurrences, removing the more marginal populations. 
Similarly, \code{th.env} is the quantile of the distribution of the environmental density at all sites of the study area. 
If \code{th.env} is set to 0.05, the delineation of the study area in the environmental space includes 95 percent  of the study area, removing the more marginal sites of the study area. 
By default, these thresholds are set to 0 but can be modified, depending on the importance of some marginal sites in the delineation of the species niche and/or the study area in the environmnental space. It is recommended to check if the shape of the delineated niche and study area corresponds to the shape of the plot of the PCA scores (or any other ordination techniques used to set the environmental space). 
Visualisation of the gridded environmental space can be done through the functions \code{\LinkA{ecospat.plot.niche}{ecospat.plot.niche}} or \code{\LinkA{ecospat.plot.niche.dyn}{ecospat.plot.niche.dyn}}
If you encounter a problem during your analyses, please first read the FAQ section of "Niche overlap" in http://www.unil.ch/ecospat/home/menuguid/ecospat-resources/tools.html
The argument \code{geomask} can be a SpatialPolygon or a raster object. 
Note that the CRS should be the same as the one used for the points.
\end{Details}
%
\begin{Value}
A grid z of RxR pixels (or a vector of R pixels) with z.uncor being the density of occurrence of the species, and z.cor the occupancy of the environment by the species (density of occurrences divided by the desinty of environment in the study area.
\end{Value}
%
\begin{Author}\relax
Olivier Broennimann \email{olivier.broennimann@unil.ch} and Blaise Petitpierre \email{bpetitpierre@gmail.com}
\end{Author}
%
\begin{References}\relax
Broennimann, O., M.C. Fitzpatrick, P.B. Pearman, B. Petitpierre, L. Pellissier, N.G. Yoccoz, W. Thuiller, M.J. Fortin, C. Randin, N.E. Zimmermann, C.H. Graham and A. Guisan. 2012. Measuring ecological niche overlap from occurrence and spatial environmental data. \emph{Global Ecology and Biogeography}, \bold{21}:481-497.


Petitpierre, B., C. Kueffer, O. Broennimann, C. Randin, C. Daehler and A. Guisan. 2012. Climatic niche shifts are rare among terrestrial plant invaders. \emph{Science}, \bold{335}:1344-1348.
\end{References}
%
\begin{SeeAlso}\relax
\code{\LinkA{ecospat.plot.niche.dyn}{ecospat.plot.niche.dyn}}
\end{SeeAlso}
%
\begin{Examples}
\begin{ExampleCode}
## Not run: 
spp <- ecospat.testNiche
clim <- ecospat.testData[2:8]

occ.sp_test <- na.exclude(ecospat.sample.envar(dfsp=spp,colspxy=2:3,colspkept=1:3,dfvar=clim,
colvarxy=1:2,colvar="all",resolution=25))

occ.sp<-cbind(occ.sp_test,spp[,4]) #add species names

# list of species
sp.list<-levels(occ.sp[,1])
sp.nbocc<-c()

for (i in 1:length(sp.list)){sp.nbocc<-c(sp.nbocc,length(which(occ.sp[,1] == sp.list[i])))} 
#calculate the nb of occurences per species

sp.list <- sp.list[sp.nbocc>4] # remove species with less than 5 occurences
nb.sp <- length(sp.list) #nb of species
ls()
# selection of variables to include in the analyses 
# try with all and then try only worldclim Variables
Xvar <- c(3:7)
nvar <- length(Xvar)

#number of interation for the tests of equivalency and similarity
iterations <- 100
#resolution of the gridding of the climate space
R <- 100
#################################### PCA-ENVIRONMENT ##################################
data<-rbind(occ.sp[,Xvar+1],clim[,Xvar]) 
w <- c(rep(0,nrow(occ.sp)),rep(1,nrow(clim)))
pca.cal <- dudi.pca(data, row.w = w, center = TRUE, scale = TRUE, scannf = FALSE, nf = 2)

####### selection of species ######
sp.list
sp.combn <- combn(1:2,2)

for(i in 1:ncol(sp.combn)) {
  row.sp1 <- which(occ.sp[,1] == sp.list[sp.combn[1,i]]) # rows in data corresponding to sp1
  row.sp2 <- which(occ.sp[,1] == sp.list[sp.combn[2,i]]) # rows in data corresponding to sp2
  name.sp1 <- sp.list[sp.combn[1,i]]
  name.sp2 <- sp.list[sp.combn[2,i]]
  # predict the scores on the axes
  scores.clim <- pca.cal$li[(nrow(occ.sp)+1):nrow(data),]  #scores for global climate
  scores.sp1 <- pca.cal$li[row.sp1,]					#scores for sp1
  scores.sp2 <- pca.cal$li[row.sp2,]					#scores for sp2
}
# calculation of occurence density and test of niche equivalency and similarity 
z1 <- ecospat.grid.clim.dyn(scores.clim, scores.clim, scores.sp1,R=100)
z2 <- ecospat.grid.clim.dyn(scores.clim, scores.clim, scores.sp2,R=100)

## End(Not run)
\end{ExampleCode}
\end{Examples}
\inputencoding{utf8}
\HeaderA{ecospat.makeDataFrame}{Make Data Frame}{ecospat.makeDataFrame}
%
\begin{Description}\relax
Create a biomod2-compatible dataframe. The function also enables to remove duplicate presences within a pixel and to set a minimum distance between presence points to avoid autocorrelation. Data from GBIF can be added.
\end{Description}
%
\begin{Usage}
\begin{verbatim}
ecospat.makeDataFrame (spec.list, expl.var, use.gbif=FALSE, precision=NULL,
year=NULL, remdups=TRUE, mindist=NULL, n=1000, type='random', PApoint=NULL,
ext=expl.var, tryf=5)
\end{verbatim}
\end{Usage}
%
\begin{Arguments}
\begin{ldescription}
\item[\code{spec.list}] Data.frame or Character. The species occurrence information must be a data.frame in the form:
\bsl{}'x-coordinates\bsl{}' , \bsl{}'y-coordinates\bsl{}' and \bsl{}'species name\bsl{}' (in the same projection/coordinate system as expl.var!).
\item[\code{expl.var}] a RasterStack object of the environmental layers.
\item[\code{use.gbif}] Logical. If TRUE presence data from GBIF will be added. It is also possible to use GBIF data only.
Default: FALSE. See ?gbif dismo for more information. Settings: geo=TRUE, removeZeros=TRUE,
all sub-taxa will be used.
\bsl{}'species name\bsl{}' in spec.list must be in the form: \bsl{}'genus species\bsl{}',
\bsl{}'genus\_species\bsl{}' or \bsl{}'genus.species\bsl{}'. If there is no species information available on GBIF an error is returned.
Try to change species name (maybe there is a synonym) or switch use.gbif off.
\item[\code{precision}] Numeric. Use precision if use.gbif = TRUE to set a minimum precision of
the presences which should be added. For precision = 1000
e.g. only presences with precision of at least 1000 meter will be added from GBIF.
When precision = NULL all presences from GBIF will be used, also presences
where precision information is NA.
\item[\code{year}] Numeric. Latest year of the collected gbif occurrences. If year=1960 only occurrences which were collected since 1960 are used.
\item[\code{remdups}] Logical. If TRUE (Default) duplicated presences within a raster pixel
will be removed. You will get only one presence per pixel.
\item[\code{mindist}] Numeric. You can set a minimum distance between presence points to avoid
autocorrelation. nndist spatstat is used to calculate the nearest neighbour (nn)
for each point. From the pair of the minimum nn, the point is removed of which the second
neighbour is closer. Unit is the same as expl.var.
\item[\code{n}] number of Pseudo-Absences. Default 1000.
\item[\code{type}] sampling dessign for selecting Pseudo-Absences. If \bsl{}'random\bsl{}' (default) background points are selected with the 
function randomPoints dismo. When selecting another sampling type (\bsl{}'regular\bsl{}', \bsl{}'stratified\bsl{}', \bsl{}'nonaligned\bsl{}',
\bsl{}'hexagonal\bsl{}', \bsl{}'clustered\bsl{}' or \bsl{}'Fibonacci\bsl{}') spsample sp will be used. This can immensely increase computation time and RAM usage if ext is a raster,
especially for big raster layers because it must be converted into a  \bsl{}'SpatialPolygonsDataFrame\bsl{}' first.
\item[\code{PApoint}] data.frame or SpatialPoints. You can use your own set of Pseudo-Absences
instead of generating new PAs. Two columns with \bsl{}'x\bsl{}' and \bsl{}'y\bsl{}' in the same
projection/coordinate system as expl.var!
\item[\code{ext}] a Spatial Object or Raster object. Extent from which PAs should be selected from (Default is expl.var).
\item[\code{tryf}] numeric > 1. Number of trials to create the requested Pseudo-Absences after removing NA points (if type='random'). See ?randomPoints dismo
\end{ldescription}
\end{Arguments}
%
\begin{Details}\relax
If you use a raster stack as explanatory variable and you want to model many species in a loop with Biomod, formating data will last very long as presences and PA's have to be extracted over and over again from the raster stack. To save computation time, it is better to convert the presences and PAs to a data.frame first.
\end{Details}
%
\begin{Value}
A data.frame object which can be used for modeling with the Biomod package.
\end{Value}
%
\begin{Author}\relax
Frank Breiner \email{frank.breiner@unil.ch}
\end{Author}
%
\begin{Examples}
\begin{ExampleCode}
## Not run: 
files <- list.files(path=paste(system.file(package="dismo"),
                               '/ex', sep=''), pattern='grd', full.names=TRUE )
predictors <- raster::stack(files[c(9,1:8)])   #file 9 has more NA values than
# the other files, this is why we choose it as the first layer (see ?randomPoints)

file <- paste(system.file(package="dismo"), "/ex/bradypus.csv", sep="")
bradypus <- read.table(file, header=TRUE, sep=',')[,c(2,3,1)]
head(bradypus)

random.spec <- cbind(as.data.frame(randomPoints(predictors,50,extf=1)),species="randomSpec")
colnames(random.spec)[1:2] <- c("lon","lat")

spec.list <- rbind(bradypus, random.spec)

df <- ecospat.makeDataFrame(spec.list, expl.var=predictors, n=5000)
head(df)

plot(predictors[[1]])
points(df[df$Bradypus.variegatus==1, c('x','y')])
points(df[df$randomSpec==1, c('x','y')], col="red")


## End(Not run)
\end{ExampleCode}
\end{Examples}
\inputencoding{utf8}
\HeaderA{ecospat.mantel.correlogram}{Mantel Correlogram}{ecospat.mantel.correlogram}
%
\begin{Description}\relax
Investigate spatial autocorrelation of environmental covariables within a set of occurrences as a function of distance.
\end{Description}
%
\begin{Usage}
\begin{verbatim}
ecospat.mantel.correlogram (dfvar, colxy, n, colvar, max, nclass, nperm)
\end{verbatim}
\end{Usage}
%
\begin{Arguments}
\begin{ldescription}
\item[\code{dfvar}] A dataframe object with the environmental variables.
\item[\code{colxy}] The range of columns for x and y in df.
\item[\code{n}] The number of random occurrences used for the test.
\item[\code{colvar}] The range of columns for variables in df.
\item[\code{max}] The maximum distance to be computed in the correlogram.
\item[\code{nclass}] The number of classes of distances to be computed in the correlogram.
\item[\code{nperm}] The number of permutations in the randomization process.
\end{ldescription}
\end{Arguments}
%
\begin{Details}\relax
Requires ecodist library. Note that computation time increase tremendously when using more than 500 occurrences (n>500)
\end{Details}
%
\begin{Value}
Draws a plot with distance vs. the mantel r  value. Black circles indicate that the values are significative different from zero. White circles indicate non significant autocorrelation. The selected distance is at the first white circle where values are non significative different from cero. 
\end{Value}
%
\begin{Author}\relax
Olivier Broennimann \email{olivier.broennimann@unil.ch}
\end{Author}
%
\begin{References}\relax
Legendre, P. and M.J. Fortin. 1989. Spatial pattern and ecological analysis. \emph{Vegetatio}, \bold{80}, 107-138.
\end{References}
%
\begin{SeeAlso}\relax
\code{\LinkA{mgram}{mgram}}
\end{SeeAlso}
%
\begin{Examples}
\begin{ExampleCode}
ecospat.mantel.correlogram(dfvar=ecospat.testData[c(2:16)],colxy=1:2, n=100, colvar=3:7, 
max=1000, nclass=10, nperm=100)
\end{ExampleCode}
\end{Examples}
\inputencoding{utf8}
\HeaderA{ecospat.max.kappa}{Maximum Kappa}{ecospat.max.kappa}
%
\begin{Description}\relax
Calculates values for Cohen's Kappa along different thresholds, considering this time 0.01 increments (i.e. 99 thresholds).
\end{Description}
%
\begin{Usage}
\begin{verbatim}
    ecospat.max.kappa(Pred, Sp.occ)
\end{verbatim}
\end{Usage}
%
\begin{Arguments}
\begin{ldescription}
\item[\code{Pred}] 
A vector of predicted probabilities

\item[\code{Sp.occ}] 
A vector of binary observations of the species occurrence

\end{ldescription}
\end{Arguments}
%
\begin{Value}
Return values for Cohen's Kappa for 99 thresholds at 0.01 increments. 
\end{Value}
%
\begin{Author}\relax
Antoine Guisan \email{antoine.guisan@unil.ch} with contributions of Luigi Maiorano \email{luigi.maiorano@gmail.com} and Valeria Di Cola \email{valeria.dicola@unil.ch}.
\end{Author}
%
\begin{References}\relax
Liu, C., P.M. Berry, T.P. Dawson, and R.G. Pearson. 2005. Selecting thresholds of occurrence in the prediction of species distributions. \emph{Ecography}, \bold{28}, 385-393.

\end{References}
%
\begin{SeeAlso}\relax
\code{\LinkA{ecospat.meva.table}{ecospat.meva.table}}, \code{\LinkA{ecospat.max.tss}{ecospat.max.tss}}, \code{\LinkA{ecospat.plot.tss}{ecospat.plot.tss}}, \code{\LinkA{ecospat.cohen.kappa}{ecospat.cohen.kappa}}, \code{\LinkA{ecospat.plot.kappa}{ecospat.plot.kappa}}
\end{SeeAlso}
%
\begin{Examples}
\begin{ExampleCode}

   ## Not run: 
Pred <- ecospat.testData$glm_Agrostis_capillaris
Sp.occ <- ecospat.testData$Agrostis_capillaris
kappa100 <- ecospat.max.kappa(Pred, Sp.occ)
   
## End(Not run)

\end{ExampleCode}
\end{Examples}
\inputencoding{utf8}
\HeaderA{ecospat.max.tss}{Maximum TSS}{ecospat.max.tss}
\keyword{file}{ecospat.max.tss}
%
\begin{Description}\relax
Calculates values for True skill statistic (TSS) along different thresholds, considering this time 0.01 increments (i.e. 99 thresholds).
\end{Description}
%
\begin{Usage}
\begin{verbatim}
    ecospat.max.tss(Pred, Sp.occ)
\end{verbatim}
\end{Usage}
%
\begin{Arguments}
\begin{ldescription}
\item[\code{Pred}] 
A vector of predicted probabilities

\item[\code{Sp.occ}] 
A vector of binary observations of the species occurrence

\end{ldescription}
\end{Arguments}
%
\begin{Value}
Return values for TSS for 99 thresholds at 0.01 increments. 
\end{Value}
%
\begin{Author}\relax
Luigi Maiorano \email{luigi.maiorano@gmail.com} with contributions of Antoine Guisan \email{antoine.guisan@unil.ch}
\end{Author}
%
\begin{References}\relax
Liu, C., P.M. Berry, T.P. Dawson, and R.G. Pearson. 2005. Selecting thresholds of occurrence in the prediction of species distributions. \emph{Ecography}, \bold{28}, 385-393.
\end{References}
%
\begin{SeeAlso}\relax
\code{\LinkA{ecospat.meva.table}{ecospat.meva.table}}, \code{\LinkA{ecospat.max.kappa}{ecospat.max.kappa}}, \code{\LinkA{ecospat.plot.tss}{ecospat.plot.tss}}, \code{\LinkA{ecospat.cohen.kappa}{ecospat.cohen.kappa}}, \code{\LinkA{ecospat.plot.kappa}{ecospat.plot.kappa}}
\end{SeeAlso}
%
\begin{Examples}
\begin{ExampleCode}

Pred <- ecospat.testData$glm_Agrostis_capillaris
Sp.occ <- ecospat.testData$Agrostis_capillaris
TSS100 <- ecospat.max.tss(Pred, Sp.occ)
\end{ExampleCode}
\end{Examples}
\inputencoding{utf8}
\HeaderA{ecospat.maxentvarimport}{Maxent Variable Importance}{ecospat.maxentvarimport}
%
\begin{Description}\relax
Calculate the importance of variables for Maxent in the same way Biomod does, by randomly permuting each predictor variable independently, and computing the associated reduction in predictive performance.
\end{Description}
%
\begin{Usage}
\begin{verbatim}
ecospat.maxentvarimport (model, dfvar, nperm)
\end{verbatim}
\end{Usage}
%
\begin{Arguments}
\begin{ldescription}
\item[\code{model}] The name of the maxent model.
\item[\code{dfvar}] A dataframe object with the environmental variables.
\item[\code{nperm}] The number of permutations in the randomization process. The default is 5.
\end{ldescription}
\end{Arguments}
%
\begin{Details}\relax
The calculation is made as biomod2 "variables\_importance" function. It's more or less base on the same principle than randomForest variables importance algorithm. The principle is to shuffle a single variable of the given data. Make model prediction with this 'shuffled' data.set. Then we compute a simple correlation (Pearson's by default) between references predictions and the 'shuffled' one. The return score is 1-cor(pred\_ref,pred\_shuffled). The highest the value, the more influence the variable has on the model. A value of this 0 assumes no influence of that variable on the model. Note that this technique does not account for interactions between the variables.
\end{Details}
%
\begin{Value}
a \code{list} which contains a \code{data.frame} containing variables importance scores for each permutation run. 
\end{Value}
%
\begin{Author}\relax
Blaise Petitpierre \email{bpetitpierre@gmail.com}
\end{Author}
%
\begin{Examples}
\begin{ExampleCode}
## Not run: 
model <- get ("me.Achillea_millefolium", envir=ecospat.env)
dfvar <- ecospat.testData[4:8]
nperm <- 5
ecospat.maxentvarimport (model, cal, nperm)
## End(Not run)
\end{ExampleCode}
\end{Examples}
\inputencoding{utf8}
\HeaderA{ecospat.mdr}{Minimum Dispersal Routes)}{ecospat.mdr}
%
\begin{Description}\relax
\code{ecospat.mdr} is a function that implement a minimum cost arborescence approach to analyse the invasion routes of a species from dates occurrence data.
\end{Description}
%
\begin{Usage}
\begin{verbatim}
ecospat.mdr (data, xcol, ycol, datecol, mode, rep, mean.date.error, fixed.sources.rows)
\end{verbatim}
\end{Usage}
%
\begin{Arguments}
\begin{ldescription}
\item[\code{data}] dataframe with occurence data. Each row correspond to an occurrence.
\item[\code{xcol}] The column in data containing x coordinates.
\item[\code{ycol}] The column in data containing y coordinates.
\item[\code{datecol}] The column in data containing dates.
\item[\code{mode}] "observed", "min" or "random". "observed" calculate routes using real dates. "min" reorder dates so the the total length of the routes are minimal. "random" reatribute dates randomly.
\item[\code{rep}] number of iteration of the analyse. if > 1, boostrap support for each route is provided.
\item[\code{mean.date.error}] mean number of years to substract to observed dates. It is the mean of the truncated negative exponential distribution from which the time to be substracted is randomly sampled.
\item[\code{fixed.sources.rows}] the rows in data (as a vector) corresponding to source occurrence(s) that initiated the invasion(s). No incoming routes are not calculated for sources.
\end{ldescription}
\end{Arguments}
%
\begin{Details}\relax
The function draws an incoming route to each occurence from the closest occurrence already occupied (with a previous date) and allows to substract a random number of time (years) to the observed dates from a truncated negative exponential distribution. It is possible to run several iterations and to get boostrap support for each route.
\code{itexp} and \code{rtexp} functions are small internal functions to set a truncated negative exponential distribution.
\end{Details}
%
\begin{Value}
A list is returned by the function with in positon [[1]], a datafame with each row corresponding to a route (with new/old coordinates, new/old dates, length of the route, timespan, dispersal rate), in position [[2]] the total route length, in position [[3]] the median dispersal rate and in position [[4]] the number of outgoing nodes (index of clustering of the network)
\end{Value}
%
\begin{Author}\relax
Olivier Broennimann \email{olivier.broennimann@unil.ch}
\end{Author}
%
\begin{References}\relax
Hordijk, W. and O. Broennimann. 2012. Dispersal routes reconstruction and the minimum cost arborescence problem. \emph{Journal of theoretical biology}, \bold{308}, 115-122.

Broennimann, O., P. Mraz, B. Petitpierre, A. Guisan, and H. Muller-Scharer. 2014. Contrasting spatio-temporal climatic niche dynamics during the eastern and western invasions of spotted knapweed in North America.\emph{Journal of biogeography}, \bold{41}, 1126-1136.
\end{References}
%
\begin{Examples}
\begin{ExampleCode}
   ## Not run: 
library(maps)
data<- ecospat.testMdr

fixed.sources.rows<-order(data$date)[1:2] #first introductions

#plot observed situation
plot(data[,2:1],pch=15,cex=0.5)
points(data[fixed.sources.rows,2:1],pch=19,col="red")
text(data[,2]+0.5,data[,1]+0.5,data[,3],cex=0.5)
map(add=T)

# mca 
obs<-ecospat.mdr(data=data,
xcol=2,
ycol=1,
datecol=3,
mode="min",
rep=100,
mean.date.error=1,
fixed.sources.rows)

#plot results
lwd<-(obs[[1]]$bootstrap.value)
x11();plot(obs[[1]][,3:4],type="n",xlab="longitude",ylab="latitude")
arrows(obs[[1]][,1],obs[[1]][,2],obs[[1]][,3],obs[[1]][,4],length = 0.05,lwd=lwd*2)
map(add=T)
points(data[fixed.sources.rows,2:1],pch=19,col="red")
text(data[fixed.sources.rows,2]+0.5,data[fixed.sources.rows,1]+0.5,data[fixed.sources.rows,3],
cex=1,col="red")
title(paste("total routes length : ",
round(obs[[2]],2)," Deg","\n","median dispersal rate : ",
round(obs[[3]],2)," Deg/year","\n","number of outcoming nodes : ",
obs[[4]]))

## End(Not run)
\end{ExampleCode}
\end{Examples}
\inputencoding{utf8}
\HeaderA{ecospat.mess}{MESS}{ecospat.mess}
%
\begin{Description}\relax
Calculate the MESS (i.e. extrapolation) as in Maxent.
\end{Description}
%
\begin{Usage}
\begin{verbatim}
ecospat.mess (proj, cal, w="default")
\end{verbatim}
\end{Usage}
%
\begin{Arguments}
\begin{ldescription}
\item[\code{proj}] A dataframe object with x, y and environmental variables, used as projection dataset.
\item[\code{cal}] A dataframe object with x, y and environmental variables, used as calibration dataset.
\item[\code{w}] The weight for each predictor (e.g. variables importance in SDM).
\end{ldescription}
\end{Arguments}
%
\begin{Details}\relax
Shows the variable that drives the multivariate environmental similarity surface (MESS) value in each grid cell.
\end{Details}
%
\begin{Value}
\begin{ldescription}
\item[\code{MESS}] The mess as calculated in Maxent, i.e. the minimal extrapolation values.
\item[\code{MESSw}] The sum of negative MESS values corrected by the total number of predictors. If there are no negative values, MESSw is the mean MESS.
\item[\code{MESSneg}] The number of predictors on which there is extrapolation.
\end{ldescription}
\end{Value}
%
\begin{Author}\relax
Blaise Petitpierre \email{bpetitpierre@gmail.com}. Modified by Daniel Scherrer \email{daniel.j.a.scherrer@gmail.com}
\end{Author}
%
\begin{References}\relax
Elith, J., M. Kearney and S. Phillips. 2010. The art of modelling range-shifting species. \emph{Methods in ecology and evolution}, \bold{1}, 330-342.

\end{References}
%
\begin{SeeAlso}\relax
\code{\LinkA{ecospat.plot.mess}{ecospat.plot.mess}}
\end{SeeAlso}
%
\begin{Examples}
\begin{ExampleCode}
x <- ecospat.testData[c(2,3,4:8)]
proj <- x[1:90,] #A projection dataset.
cal <- x[91:300,] #A calibration dataset

#Create a MESS object 
mess.object <- ecospat.mess (proj, cal, w="default")

#Plot MESS 
ecospat.plot.mess (mess.object, cex=1, pch=15)
\end{ExampleCode}
\end{Examples}
\inputencoding{utf8}
\HeaderA{ecospat.meva.table}{Model Evaluation For A Given Threshold Value}{ecospat.meva.table}
\keyword{file}{ecospat.meva.table}
%
\begin{Description}\relax
Calculates values of a series of different evaluations metrics for a model and for a given threshold value

\end{Description}
%
\begin{Usage}
\begin{verbatim}
    ecospat.meva.table(Pred, Sp.occ, th)
\end{verbatim}
\end{Usage}
%
\begin{Arguments}
\begin{ldescription}
\item[\code{Pred}] 
A vector of predicted probabilities

\item[\code{Sp.occ}] 
A vector of binary observations of the species occurrence

\item[\code{th}] 
Threshold used to cut the probability to binary values

\end{ldescription}
\end{Arguments}
%
\begin{Value}
A contingency table of observations and predicted probabilities of presence values, and a list of evaluation metrics for the selected threshold.
\end{Value}
%
\begin{Author}\relax
Antoine Guisan \email{antoine.guisan@unil.ch} with contributions of Luigi Maiorano \email{luigi.maiorano@gmail.com}
\end{Author}
%
\begin{References}\relax
Pearce, J. and S. Ferrier. 2000. Evaluating the predictive performance of habitat models developed using logistic regression. \emph{Ecol. Model.}, \bold{133}, 225-245.
\end{References}
%
\begin{SeeAlso}\relax
\code{\LinkA{ecospat.max.kappa}{ecospat.max.kappa}}, \code{\LinkA{ecospat.max.tss}{ecospat.max.tss}}, \code{\LinkA{ecospat.plot.tss}{ecospat.plot.tss}}, \code{\LinkA{ecospat.cohen.kappa}{ecospat.cohen.kappa}}, \code{\LinkA{ecospat.plot.kappa}{ecospat.plot.kappa}}

\end{SeeAlso}
%
\begin{Examples}
\begin{ExampleCode}

Pred <- ecospat.testData$glm_Agrostis_capillaris
Sp.occ <- ecospat.testData$Agrostis_capillaris

meva <- ecospat.meva.table (Pred, Sp.occ, 0.39)
\end{ExampleCode}
\end{Examples}
\inputencoding{utf8}
\HeaderA{ecospat.migclim}{Implementing Dispersal Into Species Distribution Models}{ecospat.migclim}
%
\begin{Description}\relax
Enables the implementation of species-specific dispersal constraints into projections of species distribution models under environmental change and/or landscape fragmentation scenarios.
\end{Description}
%
\begin{Usage}
\begin{verbatim}
ecospat.migclim ()
\end{verbatim}
\end{Usage}
%
\begin{Details}\relax
The MigClim model is a cellular automaton originally designed to implement dispersal constraints into projections of species distributions under environmental change and landscape fragmentation scenarios.
\end{Details}
%
\begin{Author}\relax
Robin Engler \email{robin.engler@gmail.com}, Wim Hordijk \email{wim@WorldWideWanderings.net} and Loic Pellissier \email{loic.pellissier@unifr.ch}
\end{Author}
%
\begin{References}\relax
Engler, R., W. Hordijk and A. Guisan. 2012. The MIGCLIM R package -- seamless integration of dispersal constraints into projections of species distribution models. \emph{Ecography}, \bold{35}, 872-878.

Engler, R. and A. Guisan. 2009. MIGCLIM: predicting plant distribution and dispersal in a changing climate. \emph{Diversity and Distributions}, \bold{15}, 590-601.

Engler, R., C.F. Randin, P. Vittoz, T. Czaka, M. Beniston, N.E. Zimmermann and A. Guisan. 2009. Predicting future distributions of mountain plants under climate change: does dispersal capacity matter? \emph{Ecography}, \bold{32}, 34-45.
\end{References}
%
\begin{Examples}
\begin{ExampleCode}
## Not run: 
ecospat.migclim()
### Some example data files can be downloaded from the following web page:
### http://www.unil.ch/ecospat/page89413.html
###
### Run the example as follows (set the current working directory to the
### folder where the example data files are located):
###
data(MigClim.testData)
### Run MigClim with a data frame type input.
n <- MigClim.migrate (iniDist=MigClim.testData[,1:3],
hsMap=MigClim.testData[,4:8], rcThreshold=500,
envChgSteps=5, dispSteps=5, dispKernel=c(1.0,0.4,0.16,0.06,0.03),
barrier=MigClim.testData[,9], barrierType="strong",
iniMatAge=1, propaguleProd=c(0.01,0.08,0.5,0.92),
lddFreq=0.1, lddMinDist=6, lddMaxDist=15,
simulName="MigClimTest", replicateNb=1, overWrite=TRUE,
testMode=FALSE, fullOutput=FALSE, keepTempFiles=FALSE)
## End(Not run)

\end{ExampleCode}
\end{Examples}
\inputencoding{utf8}
\HeaderA{ecospat.mpa}{Minimal Predicted Area}{ecospat.mpa}
%
\begin{Description}\relax
Calculate the minimal predicted area.
\end{Description}
%
\begin{Usage}
\begin{verbatim}
ecospat.mpa (Pred, Sp.occ.xy, perc)
\end{verbatim}
\end{Usage}
%
\begin{Arguments}
\begin{ldescription}
\item[\code{Pred}] Numeric or RasterLayer predicted suitabilities from a SDM prediction.
\item[\code{Sp.occ.xy}] xy-coordinates of the species (if Pred is a RasterLayer).
\item[\code{perc}] Percentage of Sp.occ.xy that should be encompassed by the binary map.

\end{ldescription}
\end{Arguments}
%
\begin{Details}\relax
The minimal predicted area (MPA) is the minimal surface obtained by considering all pixels with predictions above a defined probability threshold (e.g. 0.7) that still encompasses 90 percent of the species` occurrences (Engler et al. 2004).
\end{Details}
%
\begin{Value}
Returns the minimal predicted area.
\end{Value}
%
\begin{Author}\relax
Frank Breiner \email{frank.breiner@wsl.ch}
\end{Author}
%
\begin{References}\relax
Engler, R., A. Guisan and L. Rechsteiner. 2004. An improved approach for predicting the distribution of rare and endangered species from occurrence and pseudo-absence data. \emph{Journal of Applied Ecology}, \bold{41}, 263-274.
\end{References}
%
\begin{Examples}
\begin{ExampleCode}
obs <- (ecospat.testData$glm_Saxifraga_oppositifolia
[which(ecospat.testData$Saxifraga_oppositifolia==1)])

ecospat.mpa(obs)
ecospat.mpa(obs,perc=1) ## 100 percent of the presences encompassed
\end{ExampleCode}
\end{Examples}
\inputencoding{utf8}
\HeaderA{ecospat.niche.dyn.index}{Niche Expansion, Stability, and Unfilling}{ecospat.niche.dyn.index}
%
\begin{Description}\relax
Calculate niche expansion, stability and unfilling.
\end{Description}
%
\begin{Usage}
\begin{verbatim}
ecospat.niche.dyn.index (z1, z2, intersection=NA)
\end{verbatim}
\end{Usage}
%
\begin{Arguments}
\begin{ldescription}
\item[\code{z1}] A gridclim object for the native distribution.
\item[\code{z2}] A gridclim object for the invaded range.
\item[\code{intersection}] The quantile of the environmental density used to remove marginal climates. If \code{intersection=NA}, the analysis is performed on the whole environmental extent (native and invaded). If \code{intersection=0}, the analysis is performed at the intersection between native and invaded range. If \code{intersection=0.05}, the analysis is performed at the intersection of the 5th quantile of both native and invaded environmental densities.
\end{ldescription}
\end{Arguments}
%
\begin{Details}\relax
If you encounter a problem during your analyses, please first read the FAQ section of "Niche overlap" in http://www.unil.ch/ecospat/home/menuguid/ecospat-resources/tools.html
\end{Details}
%
\begin{Value}
A list of dynamic indices: dynamic.index.w [expansion.index.w, stability.index.w, restriction.index.w]
\end{Value}
%
\begin{Author}\relax
Blaise Petitpierre \email{bpetitpierre@gmail.com}
\end{Author}
%
\begin{SeeAlso}\relax
\code{\LinkA{ecospat.grid.clim.dyn}{ecospat.grid.clim.dyn}}
\end{SeeAlso}
\inputencoding{utf8}
\HeaderA{ecospat.niche.dynIndexProjGeo}{Projection of niche dynamic indices to the Geography}{ecospat.niche.dynIndexProjGeo}
%
\begin{Description}\relax
Creates a raster in geography with each pixel containing a niche dynamic index (stability, expansion, or unfilling) extracted from 2 niches generated with \code{ecospat.grid.clim.dyn}.
\end{Description}
%
\begin{Usage}
\begin{verbatim}
ecospat.niche.dynIndexProjGeo(z1,z2,env,index)
\end{verbatim}
\end{Usage}
%
\begin{Arguments}
\begin{ldescription}
\item[\code{z1}] Species 1 occurrence density grid created by \code{ecospat.grid.clim.dyn}.
\item[\code{z2}] Species 2 occurrence density grid created by \code{ecospat.grid.clim.dyn}.
\item[\code{env}] A RasterStack or RasterBrick of environmental variables corresponding to the background (\code{glob} in \code{ecospat.grid.clim.dyn}).
\item[\code{index}] "stability", "unfilling" or "expansion"
\end{ldescription}
\end{Arguments}
%
\begin{Details}\relax
extracts the niche dynamic index of objects created by \code{ecospat.niche.dyn.index} for each point of the background (\code{glob}) using \code{extract} (package raster). The values are binded to the geographic coordinates of \code{env} and a raster is then recreated using \code{rasterFromXYZ}
\end{Details}
%
\begin{Value}
raster of class RasterLayer
\end{Value}
%
\begin{Author}\relax
Olivier Broennimann \email{olivier.broennimann@unil.ch}
\end{Author}
%
\begin{References}\relax
Broennimann, O., M.C. Fitzpatrick, P.B. Pearman, B. Petitpierre, L. Pellissier, N.G. Yoccoz, W. Thuiller, M.J. Fortin, C. Randin, N.E. Zimmermann, C.H. Graham and A. Guisan. 2012. Measuring ecological niche overlap from occurrence and spatial environmental data. \emph{Global Ecology and Biogeography}, \bold{21}:481-497.


Petitpierre, B., C. Kueffer, O. Broennimann, C. Randin, C. Daehler and A. Guisan. 2012. Climatic niche shifts are rare among terrestrial plant invaders. \emph{Science}, \bold{335}:1344-1348.
\end{References}
%
\begin{SeeAlso}\relax
\code{\LinkA{ecospat.plot.niche.dyn}{ecospat.plot.niche.dyn}},\code{\LinkA{ecospat.niche.dyn.index}{ecospat.niche.dyn.index}}, \code{\LinkA{ecospat.niche.zProjGeo}{ecospat.niche.zProjGeo}}
\end{SeeAlso}
%
\begin{Examples}
\begin{ExampleCode}
## Not run: 

library(raster)

spp <- ecospat.testNiche
xy.sp1<-subset(spp,species=="sp1")[2:3] #Bromus_erectus
xy.sp2<-subset(spp,species=="sp3")[2:3] #Daucus_carota

?ecospat.testEnvRaster
load(system.file("extdata", "ecospat.testEnvRaster.Rdata", package="ecospat"))

env.sp1<-extract(env,xy.sp1)
env.sp2<-extract(env,xy.sp2)
env.bkg<-na.exclude(values(env))

#################################### PCA-ENVIRONMENT ##################################

pca.cal <- dudi.pca(env.bkg, center = TRUE, scale = TRUE, scannf = FALSE, nf = 2)

# predict the scores on the axes
scores.bkg <- pca.cal$li  #scores for background climate
scores.sp1 <- suprow(pca.cal,env.sp1)$lisup				#scores for sp1
scores.sp2 <- suprow(pca.cal,env.sp2)$lisup				#scores for sp2

# calculation of occurence density (niche z)
  
z1 <- ecospat.grid.clim.dyn(scores.bkg, scores.bkg, scores.sp1,R=100)
z2 <- ecospat.grid.clim.dyn(scores.bkg, scores.bkg, scores.sp2,R=100)

plot(z1$z.uncor)
points(scores.sp1)

plot(z2$z.uncor)
points(scores.sp2)

ecospat.niche.overlap(z1,z2 ,cor=T)

#################################### stability S in space ##################################

geozS<-ecospat.niche.dynIndexProjGeo(z1,z2,env,index="stability")

plot(geozS,main="Stability")
points(xy.sp1,col="red")
points(xy.sp2,col="blue")

## End(Not run)
\end{ExampleCode}
\end{Examples}
\inputencoding{utf8}
\HeaderA{ecospat.niche.equivalency.test}{Niche Equivalency Test}{ecospat.niche.equivalency.test}
%
\begin{Description}\relax
Run a niche equivalency test (see Warren et al 2008) based on two species occurrence density grids.
\end{Description}
%
\begin{Usage}
\begin{verbatim}
ecospat.niche.equivalency.test (z1, z2, rep, alternative, ncores = 1)
\end{verbatim}
\end{Usage}
%
\begin{Arguments}
\begin{ldescription}
\item[\code{z1}] Species 1 occurrence density grid created by \code{ecospat.grid.clim}.
\item[\code{z2}] Species 2 occurrence density grid created by \code{ecospat.grid.clim}.
\item[\code{rep}] The number of replications to perform.
\item[\code{alternative}] To indicate the type of test to be performed. It could be greater or lower.
\item[\code{ncores}] The number of cores used for parallelisation.

\end{ldescription}
\end{Arguments}
%
\begin{Details}\relax
Compares the observed niche overlap between z1 and z2 to overlaps between random niches z1.sim and z2.sim, which are built from random reallocations of occurences of z1 and z2.

\code{alternative} specifies if you want to test for niche conservatism (alternative = "greater", i.e. the niche overlap is more equivalent/similar than random) or for niche divergence (alternative = "lower", i.e. the niche overlap is less equivalent/similar than random). 

If you encounter a problem during your analyses, please first read the FAQ section of "Niche overlap" in http://www.unil.ch/ecospat/home/menuguid/ecospat-resources/tools.html

The arguments \code{ncores} allows choosing the number of cores used to parallelize the computation.  The default value is 1. On multicore computers, the optimal would be \code{ncores = detectCores() - 1}.
\end{Details}
%
\begin{Value}
a list with \$obs = observed overlaps, \$sim = simulated overlaps, \$p.D = p-value of the test on D, \$p.I = p-value of the test on I.
\end{Value}
%
\begin{Author}\relax
Olivier Broennimann \email{olivier.broennimann@unil.ch} with contributions of Blaise Petitpierre \email{bpetitpierre@gmail.com}
\end{Author}
%
\begin{References}\relax
Broennimann, O., M.C. Fitzpatrick, P.B. Pearman,B.  Petitpierre, L. Pellissier, N.G. Yoccoz, W. Thuiller, M.J. Fortin, C. Randin, N.E. Zimmermann, C.H. Graham and A. Guisan. 2012. Measuring ecological niche overlap from occurrence and spatial environmental data. \emph{Global Ecology and Biogeography}, \bold{21}, 481-497.

Warren, D.L., R.E. Glor and M. Turelli. 2008. Environmental niche equivalency versus conservatism: quantitative approaches to niche evolution. \emph{Evolution}, \bold{62}, 2868-2883.
\end{References}
%
\begin{SeeAlso}\relax
\code{\LinkA{ecospat.grid.clim.dyn}{ecospat.grid.clim.dyn}}, \code{\LinkA{ecospat.niche.similarity.test}{ecospat.niche.similarity.test}}
\end{SeeAlso}
\inputencoding{utf8}
\HeaderA{ecospat.niche.overlap}{Calculate Niche Overlap}{ecospat.niche.overlap}
%
\begin{Description}\relax
Calculate the overlap metrics D and I based on two species occurrence density grids z1 and z2 created by \code{ecospat.grid.clim}.
\end{Description}
%
\begin{Usage}
\begin{verbatim}
ecospat.niche.overlap (z1, z2, cor)
\end{verbatim}
\end{Usage}
%
\begin{Arguments}
\begin{ldescription}
\item[\code{z1}] Species 1 occurrence density grid created by \code{ecospat.grid.clim}.
\item[\code{z2}] Species 2 occurrence density grid created by \code{ecospat.grid.clim}.
\item[\code{cor}] Correct the occurrence densities of each species by the prevalence of the environments in their range (TRUE = yes, FALSE = no).
\end{ldescription}
\end{Arguments}
%
\begin{Details}\relax
if cor=FALSE, the z\$uncor objects created by \code{ecospat.grid.clim} are used to calculate the overlap. if cor=TRUE, the z\$cor objects are used.

If you encounter a problem during your analyses, please first read the FAQ section of "Niche overlap" in http://www.unil.ch/ecospat/home/menuguid/ecospat-resources/tools.html
\end{Details}
%
\begin{Value}
Overlap values D and I. D is Schoener's overlap metric (Schoener 1970). I is a modified Hellinger metric(Warren et al. 2008)
\end{Value}
%
\begin{Author}\relax
Olivier Broennimann \email{olivier.broennimann@unil.ch}
\end{Author}
%
\begin{References}\relax
Broennimann, O., M.C. Fitzpatrick, P.B. Pearman, B. Petitpierre, L. Pellissier, N.G. Yoccoz, W. Thuiller, M.J. Fortin, C. Randin, N.E. Zimmermann, C.H. Graham and A. Guisan. 2012. Measuring ecological niche overlap from occurrence and spatial environmental data. \emph{Global Ecology and Biogeography}, \bold{21}, 481-497.

Schoener, T.W. 1968. Anolis lizards of Bimini: resource partitioning in a complex fauna. \emph{Ecology}, \bold{49}, 704-726.

Warren, D.L., R.E. Glor and M. Turelli. 2008. Environmental niche equivalency versus conservatism: quantitative approaches to niche evolution. \emph{Evolution}, \bold{62}, 2868-2883.
\end{References}
%
\begin{SeeAlso}\relax
\code{\LinkA{ecospat.grid.clim.dyn}{ecospat.grid.clim.dyn}}
\end{SeeAlso}
\inputencoding{utf8}
\HeaderA{ecospat.niche.similarity.test}{Niche Similarity Test}{ecospat.niche.similarity.test}
%
\begin{Description}\relax
Run a niche similarity test (see Warren et al 2008) based on two species occurrence density grids.
\end{Description}
%
\begin{Usage}
\begin{verbatim}
ecospat.niche.similarity.test (z1, z2, rep, alternative = "greater", 
rand.type = 1, ncores= 1)
\end{verbatim}
\end{Usage}
%
\begin{Arguments}
\begin{ldescription}
\item[\code{z1}] Species 1 occurrence density grid created by \code{ecospat.grid.clim}.
\item[\code{z2}] Species 2 occurrence density grid created by \code{ecospat.grid.clim}.
\item[\code{rep}] The number of replications to perform.
\item[\code{alternative}] To indicate the type of test to be performed. It could be greater or lower.
\item[\code{rand.type}] Type of randomization on the density grids (1 or 2).
\item[\code{ncores}] The number of cores used for parallelisation.

\end{ldescription}
\end{Arguments}
%
\begin{Details}\relax
Compares the observed niche overlap between z1 and z2 to overlaps between z1 and random niches (z2.sim) as available in the range of z2 (z2\$Z). z2.sim has the same pattern as z2 but the center is randomly translatated in the availabe z2\$Z space and weighted by z2\$Z densities.
If rand.type = 1, both z1 and z2 are randomly shifted, if rand.type =2, only z2 is randomly shifted.

\code{alternative} specifies if you want to test for niche conservatism (alternative = "greater", i.e. the niche overlap is more equivalent/similar than random) or for niche divergence (alternative = "lower", i.e. the niche overlap is less equivalent/similar than random). 

If you encounter a problem during your analyses, please first read the FAQ section of "Niche overlap" in http://www.unil.ch/ecospat/home/menuguid/ecospat-resources/tools.html

The arguments \code{ncores} allows choosing the number of cores used to parallelize the computation. The default value is 1. On multicore computers, the optimal would be \code{ncores = detectCores() - 1}. 

\end{Details}
%
\begin{Value}
a list with \$obs = observed overlaps, \$sim = simulated overlaps, \$p.D = p-value of the test on D, \$p.I = p-value of the test on I.
\end{Value}
%
\begin{Author}\relax
Olivier Broennimann \email{olivier.broennimann@unil.ch} with contributions of Blaise Petitpierre \email{bpetitpierre@gmail.com}
\end{Author}
%
\begin{References}\relax
Broennimann, O., M.C. Fitzpatrick, P.B. Pearman, B. Petitpierre, L. Pellissier, N.G. Yoccoz, W. Thuiller, M.J. Fortin, C. Randin, N.E. Zimmermann, C.H. Graham and A. Guisan. 2012. Measuring ecological niche overlap from occurrence and spatial environmental data. \emph{Global Ecology and Biogeography}, \bold{21}, 481-497.

Warren, D.L., R.E. Glor and M. Turelli. 2008. Environmental niche equivalency versus conservatism: quantitative approaches to niche evolution. \emph{Evolution}, \bold{62}, 2868-2883.
\end{References}
%
\begin{SeeAlso}\relax
\code{\LinkA{ecospat.grid.clim.dyn}{ecospat.grid.clim.dyn}}, \code{\LinkA{ecospat.niche.equivalency.test}{ecospat.niche.equivalency.test}}
\end{SeeAlso}
\inputencoding{utf8}
\HeaderA{ecospat.niche.zProjGeo}{Projection of Occurrence Densities to the Geography}{ecospat.niche.zProjGeo}
%
\begin{Description}\relax
Creates a raster in geography with each pixel containing the occurrence densities extracted from a z object generated with \code{ecospat.grid.clim.dyn}.
\end{Description}
%
\begin{Usage}
\begin{verbatim}
ecospat.niche.zProjGeo(z1,env,cor)
\end{verbatim}
\end{Usage}
%
\begin{Arguments}
\begin{ldescription}
\item[\code{z1}] Species 1 occurrence density grid created by \code{ecospat.grid.clim.dyn}.
\item[\code{env}] A RasterStack or RasterBrick of environmental variables corresponding to the background (\code{glob} in \code{ecospat.grid.clim.dyn}).
\item[\code{cor}] FALSE by default. If TRUE corrects the occurrence densities of each species by the prevalence of the environments in their range
\end{ldescription}
\end{Arguments}
%
\begin{Details}\relax
extracts the occurrence density of z objects created by \code{ecospat.grid.clim.dyn} for each point of the background (\code{glob}) using \code{extract} (package raster). The values are binded to the geographic coordinates of \code{env} and a raster is then recreated using \code{rasterFromXYZ}
\end{Details}
%
\begin{Value}
raster of class RasterLayer
\end{Value}
%
\begin{Author}\relax
Olivier Broennimann \email{olivier.broennimann@unil.ch}
\end{Author}
%
\begin{References}\relax
Broennimann, O., M.C. Fitzpatrick, P.B. Pearman, B. Petitpierre, L. Pellissier, N.G. Yoccoz, W. Thuiller, M.J. Fortin, C. Randin, N.E. Zimmermann, C.H. Graham and A. Guisan. 2012. Measuring ecological niche overlap from occurrence and spatial environmental data. \emph{Global Ecology and Biogeography}, \bold{21}:481-497.


Petitpierre, B., C. Kueffer, O. Broennimann, C. Randin, C. Daehler and A. Guisan. 2012. Climatic niche shifts are rare among terrestrial plant invaders. \emph{Science}, \bold{335}:1344-1348.
\end{References}
%
\begin{SeeAlso}\relax
\code{\LinkA{ecospat.plot.niche.dyn}{ecospat.plot.niche.dyn}}, \code{\LinkA{ecospat.niche.dynIndexProjGeo}{ecospat.niche.dynIndexProjGeo}}
\end{SeeAlso}
%
\begin{Examples}
\begin{ExampleCode}
## Not run: 

library(raster)

spp <- ecospat.testNiche
xy.sp1<-subset(spp,species=="sp1")[2:3] #Bromus_erectus

load(system.file("extdata", "ecospat.testEnvRaster.Rdata", package="ecospat"))
#?ecospat.testEnvRaster

env.sp1<-extract(env,xy.sp1)
env.bkg<-na.exclude(values(env))

#################################### PCA-ENVIRONMENT ##################################

pca.cal <- dudi.pca(env.bkg, center = TRUE, scale = TRUE, scannf = FALSE, nf = 2)

# predict the scores on the axes
scores.bkg <- pca.cal$li  #scores for background climate
scores.sp1 <- suprow(pca.cal,env.sp1)$lisup				#scores for sp1

# calculation of occurence density (niche z)
  
z1 <- ecospat.grid.clim.dyn(scores.bkg, scores.bkg, scores.sp1,R=100)

plot(z1$z.uncor)
points(scores.sp1)

#################################### occurrence density in space ##################################

# sp1
geoz1<-ecospat.niche.zProjGeo(z1,env)
plot(geoz1,main="z1")
points(xy.sp1)

## End(Not run)
\end{ExampleCode}
\end{Examples}
\inputencoding{utf8}
\HeaderA{ecospat.npred}{Number Of Predictors}{ecospat.npred}
%
\begin{Description}\relax
Calculate the maximum number of predictors to include in the model with a desired correlation between predictors.
\end{Description}
%
\begin{Usage}
\begin{verbatim}
ecospat.npred (x, th)
\end{verbatim}
\end{Usage}
%
\begin{Arguments}
\begin{ldescription}
\item[\code{x}] Correlation matrix of the predictors.
\item[\code{th}] Desired threshold of correlation between predictors.
\end{ldescription}
\end{Arguments}
%
\begin{Value}
Returns the number of predictors to use.
\end{Value}
%
\begin{Author}\relax
Blaise Petitpierre \email{bpetitpierre@gmail.com}
\end{Author}
%
\begin{Examples}
\begin{ExampleCode}
colvar <- ecospat.testData[c(4:8)]
x <- cor(colvar, method="pearson")
ecospat.npred (x, th=0.75)
\end{ExampleCode}
\end{Examples}
\inputencoding{utf8}
\HeaderA{ecospat.occ.desaggregation}{Species Occurrences Desaggregation}{ecospat.occ.desaggregation}
%
\begin{Description}\relax
Remove species occurrences in a dataframe which are closer to each other than a specified distance threshold.
\end{Description}
%
\begin{Usage}
\begin{verbatim}
ecospat.occ.desaggregation (xy, min.dist, by)
\end{verbatim}
\end{Usage}
%
\begin{Arguments}
\begin{ldescription}
\item[\code{xy}] A dataframe with xy-coordinates (x-column must be named 'x' and y-column 'y')
\item[\code{min.dist}] The minimun distance between points in the sub-dataframe.
\item[\code{by}] Grouping element in the dataframe (e.g. species, NULL)
\end{ldescription}
\end{Arguments}
%
\begin{Details}\relax
This function will desaggregate the original number of occurrences, according to a specified distance.
\end{Details}
%
\begin{Value}
A subset of the initial dataframe. The number of points is printed as "initial", "kept" and "out".
\end{Value}
%
\begin{Author}\relax
Frank Breiner \email{frank.breiner@unil.ch} 

with contributions of Olivier Broennimann \email{olivier.broennimann@unil.ch}
\end{Author}
%
\begin{Examples}
\begin{ExampleCode}

## Not run: 
spp <- ecospat.testNiche
colnames(spp)[2:3] <- c('x','y')
sp1 <- spp[1:32,2:3]

occ.sp1 <- ecospat.occ.desaggregation(xy=sp1, min.dist=500, by=NULL)
occ.all.sp <- ecospat.occ.desaggregation(xy=spp, min.dist=500, by='Spp')

## End(Not run)
\end{ExampleCode}
\end{Examples}
\inputencoding{utf8}
\HeaderA{ecospat.occupied.patch}{Extract occupied patches of a species in geographic space.)}{ecospat.occupied.patch}
\keyword{file}{ecospat.occupied.patch}
%
\begin{Description}\relax
This function determines the occupied patch of a species using standard IUCN criteria (AOO, EOO) or predictive binary maps from Species Distribution Models.
\end{Description}
%
\begin{Usage}
\begin{verbatim}
ecospat.occupied.patch (bin.map, Sp.occ.xy, buffer = 0)
\end{verbatim}
\end{Usage}
%
\begin{Arguments}
\begin{ldescription}
 
\item[\code{bin.map}] Binary map (single layer or raster stack) from a Species Distribution Model.
\item[\code{Sp.occ.xy}] xy-coordinates of the species presence.
\item[\code{buffer}] numeric. Calculate occupied patch models from the binary map using a buffer (predicted area occupied by the species or within a buffer around the species, for details see ?extract).
\end{ldescription}
\end{Arguments}
%
\begin{Details}\relax
Predictive maps derived from SDMs inform about the potential distribution (or habitat suitability) of a species. Often it is useful to get information about the area of the potential distribution which is occupied by a species, e.g. for Red List assessments.
\end{Details}
%
\begin{Value}
a RasterLayer with value 1.
\end{Value}
%
\begin{Author}\relax
Frank Breiner \email{frank.breiner@wsl.ch}
\end{Author}
%
\begin{References}\relax

IUCN Standards and Petitions Subcommittee. 2016. Guidelines for Using the IUCN Red List Categories and Criteria. Version 12. Prepared by the Standards and Petitions Subcommittee. Downloadable from http://www.iucnredlist.org/documents/RedListGuidelines.pdf

\end{References}
%
\begin{SeeAlso}\relax
\code{\LinkA{ecospat.rangesize}{ecospat.rangesize}}, \code{\LinkA{ecospat.mpa}{ecospat.mpa}}, \code{\LinkA{ecospat.binary.model}{ecospat.binary.model}}
\end{SeeAlso}
%
\begin{Examples}
\begin{ExampleCode}

## Not run: 

library(dismo)


library(dismo)


# only run if the maxent.jar file is available, in the right folder
jar <- paste(system.file(package="dismo"), "/java/maxent.jar", sep='')

# checking if maxent can be run (normally not part of your script)
file.exists(jar)
require(rJava))

# get predictor variables
fnames <- list.files(path=paste(system.file(package="dismo"), '/ex', sep=''), 
                     pattern='grd', full.names=TRUE )
predictors <- stack(fnames)
#plot(predictors)

# file with presence points
occurence <- paste(system.file(package="dismo"), '/ex/bradypus.csv', sep='')
occ <- read.table(occurence, header=TRUE, sep=',')[,-1]
colnames(occ) <- c("x","y")
occ <- ecospat.occ.desaggregation(occ,min.dist=1)

# fit model, biome is a categorical variable
me <- maxent(predictors, occ, factors='biome')

# predict to entire dataset
pred <- predict(me, predictors) 

plot(pred)
points(occ)


# use MPA to convert suitability to binary map
mpa.cutoff <- ecospat.mpa(pred,occ)

pred.bin.mpa <- ecospat.binary.model(pred,mpa.cutoff)
names(pred.bin.mpa) <- "me.mpa"
pred.bin.arbitrary <- ecospat.binary.model(pred,0.5)
names(pred.bin.arbitrary) <- "me.arbitrary"


mpa.ocp  <- ecospat.occupied.patch(pred.bin.mpa,occ)
arbitrary.ocp  <- ecospat.occupied.patch(pred.bin.arbitrary,occ)

par(mfrow=c(1,2))
plot(mpa.ocp) ## occupied patches: green area
points(occ,col="red",cex=0.5,pch=19)
plot(arbitrary.ocp)
points(occ,col="red",cex=0.5,pch=19)

## with buffer:
mpa.ocp  <- ecospat.occupied.patch(pred.bin.mpa,occ, buffer=500000)
arbitrary.ocp  <- ecospat.occupied.patch(pred.bin.arbitrary,occ, buffer=500000)

plot(mpa.ocp) ## occupied patches: green area
points(occ,col="red",cex=0.5,pch=19)
plot(arbitrary.ocp)
points(occ,col="red",cex=0.5,pch=19)

## End(Not run)

\end{ExampleCode}
\end{Examples}
\inputencoding{utf8}
\HeaderA{ecospat.permut.glm}{GLM Permutation Function}{ecospat.permut.glm}
%
\begin{Description}\relax
A permutation function to get p-values on GLM coefficients and deviance.
\end{Description}
%
\begin{Usage}
\begin{verbatim}
ecospat.permut.glm (glm.obj, nperm)
\end{verbatim}
\end{Usage}
%
\begin{Arguments}
\begin{ldescription}
\item[\code{glm.obj}] Any calibrated GLM or GAM object with a binomial error distribution.
\item[\code{nperm}] The number of permutations in the randomization process.
\end{ldescription}
\end{Arguments}
%
\begin{Details}\relax
Rows of the response variable are permuted and a new GLM is calibrated as well as deviance, adjusted deviance and coefficients are calculated. These random parameters are compared to the true parameters in order to derive p-values.
\end{Details}
%
\begin{Value}
Return p-values that are how the true parameters of the original model deviate from the disribution of the random parameters. A p-value of zero means that the true parameter is completely outside the random distribution.
\end{Value}
%
\begin{Author}\relax
Christophe Randin \email{christophe.randin@unibas.ch}, Antoine Guisan \email{antoine.guisan@unil.ch} and Trevor Hastie
\end{Author}
%
\begin{References}\relax
Hastie, T., R. Tibshirani and J. Friedman. 2001. \emph{Elements of Statistical Learning; Data Mining, Inference, and Prediction}, Springer-Verlag, New York.

Legendre, P. and L. Legendre. 1998. \emph{Numerical Ecology}, 2nd English edition. Elsevier Science BV, Amsterdam.
\end{References}
%
\begin{Examples}
\begin{ExampleCode}

## Not run: 
ecospat.permut.glm (get ("glm.Achillea_atrata", envir=ecospat.env), 1000)

## End(Not run)
\end{ExampleCode}
\end{Examples}
\inputencoding{utf8}
\HeaderA{ecospat.plot.contrib}{Plot Variables Contribution}{ecospat.plot.contrib}
%
\begin{Description}\relax
Plot the contribution of the initial variables to the analysis (i.e. correlation circle). Typically these are the eigen vectors and eigen values in ordinations.
\end{Description}
%
\begin{Usage}
\begin{verbatim}
ecospat.plot.contrib (contrib, eigen)
\end{verbatim}
\end{Usage}
%
\begin{Arguments}
\begin{ldescription}
\item[\code{contrib}] A dataframe of the contribution of each original variable on each axis of the analysis, i.e. the eigen vectors. 
\item[\code{eigen}] A vector of the importance of the axes in the ordination, i.e. a vector of eigen values.
\end{ldescription}
\end{Arguments}
%
\begin{Details}\relax
Requires ade4 library. If using \code{\LinkA{princomp}{princomp}} , use \$loadings and \$sdev of the princomp object. if using \code{\LinkA{dudi.pca}{dudi.pca}}, use \$li and \$eig of the dudi.pca object.
\end{Details}
%
\begin{Author}\relax
Olivier Broennimann \email{olivier.broennimann@unil.ch}
\end{Author}
%
\begin{References}\relax
Broennimann, O., M.C. Fitzpatrick, P.B. Pearman, B. Petitpierre, L. Pellissier, N.G. Yoccoz, W. Thuiller, M.J. Fortin, C. Randin, N.E. Zimmermann, C.H. Graham and A. Guisan. 2012. Measuring ecological niche overlap from occurrence and spatial environmental data. \emph{Global Ecology and Biogeography}, \bold{21}, 481-497.
\end{References}
%
\begin{SeeAlso}\relax
\code{\LinkA{ecospat.plot.niche.dyn}{ecospat.plot.niche.dyn}},\code{\LinkA{ecospat.plot.overlap.test}{ecospat.plot.overlap.test}},
\code{\LinkA{ecospat.niche.similarity.test}{ecospat.niche.similarity.test}},\code{princomp}
\end{SeeAlso}
\inputencoding{utf8}
\HeaderA{ecospat.plot.kappa}{Plot Kappa}{ecospat.plot.kappa}
\keyword{file}{ecospat.plot.kappa}
%
\begin{Description}\relax
Plots the values for Cohen's Kappa along different thresholds.
\end{Description}
%
\begin{Usage}
\begin{verbatim}
    ecospat.plot.kappa(Pred, Sp.occ)
\end{verbatim}
\end{Usage}
%
\begin{Arguments}
\begin{ldescription}
\item[\code{Pred}] 
A vector of predicted probabilities

\item[\code{Sp.occ}] 
A vector of binary observations of the species occurrence

\end{ldescription}
\end{Arguments}
%
\begin{Value}
A plot of the Cohen's Kappa values along different thresholds.
\end{Value}
%
\begin{Author}\relax
Luigi Maiorano \email{luigi.maiorano@gmail.com} with contributions of Valeria Di Cola \email{valeria.dicola@unil.ch}.
\end{Author}
%
\begin{References}\relax
Liu, C., P.M. Berry, T.P. Dawson, and R.G. Pearson. 2005. Selecting thresholds of occurrence in the prediction of species distributions. \emph{Ecography}, \bold{28}, 385-393.

Landis, J.R. and G.G. Koch. 1977. The measurement of observer agreement for categorical data. \emph{biometrics}, \bold{33},159-174.

\end{References}
%
\begin{SeeAlso}\relax
\code{\LinkA{ecospat.meva.table}{ecospat.meva.table}}, \code{\LinkA{ecospat.max.tss}{ecospat.max.tss}}, \code{\LinkA{ecospat.plot.tss}{ecospat.plot.tss}}, \code{\LinkA{ecospat.cohen.kappa}{ecospat.cohen.kappa}}, \code{\LinkA{ecospat.max.kappa}{ecospat.max.kappa}}
\end{SeeAlso}
%
\begin{Examples}
\begin{ExampleCode}


Pred <- ecospat.testData$glm_Agrostis_capillaris
Sp.occ <- ecospat.testData$Agrostis_capillaris
ecospat.plot.kappa(Pred, Sp.occ)
\end{ExampleCode}
\end{Examples}
\inputencoding{utf8}
\HeaderA{ecospat.plot.mess}{Plot MESS}{ecospat.plot.mess}
%
\begin{Description}\relax
Plot the MESS extrapolation index onto the geographical space.
\end{Description}
%
\begin{Usage}
\begin{verbatim}
ecospat.plot.mess (mess.object, cex=1, pch=15)
\end{verbatim}
\end{Usage}
%
\begin{Arguments}
\begin{ldescription}
\item[\code{xy}] The x and y coordinates of the projection dataset.
\item[\code{mess.object}] A dataframe as returned by the \code{ecospat.mess} function.
\item[\code{cex}] Specify the size of the symbol.
\item[\code{pch}] Specify the point symbols.
\end{ldescription}
\end{Arguments}
%
\begin{Value}
Returns a plot of the the MESS extrapolation index onto the geographical space.
\end{Value}
%
\begin{Author}\relax
Blaise Petitpierre \email{bpetitpierre@gmail.com}
\end{Author}
%
\begin{References}\relax
Elith, J., M. Kearney and S. Phillips. 2010. The art of modelling range-shifting species. \emph{Methods in ecology and evolution}, \bold{1}, 330-342.
\end{References}
%
\begin{SeeAlso}\relax
\code{\LinkA{ecospat.mess}{ecospat.mess}}
\end{SeeAlso}
%
\begin{Examples}
\begin{ExampleCode}
## Not run: 
x <- ecospat.testData[c(2,3,4:8)]
proj <- x[1:90,] #A projection dataset.
cal <- x[91:300,] #A calibration dataset

#Create a MESS object 
mess.object <- ecospat.mess (proj, cal, w="default")

#Plot MESS 
ecospat.plot.mess (mess.object, cex=1, pch=15)

## End(Not run)
\end{ExampleCode}
\end{Examples}
\inputencoding{utf8}
\HeaderA{ecospat.plot.niche}{Plot Niche}{ecospat.plot.niche}
%
\begin{Description}\relax
Plot a niche z created by \code{ecospat.grid.clim.dyn}.
\end{Description}
%
\begin{Usage}
\begin{verbatim}
ecospat.plot.niche (z, title, name.axis1, name.axis2, cor=FALSE)
\end{verbatim}
\end{Usage}
%
\begin{Arguments}
\begin{ldescription}
\item[\code{z}] A gridclim object for the species distribution created by \code{ecospat.grid.clim.dyn}.
\item[\code{title}] A title for the plot.
\item[\code{name.axis1}] A label for the first axis.
\item[\code{name.axis2}] A label for the second axis.
\item[\code{cor}] Correct the occurrence densities of the species by the prevalence of the environments in its range (TRUE = yes, FALSE = no).
\end{ldescription}
\end{Arguments}
%
\begin{Details}\relax
if z is bivariate, a bivariate plot of the niche of the species. if z is univariate, a histogram of the niche of the species
\end{Details}
%
\begin{Author}\relax
Olivier Broennimann \email{olivier.broennimann@unil.ch}
\end{Author}
%
\begin{References}\relax
Broennimann, O., M.C. Fitzpatrick, P.B. Pearman, B. Petitpierre, L. Pellissier, N.G. Yoccoz, W. Thuiller, M.J. Fortin, C. Randin, N.E. Zimmermann, C.H. Graham and A. Guisan. 2012. Measuring ecological niche overlap from occurrence and spatial environmental data. \emph{Global Ecology and Biogeography}, \bold{21}, 481-497.
\end{References}
%
\begin{SeeAlso}\relax
\code{\LinkA{ecospat.grid.clim.dyn}{ecospat.grid.clim.dyn}}
\end{SeeAlso}
\inputencoding{utf8}
\HeaderA{ecospat.plot.niche.dyn}{Niche Categories and Species Density}{ecospat.plot.niche.dyn}
%
\begin{Description}\relax
Plot niche categories and species density created by \code{ecospat.grid.clim.dyn}.
\end{Description}
%
\begin{Usage}
\begin{verbatim}
ecospat.plot.niche.dyn (z1, z2, quant, title, 
name.axis1, name.axis2, interest, colz1, colz2,colinter, colZ1, colZ2)
\end{verbatim}
\end{Usage}
%
\begin{Arguments}
\begin{ldescription}
\item[\code{z1}] A gridclim object for the native distribution.
\item[\code{z2}] A gridclim object for the invaded range.
\item[\code{quant}] The quantile of the environmental density used to delimit marginal climates.
\item[\code{title}] The title of the plot.
\item[\code{name.axis1}] A label for the first axis.
\item[\code{name.axis2}] A label for the second axis
\item[\code{interest}] Choose which density to plot: if \code{interest=1}, plot native density, if \code{interest=2}, plot invasive density.
\item[\code{colz1}] The color used to depict unfilling area.
\item[\code{colz2}] The color used to depict expansion area.
\item[\code{colinter}] The color used to depict overlap area.
\item[\code{colZ1}] The color used to delimit the native extent.
\item[\code{colZ2}] The color used to delimit the invaded extent.

\end{ldescription}
\end{Arguments}
%
\begin{Author}\relax
Blaise Petitpierre \email{bpetitpierre@gmail.com}
\end{Author}
\inputencoding{utf8}
\HeaderA{ecospat.plot.overlap.test}{Plot Overlap Test}{ecospat.plot.overlap.test}
%
\begin{Description}\relax
Plot a histogram of observed and randomly simulated overlaps, with p-values of equivalency or similarity tests.
\end{Description}
%
\begin{Usage}
\begin{verbatim}
ecospat.plot.overlap.test (x, type, title)
\end{verbatim}
\end{Usage}
%
\begin{Arguments}
\begin{ldescription}
\item[\code{x}] Object created by 

\code{ecospat.niche.similarity.test} or \code{ecospat.niche.equivalency.test}.
\item[\code{type}] Must be either ``D'' or ``I''.
\item[\code{title}] The title for the plot.
\end{ldescription}
\end{Arguments}
%
\begin{Author}\relax
Olivier Broennimann \email{olivier.broennimann@unil.ch}
\end{Author}
%
\begin{References}\relax
Broennimann, O., M.C. Fitzpatrick, P.B. Pearman, B. Petitpierre, L. Pellissier, N.G. Yoccoz, W. Thuiller, M.J. Fortin, C. Randin, N.E. Zimmermann, C.H. Graham and A. Guisan. 2012. Measuring ecological niche overlap from occurrence and spatial environmental data. \emph{Global Ecology and Biogeography}, \bold{21}, 481-497.
\end{References}
%
\begin{SeeAlso}\relax
\code{\LinkA{ecospat.niche.similarity.test}{ecospat.niche.similarity.test}}, \code{\LinkA{ecospat.niche.equivalency.test}{ecospat.niche.equivalency.test}}
\end{SeeAlso}
\inputencoding{utf8}
\HeaderA{ecospat.plot.tss}{Plot True skill statistic (TSS)}{ecospat.plot.tss}
\keyword{file}{ecospat.plot.tss}
%
\begin{Description}\relax
Plots the values for True skill statistic (TSS) along different thresholds.
\end{Description}
%
\begin{Usage}
\begin{verbatim}
    ecospat.plot.tss(Pred, Sp.occ)
\end{verbatim}
\end{Usage}
%
\begin{Arguments}
\begin{ldescription}
\item[\code{Pred}] 
A vector of predicted probabilities

\item[\code{Sp.occ}] 
A vector of binary observations of the species occurrence

\end{ldescription}
\end{Arguments}
%
\begin{Value}
A plot of the TSS values along different thresholds.
\end{Value}
%
\begin{Author}\relax
Luigi Maiorano \email{luigi.maiorano@gmail.com}
\end{Author}
%
\begin{References}\relax

Liu, C., P.M. Berry, T.P. Dawson, and R.G. Pearson. 2005. Selecting thresholds of occurrence in the prediction of species distributions. \emph{Ecography}, \bold{28}, 385-393.

Liu, C., M. White and G. Newell. 2013. Selecting thresholds for the prediction of species occurrence with presence-only data. \emph{Journal of Biogeography}, \emph{40}, 778-789.

\end{References}
%
\begin{SeeAlso}\relax
\code{\LinkA{ecospat.meva.table}{ecospat.meva.table}}, \code{\LinkA{ecospat.max.tss}{ecospat.max.tss}}, \code{\LinkA{ecospat.plot.kappa}{ecospat.plot.kappa}}, \code{\LinkA{ecospat.cohen.kappa}{ecospat.cohen.kappa}}, \code{\LinkA{ecospat.max.kappa}{ecospat.max.kappa}}
\end{SeeAlso}
%
\begin{Examples}
\begin{ExampleCode}
Pred <- ecospat.testData$glm_Agrostis_capillaris
Sp.occ <- ecospat.testData$Agrostis_capillaris
ecospat.plot.tss(Pred, Sp.occ)
\end{ExampleCode}
\end{Examples}
\inputencoding{utf8}
\HeaderA{ecospat.rand.pseudoabsences}{Sample Pseudo-Absences}{ecospat.rand.pseudoabsences}
%
\begin{Description}\relax
Randomly sample pseudoabsences from an environmental dataframe covering the study area.
\end{Description}
%
\begin{Usage}
\begin{verbatim}
ecospat.rand.pseudoabsences (nbabsences, glob, colxyglob, colvar="all", 
presence, colxypresence, mindist)
\end{verbatim}
\end{Usage}
%
\begin{Arguments}
\begin{ldescription}
\item[\code{nbabsences}] The number of pseudoabsences desired.
\item[\code{glob}] A two-column dataframe (or a vector) of the environmental values (in column) for background pixels of the whole study area (in row).
\item[\code{colxyglob}] The range of columns for x and y in glob.
\item[\code{colvar}] The range of columns for the environmental variables in glob. colvar="all" keeps all the variables in glob in the final dataframe. colvar=NULL keeps only x and y.
\item[\code{presence}] A presence-absence dataframe for each species (columns) in each location or grid cell (rows).
\item[\code{colxypresence}] The range of columns for x and y in presence.
\item[\code{mindist}] The minimum distance from presences within wich pseudoabsences should not be drawn (buffer distance around presences).
\end{ldescription}
\end{Arguments}
%
\begin{Value}
A dataframe of random absences.
\end{Value}
%
\begin{Author}\relax
Olivier Broennimann \email{olivier.broennimann@unil.ch}
\end{Author}
%
\begin{Examples}
\begin{ExampleCode}
glob <- ecospat.testData[2:8]
presence <- ecospat.testData[c(2:3,9)]
presence <- presence[presence[,3]==1,1:2]
ecospat.rand.pseudoabsences (nbabsences=10, glob=glob, colxyglob=1:2, colvar = "all", 
presence= presence, colxypresence=1:2, mindist=20)
\end{ExampleCode}
\end{Examples}
\inputencoding{utf8}
\HeaderA{ecospat.rangesize}{Quantification of the range size of a species using habitat suitability maps and IUCN criteria)}{ecospat.rangesize}
\keyword{file}{ecospat.rangesize}
%
\begin{Description}\relax
This function quantifies the range size of a species using standard IUCN criteria (Area of Occupancy AOO, Extent of Occurence EOO) or binary maps derived from Species Distribution Models.
\end{Description}
%
\begin{Usage}
\begin{verbatim}
ecospat.rangesize (bin.map, ocp, buffer, eoo.around.model, eoo.around.modelocp, 
xy, EOO, Model.within.eoo, AOO, resol, AOO.circles, d, lonlat, return.obj, 
save.obj, save.rangesize, directory)

ecospat.rangesize (bin.map = NULL,
                   ocp = T,
                   buffer = 0,
                   eoo.around.model = T,
                   eoo.around.modelocp = F,
                   xy = NULL,
                   EOO = T,
                   Model.within.eoo = T,
                   AOO = T,
                   resol = c(2000, 2000),
                   AOO.circles = F,
                   d = sqrt((2000 * 2)/pi),
                   lonlat = FALSE,
                   return.obj = T,
                   save.obj = F,
                   save.rangesize = F,
                   directory = getwd())



\end{verbatim}
\end{Usage}
%
\begin{Arguments}
\begin{ldescription}
\item[\code{bin.map}] Binary map (single layer or raster stack) from a Species Distribution Model.
\item[\code{ocp}] logical. Calculate occupied patch models from the binary map (predicted area occupied by the species)
\item[\code{buffer}] numeric. Calculate occupied patch models from the binary map using a buffer (predicted area occupied by the species or within a buffer around the species, for details see ?extract).
\item[\code{eoo.around.model}] logical. The EOO around all positive predicted values from the binary map.
\item[\code{eoo.around.modelocp}] logical. EOO around all positive predicted values of occupied patches.
\item[\code{xy}] xy-coordinates of the species presence
\item[\code{EOO}] logical. Extent of Occurrence. Convex Polygon around occurrences.
\item[\code{Model.within.eoo}] logical. Area predicted as suitable by the model within EOO.
\item[\code{AOO}] logical. Area of Occupancy ddervied by the occurrences.
\item[\code{resol}] Resolution of the grid frame at which AOO should be calculated.
\item[\code{AOO.circles}] logical. AOO calculated by circles around the occurrences instead of using a grid frame.
\item[\code{d}] Radius of the AOO.circles around the occurrences.
\item[\code{lonlat}] Are these longitude/latidue coordinates? (Default = FALSE).
\item[\code{return.obj}] logical. should the objects created to estimate range size be returned (rasterfiles and spatial polygons). Default: TRUE
\item[\code{save.obj}] logical. should objects be saved on hard drive?
\item[\code{save.rangesize}] logical. should range size estimations be saved on hard drive .
\item[\code{directory}] directory in which objects should be saved (Default = getwd())

\end{ldescription}
\end{Arguments}
%
\begin{Details}\relax
The range size of a species is important for many conservation purposes, e.g. to assess the status of threat for IUCN Red Lists. This function quantifies the range size using different IUCN measures, i.e. the Area Of Occupancy (AOO), the Extent Of Occurrence (EOO) and from binary maps derived from Species Distribution Models (SDMs). Different ways to extract range size from SDMs are available, e.g. using occupied patches, the suitable habitat within EOO or a convex hull around the suitable habitat.
\end{Details}
%
\begin{Value}
A list with the values of range size quantification and the stored objects used for quantification (of class RasterLayers, ahull, ConvexHull).
\end{Value}
%
\begin{Author}\relax
Frank Breiner \email{frank.breiner@wsl.ch}
\end{Author}
%
\begin{References}\relax

IUCN. 2012. IUCN Red List Categories and Criteria: Version 3.1. Second edition. Gland, Switzerland and Cambridge, UK: IUCN. iv + 32pp.

IUCN Standards and Petitions Subcommittee. 2016. Guidelines for Using the IUCN Red List Categories and Criteria. Version 12. Prepared by the Standards and Petitions Subcommittee. Downloadable from http://www.iucnredlist.org/documents/RedListGuidelines.pdf

Pateiro-Lopez, B., and A. Rodriguez-Casal. 2010. Generalizing the Convex Hull of a Sample: The R Package alphahull. \emph{Journal of Statistical software}, \bold{34}, 1-28.
\end{References}
%
\begin{SeeAlso}\relax
\code{\LinkA{ecospat.occupied.patch}{ecospat.occupied.patch}}, \code{\LinkA{ecospat.mpa}{ecospat.mpa}}, \code{\LinkA{ecospat.binary.model}{ecospat.binary.model}}
\end{SeeAlso}
%
\begin{Examples}
\begin{ExampleCode}
 library(dismo)


# only run if the maxent.jar file is available, in the right folder
jar <- paste(system.file(package="dismo"), "/java/maxent.jar", sep='')

# checking if maxent can be run (normally not part of your script)
file.exists(jar)
require(rJava)

# get predictor variables
fnames <- list.files(path=paste(system.file(package="dismo"), '/ex', sep=''), 
                     pattern='grd', full.names=TRUE )
predictors <- stack(fnames)
#plot(predictors)

# file with presence points
occurence <- paste(system.file(package="dismo"), '/ex/bradypus.csv', sep='')
occ <- read.table(occurence, header=TRUE, sep=',')[,-1]
colnames(occ) <- c("x","y")
occ <- ecospat.occ.desaggregation(occ,min.dist=1)

# fit model, biome is a categorical variable
me <- maxent(predictors, occ, factors='biome')

# predict to entire dataset
pred <- predict(me, predictors) 

plot(pred)
points(occ)


# use MPA to convert suitability to binary map
mpa.cutoff <- ecospat.mpa(pred,occ)

# use Boyce index to convert suitability to binary map
boyce <- ecospat.boyce(pred,  occ)
### use the boyce index to find a threshold
pred.bin.arbitrary <- ecospat.binary.model(pred,0.5)


pred.bin.mpa <- ecospat.binary.model(pred,mpa.cutoff)
names(pred.bin.mpa) <- "me.mpa"
pred.bin.arbitrary <- ecospat.binary.model(pred,0.5)
names(pred.bin.arbitrary) <- "me.arbitrary"

rangesize <- ecospat.rangesize(stack(pred.bin.mpa,pred.bin.arbitrary),
                               xy=occ,
                               resol=c(1,1),
                               eoo.around.modelocp =T,
                               AOO.circles = T,
                               d=200000,
                               lonlat =T)


## Range size quantification
rangesize$RangeSize

names(rangesize$RangeObjects)


par(mfrow=c(1,3))

plot(ecospat.binary.model(pred,0),legend=F, main="IUCN criteria")
## IUCN criteria & derivates
# plot AOO
plot(rangesize$RangeObjects$AOO,add=T, col="red",legend=F)
# plot EOO
plot(rangesize$RangeObjects$EOO@polygons,add=T, border="red", lwd=2)
# plot circles around occurrences
plot(rangesize$RangeObjects$AOO.circle@polygons,add=T,border="blue")

for(i in 1:2){
## plot the occupied patches of the model
plot(rangesize$RangeObjects$models.ocp[[i]],col=c("grey","blue","darkgreen"),
main=names(rangesize$RangeObjects$models.ocp[[i]]),legend=F)
points(occ,col="red",cex=0.5,pch=19)
## plot EOO around model
plot(rangesize$RangeObjects$eoo.around.model[[i]]@polygons,add=T,border="blue",lwd=2)
## plot EOO around occupied patches
plot(rangesize$RangeObjects$eoo.around.mo.ocp[[i]]@polygons,add=T,border="darkgreen",
lwd=2)
## plot the modeled area within EOO
#plot(rangesize$RangeObjects$model.within.eoo[[i]],col=c("grey","blue","darkgreen"),legend=F)
#points(occ,col="red",cex=0.5,pch=19)
#plot(rangesize$RangeObjects$EOO@polygons,add=T, border="red", lwd=2)
}

### Alpha-hulls are not included in the function yet because of Licence limitations.
### However, alpha-hulls can easily be included manually (see also the help file of 
### the alpha hull package):

require(alphahull)
  alpha = 2 # alpha value of 2 recommended by IUCN
  
  del<-delvor(occ)
  dv<-del$mesh
  mn <- mean(sqrt(abs(del$mesh[,3]-del$mesh[,5])^2+abs(del$mesh[,4]-del$mesh[,6])^2))*alpha
  alpha.hull<-ahull(del,alpha=mn) 
  
  #Size of alpha-hulls
  areaahull(h)


# plot alphahulls
plot(rangesize$RangeObjects$models.ocp[[i]],col=c("grey","blue","darkgreen"),
  main=names(rangesize$RangeObjects$models.ocp[[i]]),legend=F)
plot(alpha.hull,add=T,lwd=1)
 
 
\end{ExampleCode}
\end{Examples}
\inputencoding{utf8}
\HeaderA{ecospat.rcls.grd}{Reclassifying grids function}{ecospat.rcls.grd}
%
\begin{Description}\relax
Function for reclassifying grid files to get a combined statification from more than one grid
\end{Description}
%
\begin{Usage}
\begin{verbatim}
ecospat.rcls.grd(in_grid,no.classes)
\end{verbatim}
\end{Usage}
%
\begin{Arguments}
\begin{ldescription}
\item[\code{in\_grid}] The grid to be reclassified.
\item[\code{no.classes}] The number of desired new classes.
\end{ldescription}
\end{Arguments}
%
\begin{Details}\relax
This function reclassifies the input grid into a number of new classes that the user defines. The boundaries of each class are decided automatically by splitting the range of values of the input grid into the user defined number of classes.
\end{Details}
%
\begin{Value}
Returns a reclassified Raster object
\end{Value}
%
\begin{Author}\relax
Achilleas Psomas \email{achilleas.psomas@wsl.ch} and Niklaus E. Zimmermann \email{niklaus.zimmermann@wsl.ch}
\end{Author}
%
\begin{Examples}
\begin{ExampleCode}

   ## Not run: 
bio3<- raster(system.file("external/bioclim/current/bio3.grd",package="biomod2"))
bio12<- raster(system.file("external/bioclim/current/bio12.grd",package="biomod2"))
B3.rcl<-ecospat.rcls.grd(bio3,9) 
B12.rcl<-ecospat.rcls.grd(bio12,9)
B3B12.comb <- B12.rcl+B3.rcl*10

# Plotting a histogram of the classes
hist(B3B12.comb,breaks=100,col=heat.colors(88)) 
# Plotting the new RasterLayer (9x9 classes)
plot(B3B12.comb,col=rev(rainbow(88)),main="Stratified map") 


## End(Not run)
\end{ExampleCode}
\end{Examples}
\inputencoding{utf8}
\HeaderA{ecospat.recstrat\_prop}{Random Ecologically Stratified Sampling of propotional numbers}{ecospat.recstrat.Rul.prop}
%
\begin{Description}\relax
This function randomly collects a user-defined total number of samples from the stratification layer. 
\end{Description}
%
\begin{Usage}
\begin{verbatim}
  ecospat.recstrat_prop(in_grid, sample_no)
\end{verbatim}
\end{Usage}
%
\begin{Arguments}
\begin{ldescription}
\item[\code{in\_grid}] 
The stratification grid to be sampled.

\item[\code{sample\_no}] 
The total number of pixels to be sampled.

\end{ldescription}
\end{Arguments}
%
\begin{Details}\relax
The number of samples per class are determined proportional to the abundance of each class.
The number of classes in the stratification layer are determined automatically from the integer input map.
If the proportion of samples for a certain class is below one then no samples are collected for this class.
\end{Details}
%
\begin{Value}
Returns a dataframe with the selected sampling locations their coordinates and the strata they belong in.
\end{Value}
%
\begin{Author}\relax
Achilleas Psomas \email{achilleas.psomas@wsl.ch} and Niklaus E. Zimmermann \email{niklaus.zimmermann@wsl.ch}
\end{Author}
%
\begin{SeeAlso}\relax
\code{\LinkA{ecospat.recstrat\_regl}{ecospat.recstrat.Rul.regl}}
\code{\LinkA{ecospat.rcls.grd}{ecospat.rcls.grd}}
\end{SeeAlso}
%
\begin{Examples}
\begin{ExampleCode}
  ## Not run: 
    bio3<- raster(system.file("external/bioclim/current/bio3.grd",package="biomod2"))
    bio12<- raster(system.file("external/bioclim/current/bio12.grd",package="biomod2"))
    B3.rcl<-ecospat.rcls.grd(bio3,9) 
    B12.rcl<-ecospat.rcls.grd(bio12,9)
    B3B12.comb <- B12.rcl+B3.rcl*10
    B3B12.prop_samples <- ecospat.recstrat_prop(B3B12.comb,100)
    plot(B3B12.comb)
    points(B3B12.prop_samples$x,B3B12.prop_samples$y,pch=16,cex=0.6,col=B3B12.prop_samples$class)
    
    
  
## End(Not run)
  
\end{ExampleCode}
\end{Examples}
\inputencoding{utf8}
\HeaderA{ecospat.recstrat\_regl}{Random Ecologically Stratified Sampling of equal numbers}{ecospat.recstrat.Rul.regl}
%
\begin{Description}\relax
This function randomly takes an equal number of samples per class in the stratification layer. 
\end{Description}
%
\begin{Usage}
\begin{verbatim}
  ecospat.recstrat_regl(in_grid, sample_no)
\end{verbatim}
\end{Usage}
%
\begin{Arguments}
\begin{ldescription}
\item[\code{in\_grid}] 
The stratification grid to be sampled.

\item[\code{sample\_no}] 
The total number of pixels to be sampled.

\end{ldescription}
\end{Arguments}
%
\begin{Details}\relax
The number of classes in the stratification layer is determined automatically from the integer input map. 
If the number of pixels in a class is higher than the number of samples, 
then a random selection without re-substitution is performed, 
otherwise all pixels of that class are selected.
\end{Details}
%
\begin{Value}
Returns a dataframe with the selected sampling locations their coordinates and the strata they belong in.
\end{Value}
%
\begin{Author}\relax
Achilleas Psomas \email{achilleas.psomas@wsl.ch} and Niklaus E. Zimmermann \email{niklaus.zimmermann@wsl.ch}
\end{Author}
%
\begin{SeeAlso}\relax
\code{\LinkA{ecospat.recstrat\_prop}{ecospat.recstrat.Rul.prop}}
\code{\LinkA{ecospat.rcls.grd}{ecospat.rcls.grd}}
\end{SeeAlso}
%
\begin{Examples}
\begin{ExampleCode}
  
  ## Not run: 
    bio3<- raster(system.file("external/bioclim/current/bio3.grd",package="biomod2"))
    bio12<- raster(system.file("external/bioclim/current/bio12.grd",package="biomod2"))
    B3.rcl<-ecospat.rcls.grd(bio3,9) 
    B12.rcl<-ecospat.rcls.grd(bio12,9)
    B3B12.comb <- B12.rcl+B3.rcl*10
    B3B12.regl_samples <- ecospat.recstrat_prop(B3B12.comb,100)
    plot(B3B12.comb)
    points(B3B12.regl_samples$x,B3B12.regl_samples$y,pch=16,cex=0.6,col=B3B12.regl_samples$class)
  
## End(Not run)

\end{ExampleCode}
\end{Examples}
\inputencoding{utf8}
\HeaderA{ecospat.sample.envar}{Sample Environmental Variables}{ecospat.sample.envar}
%
\begin{Description}\relax
Add environmental values to a species dataframe.
\end{Description}
%
\begin{Usage}
\begin{verbatim}
ecospat.sample.envar (dfsp, colspxy, colspkept = "xy", dfvar, 
colvarxy, colvar = "all", resolution)
\end{verbatim}
\end{Usage}
%
\begin{Arguments}
\begin{ldescription}
\item[\code{dfsp}] A species dataframe with x (long), y (lat) and optional other variables.
\item[\code{colspxy}] The range of columns for x (long) and y (lat) in dfsp.
\item[\code{colspkept}] The columns of dfsp that should be kept in the final dataframe (default: xy).
\item[\code{dfvar}] A dataframe object with x, y and environmental variables.
\item[\code{colvarxy}] The range of columns for x and y in dfvar.
\item[\code{colvar}] The range of enviromental variable columns in dfvar (default: all except xy).
\item[\code{resolution}] The distance between x,y of species and environmental datafreme beyond which values shouldn't be added.
\end{ldescription}
\end{Arguments}
%
\begin{Details}\relax
The xy (lat/long) coordinates of the species occurrences are compared to those of the environment dataframe and the value of the closest pixel is added to the species dataframe. When the closest environment pixel is more distant than the given resolution, NA is added instead of the value. This function is similar to sample() in ArcGIS.
\end{Details}
%
\begin{Value}
A Dataframe with the same rows as dfsp, with environmental values from dfvar in column.
\end{Value}
%
\begin{Author}\relax
Olivier Broennimann \email{olivier.broennimann@unil.ch}
\end{Author}
%
\begin{Examples}
\begin{ExampleCode}
## Not run: 
spp <- ecospat.testNiche
sp1 <- spp[1:32,1:3]
occ.sp1 <- ecospat.occ.desaggregation(dfvar=sp1,colxy=2:3,colvar=NULL, min.dist=500,plot=TRUE)
clim <- ecospat.testData[2:8]

occ_sp1 <- na.exclude(ecospat.sample.envar(dfsp=occ.sp1,colspxy=1:2,colspkept=1:2,dfvar=clim,
colvarxy=1:2,colvar="all",resolution=25))

## End(Not run)
\end{ExampleCode}
\end{Examples}
\inputencoding{utf8}
\HeaderA{ecospat.SESAM.prr}{SESAM Probability Ranking Rule}{ecospat.SESAM.prr}
%
\begin{Description}\relax
Implement the SESAM framework to predict community composition using a `probability ranking` rule.
\end{Description}
%
\begin{Usage}
\begin{verbatim}
ecospat.SESAM.prr(proba, sr)
\end{verbatim}
\end{Usage}
%
\begin{Arguments}
\begin{ldescription}
\item[\code{proba}] 
A data frame object of SDMs probabilities (or other sources) for all species. Column names (species names SDM) and row name (sampling sites) (need to have defined row names).

\item[\code{sr}] 
A data frame object with species richness value in the first column. Sites should be arranged in the same order as in the `prob` argument. 

\end{ldescription}
\end{Arguments}
%
\begin{Details}\relax
The SESAM framework implemented in ecospat is based on 1) probabilities of individual species presence for each site - these can be obtained for example by fitting SDMs. This step represents the application of an environmental filter to the community assembly, 2) richness predictions for each site - the richness prediction can be derived in different ways, for instance by summing probabilities from individual species presence for each site  or by fitting direct richness models. This step represents the application of a macroecological constraint to the number of species that can coexist in the considered unit, 3) a biotic rule to decide which species potentially present in the site are retained in the final prediction to match the richness value predicted. The biotic rule applied here is called `probability ranking` rule: the community composition in each site is determined by ranking the species in decreasing order of their predicted probability of presence from SDMs up to a richness prediction.
\end{Details}
%
\begin{Value}
Returns a `.txt` file saved in the working directory that contains the community prediction by the SESAM framework, i.e. binary predictions for all species (columns) for each site (rows).
\end{Value}
%
\begin{Author}\relax
Manuela D`Amen \email{manuela.damen@unil.ch} and Anne Dubuis \email{anne.dubuis@gmail.com}
\end{Author}
%
\begin{References}\relax
D`Amen, M., A. Dubuis, R.F. Fernandes, J. Pottier, L. Pellissier and A. Guisan. 2015. Using species richness and functional traits predictions to constrain assemblage predictions from stacked species distribution models. \emph{J. Biogeogr.}, \bold{42}, 1255-1266.

Guisan, A. and C. Rahbek. 2011. SESAM - a new framework integrating macroecological and species distribution models for predicting spatio-temporal patterns of species assemblages. \emph{J. Biogeogr.}, \bold{38}, 1433-1444.
\end{References}
%
\begin{Examples}
\begin{ExampleCode}
proba <- ecospat.testData[,73:92]
sr <- as.data.frame(rowSums(proba))
ecospat.SESAM.prr(proba, sr)

\end{ExampleCode}
\end{Examples}
\inputencoding{utf8}
\HeaderA{ecospat.shift.centroids}{Draw Centroid Arrows}{ecospat.shift.centroids}
%
\begin{Description}\relax
Draw arrows linking the centroid of the native and exotic (non-native) distribution (continuous line) and between native and invaded extent (dashed line).
\end{Description}
%
\begin{Usage}
\begin{verbatim}
ecospat.shift.centroids(sp1, sp2, clim1, clim2,col)
\end{verbatim}
\end{Usage}
%
\begin{Arguments}
\begin{ldescription}
\item[\code{sp1}] The scores of the species native distribution along the the two first axes of the PCA.
\item[\code{sp2}] The scores of the species invasive distribution along the the two first axes of the PCA.
\item[\code{clim1}] The scores of the entire native extent along the the two first axes of the PCA.
\item[\code{clim2}] The scores of the entire invaded extent along the the two first axes of the PCA.
\item[\code{col}] Colour of the arrow.
\end{ldescription}
\end{Arguments}
%
\begin{Details}\relax
Allows to visualize the shift of the niche centroids of the species and the centroids of the climatic conditions in the study area. To compare invasive species niche, the arrow links the centroid of the native and inasive distribution (continuous line) and between native and invaded extent (dashed line). 
\end{Details}
%
\begin{Value}
Arrow on the overlap test plot
\end{Value}
%
\begin{Author}\relax
Blaise Petitpierre \email{bpetitpierre@gmail.com}
\end{Author}
\inputencoding{utf8}
\HeaderA{ecospat.testData}{Test Data For The Ecospat package}{ecospat.testData}
%
\begin{Description}\relax
Data frame that contains vegetation plots data: presence records of 50 species, a set of environmental variables (topo-climatic) and SDM predictions for some species in the Western Swiss Alps (Canton de Vaud, Switzerland).
\end{Description}
%
\begin{Usage}
\begin{verbatim}
data("ecospat.testData")
\end{verbatim}
\end{Usage}
%
\begin{Format}
A data frame with 300 observations on the following 96 variables.
\begin{description}

\item[\code{numplots}] Number of the vegetation plot.
\item[\code{long}] Longitude, in Swiss plane coordinate system of the vegetation plot.
\item[\code{lat}] Latitude, in Swiss plane coordinate system of the vegetation plot.
\item[\code{ddeg}] Growing degree days (with a 0 degrees Celsius threshold).
\item[\code{mind}] Moisture index over the growing season (average values for June to August in mm day-1).
\item[\code{srad}] The annual sum of radiation (in kJ m-2 year-1).
\item[\code{slp}] Slope (in degrees) calculated from the DEM25.
\item[\code{topo}] Topographic position (an integrated and unitless measure of topographic exposure.
\item[\code{Achillea\_atrata}] 
\item[\code{Achillea\_millefolium}] 
\item[\code{Acinos\_alpinus}] 
\item[\code{Adenostyles\_glabra}] 
\item[\code{Aposeris\_foetida}] 
\item[\code{Arnica\_montana}] 
\item[\code{Aster\_bellidiastrum}] 
\item[\code{Bartsia\_alpina}] 
\item[\code{Bellis\_perennis}] 
\item[\code{Campanula\_rotundifolia}] 
\item[\code{Centaurea\_montana}] 
\item[\code{Cerastium\_latifolium}] 
\item[\code{Cruciata\_laevipes}] 
\item[\code{Doronicum\_grandiflorum}] 
\item[\code{Galium\_album}] 
\item[\code{Galium\_anisophyllon}] 
\item[\code{Galium\_megalospermum}] 
\item[\code{Gentiana\_bavarica}] 
\item[\code{Gentiana\_lutea}] 
\item[\code{Gentiana\_purpurea}] 
\item[\code{Gentiana\_verna}] 
\item[\code{Globularia\_cordifolia}] 
\item[\code{Globularia\_nudicaulis}] 
\item[\code{Gypsophila\_repens}] 
\item[\code{Hieracium\_lactucella}] 
\item[\code{Homogyne\_alpina}] 
\item[\code{Hypochaeris\_radicata}] 
\item[\code{Leontodon\_autumnalis}] 
\item[\code{Leontodon\_helveticus}] 
\item[\code{Myosotis\_alpestris}] 
\item[\code{Myosotis\_arvensis}] 
\item[\code{Phyteuma\_orbiculare}] 
\item[\code{Phyteuma\_spicatum}] 
\item[\code{Plantago\_alpina}] 
\item[\code{Plantago\_lanceolata}] 
\item[\code{Polygonum\_bistorta}] 
\item[\code{Polygonum\_viviparum}] 
\item[\code{Prunella\_grandiflora}] 
\item[\code{Rhinanthus\_alectorolophus}] 
\item[\code{Rumex\_acetosa}] 
\item[\code{Rumex\_crispus}] 
\item[\code{Vaccinium\_gaultherioides}] 
\item[\code{Veronica\_alpina}] 
\item[\code{Veronica\_aphylla}] 
\item[\code{Agrostis\_capillaris}] 
\item[\code{Bromus\_erectus\_sstr}] 
\item[\code{Campanula\_scheuchzeri}] 
\item[\code{Carex\_sempervirens}] 
\item[\code{Cynosurus\_cristatus}] 
\item[\code{Dactylis\_glomerata}] 
\item[\code{Daucus\_carota}] 
\item[\code{Festuca\_pratensis\_sl}] 
\item[\code{Geranium\_sylvaticum}] 
\item[\code{Leontodon\_hispidus\_sl}] 
\item[\code{Potentilla\_erecta}] 
\item[\code{Pritzelago\_alpina\_sstr}] 
\item[\code{Prunella\_vulgaris}] 
\item[\code{Ranunculus\_acris\_sl}] 
\item[\code{Saxifraga\_oppositifolia}] 
\item[\code{Soldanella\_alpina}] 
\item[\code{Taraxacum\_officinale\_aggr}] 
\item[\code{Trifolium\_repens\_sstr}] 
\item[\code{Veronica\_chamaedrys}] 
\item[\code{Parnassia\_palustris}] 
\item[\code{glm\_Agrostis\_capillaris}] GLM model for the species Agrostis\_capillaris.
\item[\code{glm\_Leontodon\_hispidus\_sl}] GLM model for the species Leontodon\_hispidus\_sl.
\item[\code{glm\_Dactylis\_glomerata}] GLM model for the species Dactylis\_glomerata.
\item[\code{glm\_Trifolium\_repens\_sstr}] GLM model for the species Trifolium\_repens\_sstr.
\item[\code{glm\_Geranium\_sylvaticum}] GLM model for the species Geranium\_sylvaticum.
\item[\code{glm\_Ranunculus\_acris\_sl}] GLM model for the species Ranunculus\_acris\_sl.
\item[\code{glm\_Prunella\_vulgaris}] GLM model for the species Prunella\_vulgaris.
\item[\code{glm\_Veronica\_chamaedrys}] GLM model for the species Veronica\_chamaedrys.
\item[\code{glm\_Taraxacum\_officinale\_aggr}] GLM model for the species Taraxacum\_officinale\_aggr.
\item[\code{glm\_Plantago\_lanceolata}] GLM model for the species Plantago\_lanceolata.
\item[\code{glm\_Potentilla\_erecta}] GLM model for the species Potentilla\_erecta.
\item[\code{glm\_Carex\_sempervirens}] GLM model for the species Carex\_sempervirens.
\item[\code{glm\_Soldanella\_alpina}] GLM model for the species Soldanella\_alpina.
\item[\code{glm\_Cynosurus\_cristatus}] GLM model for the species Cynosurus\_cristatus.
\item[\code{glm\_Campanula\_scheuchzeri}] GLM model for the species Campanula\_scheuchzeri.
\item[\code{glm\_Festuca\_pratensis\_sl}] GLM model for the species Festuca\_pratensis\_sl.
\item[\code{gbm\_Bromus\_erectus\_sstr}] GBM model for the species Bromus\_erectus\_sstr.
\item[\code{glm\_Saxifraga\_oppositifolia}] GLM model for the species Saxifraga\_oppositifolia.
\item[\code{glm\_Daucus\_carota}] GLM model for the species Daucus\_carota.
\item[\code{glm\_Pritzelago\_alpina\_sstr}] GLM model for the species Pritzelago\_alpina\_sstr.
\item[\code{glm\_Bromus\_erectus\_sstr}] GLM model for the species Bromus\_erectus\_sstr.
\item[\code{gbm\_Saxifraga\_oppositifolia}] GBM model for the species Saxifraga\_oppositifolia.
\item[\code{gbm\_Daucus\_carota}] GBM model for the species Daucus\_carota.
\item[\code{gbm\_Pritzelago\_alpina\_sstr}] GBM model for the species Pritzelago\_alpina\_sstr.

\end{description}

\end{Format}
%
\begin{Details}\relax
The study area is the Western Swiss Alps of Canton de Vaud, Switzerland.

Five topo-climatic explanatory variables to calibrate the SDMs: growing degree days (with a 0 degrees Celsius threshold); moisture index over the growing season (average values for June to August in mm day-1); slope (in degrees); topographic position (an integrated and unitless measure of topographic exposure; Zimmermann et al., 2007); and the annual sum of radiation (in kJ m-2 year-1). The spatial resolution of the predictor is 25 m x 25 m so that the models could capture most of the small-scale variations of the climatic factors in the mountainous areas.

Two modelling techniques were used to produce the SDMs: generalized linear models (GLM; McCullagh \& Nelder, 1989; R library 'glm') and generalized boosted models (GBM; Friedman, 2001; R library 'gbm'). The SDMs correpond to 20 species: Agrostis\_capillaris, Leontodon\_hispidus\_sl, Dactylis\_glomerata, Trifolium\_repens\_sstr, Geranium\_sylvaticum, Ranunculus\_acris\_sl, Prunella\_vulgaris, Veronica\_chamaedrys, Taraxacum\_officinale\_aggr, Plantago\_lanceolata, Potentilla\_erecta, Carex\_sempervirens, Soldanella\_alpina, Cynosurus\_cristatus, Campanula\_scheuchzeri, Festuca\_pratensis\_sl, Daucus\_carota, Pritzelago\_alpina\_sstr, Bromus\_erectus\_sstr and Saxifraga\_oppositifolia.
\end{Details}
%
\begin{Author}\relax
Antoine Guisan \email{antoine.guisan@unil.ch}, Anne Dubuis \email{anne.dubuis@gmail.com} and Valeria Di Cola \email{valeria.dicola@unil.ch}
\end{Author}
%
\begin{References}\relax
Guisan, A. 1997. Distribution de taxons vegetaux dans un environnement alpin: Application de modelisations statistiques dans un systeme d'information geographique. PhD Thesis, University of Geneva, Switzerland.

Guisan, A., J.P. Theurillat. and F. Kienast. 1998. Predicting the potential distribution of plant species in an alpine environment. \emph{Journal of Vegetation Science}, \bold{9}, 65-74.

Guisan, A. and J.P. Theurillat. 2000. Assessing alpine plant vulnerability to climate change: A modeling perspective. \emph{Integrated Assessment}, \bold{1}, 307-320.

Guisan, A. and J.P. Theurillat. 2000. Equilibrium modeling of alpine plant distribution and climate change : How far can we go? \emph{Phytocoenologia}, \bold{30}(3-4), 353-384.

Dubuis A., J. Pottier, V. Rion, L. Pellissier, J.P. Theurillat and A. Guisan. 2011. Predicting spatial patterns of plant species richness: A comparison of direct macroecological and species stacking approaches. \emph{Diversity and Distributions}, \bold{17}, 1122-1131.

Zimmermann, N.E., T.C. Edwards, G.G Moisen, T.S. Frescino and J.A. Blackard. 2007. Remote sensing-based predictors improve distribution models of rare, early successional and broadleaf tree species in Utah. \emph{Journal of Applied Ecology} \bold{44}, 1057-1067.
\end{References}
%
\begin{Examples}
\begin{ExampleCode}
data(ecospat.testData)
str(ecospat.testData)
dim(ecospat.testData)
names(ecospat.testData)
\end{ExampleCode}
\end{Examples}
\inputencoding{utf8}
\HeaderA{ecospat.testEnvRaster}{Test Environmental Rasters for The Ecospat package}{ecospat.testEnvRaster}
%
\begin{Description}\relax
A stack of 5 topoclimatic rasters at 250m resolution for the Western Swiss Alps. It includes "ddeg0" (growing degree-days above 0C), "mind68" (moisture index for month June to August),  "srad68" (solar radiation for month June to August), "slope25" (average of slopes at 25m resolution) and "topos25" (average of topographic positions at 25m resolution)
\end{Description}
%
\begin{Format}
ecospat.testEnvRaster is a RasterBrick encapsulated in a .Rdata that contains the following rasters:

[1] "ddeg0"       
[2] "mind68"      
[3] "srad68"      
[4] "slope25"            
[5] "topos25"
\end{Format}
%
\begin{Author}\relax
Olivier Broennimann \email{olivier.broennimann@unil.ch}

\end{Author}
%
\begin{References}\relax
Zimmermann, N.E., F. Kienast. 2009. Predictive mapping of alpine grasslands in Switzerland: Species versus community approach. \emph{Journal of Vegetation Science}, \bold{10}, 469-482.
\end{References}
%
\begin{Examples}
\begin{ExampleCode}
fpath <- system.file("extdata", "ecospat.testEnvRaster.RData", package="ecospat")
load(fpath)
plot(env)
\end{ExampleCode}
\end{Examples}
\inputencoding{utf8}
\HeaderA{ecospat.testMdr}{Test Data For The ecospat.mdr function}{ecospat.testMdr}
%
\begin{Description}\relax
Data frame that contains presence records the species \code{Centaurea stoebe} along years in North America.
\end{Description}
%
\begin{Usage}
\begin{verbatim}
data("ecospat.testMdr")
\end{verbatim}
\end{Usage}
%
\begin{Format}
A data frame with 102 observations of \code{Centaurea stoebe}.
\begin{description}

\item[\code{latitude}] Latitude, in WGS coordinate system.
\item[\code{longitude}] Longitude, in WGS coordinate system.
\item[\code{date}] Year of the presence record.

\end{description}

\end{Format}
%
\begin{Details}\relax
Simplified dataset to exemplify the use of the ecospat.mdr function to calculate minimum dispersal routes.
\end{Details}
%
\begin{Author}\relax
Olivier Broennimann \email{olivier.broennimann@unil.ch}
\end{Author}
%
\begin{References}\relax
Broennimann, O., P. Mraz, B. Petitpierre, A. Guisan, and H. Muller-Scharer. 2014. Contrasting spatio-temporal climatic niche dynamics during the eastern and western invasions of spotted knapweed in North America.\emph{Journal of biogeography}, \bold{41}, 1126-1136.

Hordijk, W. and O. Broennimann. 2012. Dispersal routes reconstruction and the minimum cost arborescence problem. \emph{Journal of theoretical biology}, \bold{308}, 115-122.
\end{References}
%
\begin{Examples}
\begin{ExampleCode}
data(ecospat.testMdr)
str(ecospat.testMdr)
dim(ecospat.testMdr)
\end{ExampleCode}
\end{Examples}
\inputencoding{utf8}
\HeaderA{ecospat.testNiche}{Test Data For The Niche Overlap Analysis}{ecospat.testNiche}
%
\begin{Description}\relax
Data frame that contains occurrence sites for each species, long, lat and the name of the species at each site.
\end{Description}
%
\begin{Usage}
\begin{verbatim}
data(ecospat.testNiche)
\end{verbatim}
\end{Usage}
%
\begin{Format}
ecospat.testNiche is a data frame with the following columns:
\begin{description}

\item[\code{species}] sp1, sp2, sp3 and sp4.
\item[\code{long}] Longitude, in Swiss plane coordinate system of the vegetation plot.
\item[\code{lat}] Latitude, in Swiss plane coordinate system of the vegetation plot.
\item[\code{Spp}] Scientific name of the species used in the exmaple: Bromus\_erectus\_sstr, Saxifraga\_oppositifolia, Daucus\_carota and Pritzelago\_alpina\_sstr.

\end{description}

\end{Format}
%
\begin{Details}\relax
List of occurence sites for the species.
\end{Details}
%
\begin{Author}\relax
Antoine Guisan \email{antoine.guisan@unil.ch}, Anne Dubuis \email{anne.dubuis@gmail.com} and Valeria Di Cola \email{valeria.dicola@unil.ch}
\end{Author}
%
\begin{SeeAlso}\relax
\code{\LinkA{ecospat.testData}{ecospat.testData}}
\end{SeeAlso}
%
\begin{Examples}
\begin{ExampleCode}
data(ecospat.testNiche)
dim(ecospat.testNiche)
names(ecospat.testNiche)
\end{ExampleCode}
\end{Examples}
\inputencoding{utf8}
\HeaderA{ecospat.testNiche.inv}{Test Data For The Niche Dynamics Analysis In The Invaded Range Of A Hypothetical Species}{ecospat.testNiche.inv}
%
\begin{Description}\relax
Data frame that contains geographical coordinates, environmental variables, occurrence sites for the studied species and the prediction of its distribution in the invaded range. These predictions are provided by SDM calibrated on the native range.
\end{Description}
%
\begin{Usage}
\begin{verbatim}
data(ecospat.testNiche.inv)
\end{verbatim}
\end{Usage}
%
\begin{Format}
ecospat.testNiche.inv is a data frame with the following columns:
\begin{description}

\item[\code{x}] Longitude, in WGS84 coordinate system of the species occurrence.
\item[\code{y}] Latitude, in WGS84 coordinate system of the species occurrence.
\item[\code{aetpet}] Ratio of actual to potential evapotranspiration.
\item[\code{gdd}] Growing degree-days above 5 degrees C.
\item[\code{p}] Annual amount of precipitations.
\item[\code{pet}] Potential evapotranspiration.
\item[\code{stdp}] Annual variation of precipitations.
\item[\code{tmax}] Maximum temperature of the warmest month.
\item[\code{tmin}] Minimum temperature of the coldest month.
\item[\code{tmp}] Annual mean temperature.
\item[\code{species\_occ}] Presence records of the species occurrence.
\item[\code{predictions}] Species Distribution Model predictions of the studied species.

\end{description}

\end{Format}
%
\begin{Details}\relax
The study area is Australia, which is the invaded range of the hypothetical species.

Eight topo-climatic explanatory variables to quantify niche differences: ratio of the actual potential evapotranspiration; growing degree days; precipitation; potential evapotranspiration; annual variation of precipitations; maximum temperature of the warmest month; minimum temperature of the coldest month; and annual mean temperature.
\end{Details}
%
\begin{Author}\relax
Blaise Petitpierre \email{bpetitpierre@gmail.com} and Valeria Di Cola \email{valeria.dicola@unil.ch}
\end{Author}
%
\begin{References}\relax
Petitpierre, B., C. Kueffer, O. Broennimann, C. Randin, C. Daehler and A. Guisan. 2012. Climatic niche shifts are rare among terrestrial plant invaders. \emph{Science}, \bold{335}, 1344-1348.
\end{References}
%
\begin{SeeAlso}\relax
\code{\LinkA{ecospat.testNiche.nat}{ecospat.testNiche.nat}}
\end{SeeAlso}
%
\begin{Examples}
\begin{ExampleCode}
data(ecospat.testNiche.inv)
str(ecospat.testNiche.inv)
dim(ecospat.testNiche.inv)
names(ecospat.testNiche.inv)
\end{ExampleCode}
\end{Examples}
\inputencoding{utf8}
\HeaderA{ecospat.testNiche.nat}{Test Data For The Niche Dynamics Analysis In The Native Range Of A Hypothetical Species}{ecospat.testNiche.nat}
%
\begin{Description}\relax
Data frame that contains geographical coordinates, environmental variables, occurrence sites for the studied species and the prediction of its distribution in the native range. These predictions are provided by SDM calibrated on the native range.
\end{Description}
%
\begin{Usage}
\begin{verbatim}
data(ecospat.testNiche.nat)
\end{verbatim}
\end{Usage}
%
\begin{Format}
ecospat.testNiche.nat is a data frame with the following columns:
\begin{description}

\item[\code{x}] Longitude, in WGS84 coordinate system of the species occurrence.
\item[\code{y}] Latitude, in WGS84 coordinate system of the species occurrence.
\item[\code{aetpet}] Ratio of actual to potential evapotranspiration.
\item[\code{gdd}] Growing degree-days above 5 degrees C.
\item[\code{p}] Annual amount of precipitations.
\item[\code{pet}] Potential evapotranspiration.
\item[\code{stdp}] Annual variation of precipitations.
\item[\code{tmax}] Maximum temperature of the warmest month.
\item[\code{tmin}] Minimum temperature of the coldest month.
\item[\code{tmp}] Annual mean temperature.
\item[\code{species\_occ}] Presence records of the species occurrence.
\item[\code{predictions}] Species Distribution Model predictions of the studied species.

\end{description}

\end{Format}
%
\begin{Details}\relax
The study area is North America, which is the native range of the hypothetical species.

Eight topo-climatic explanatory variables to quantify niche differences: ratio of the actual potential evapotranspiration; growing degree days; precipitation; potential evapotranspiration; annual variation of precipitations; maximum temperature of the warmest month; minimum temperature of the coldest month; and annual mean temperature.
\end{Details}
%
\begin{Author}\relax
Blaise Petitpierre \email{bpetitpierre@gmail.com} and Valeria Di Cola \email{valeria.dicola@unil.ch}
\end{Author}
%
\begin{References}\relax
Petitpierre, B., C. Kueffer, O. Broennimann, C. Randin, C. Daehler and A. Guisan. 2012. Climatic niche shifts are rare among terrestrial plant invaders. \emph{Science}, \bold{335}, 1344-1348.
\end{References}
%
\begin{SeeAlso}\relax
\code{\LinkA{ecospat.testNiche.inv}{ecospat.testNiche.inv}}
\end{SeeAlso}
%
\begin{Examples}
\begin{ExampleCode}
data(ecospat.testNiche.nat)
str(ecospat.testNiche.nat)
dim(ecospat.testNiche.nat)
names(ecospat.testNiche.nat)
\end{ExampleCode}
\end{Examples}
\inputencoding{utf8}
\HeaderA{ecospat.testTree}{Test Tree For The Ecospat package}{ecospat.testTree}
%
\begin{Description}\relax
The tree object is a phylogenetic tree of class 'phylo' (see read.tree) that contains data of 50 angiosperm species from the Western Swiss Alps.
\end{Description}
%
\begin{Format}
ecospat.testTree is a tree contains the following species:

[1] "Rumex\_acetosa"            
[2] "Polygonum\_bistorta"       
[3] "Polygonum\_viviparum"      
[4] "Rumex\_crispus"            
[5] "Cerastium\_latifolium"     
[6] "Silene\_acaulis"           
[7] "Gypsophila\_repens"        
[8] "Vaccinium\_gaultherioides" 
[9] "Soldanella\_alpina"        
[10] "Cruciata\_laevipes"        
[11] "Galium\_album"             
[12] "Galium\_anisophyllon"      
[13] "Galium\_megalospermum"     
[14] "Gentiana\_verna"           
[15] "Gentiana\_bavarica"        
[16] "Gentiana\_purpurea"        
[17] "Gentiana\_lutea"           
[18] "Bartsia\_alpina"           
[19] "Rhinanthus\_alectorolophus"
[20] "Prunella\_grandiflora"     
[21] "Acinos\_alpinus"           
[22] "Plantago\_alpina"          
[23] "Plantago\_lanceolata"      
[24] "Veronica\_officinalis"     
[25] "Veronica\_aphylla"         
[26] "Veronica\_alpina"          
[27] "Veronica\_chamaedrys"      
[28] "Veronica\_persica"         
[29] "Globularia\_cordifolia"    
[30] "Globularia\_nudicaulis"    
[31] "Myosotis\_alpestris"       
[32] "Myosotis\_arvensis"        
[33] "Aposeris\_foetida"         
[34] "Centaurea\_montana"        
[35] "Hieracium\_lactucella"     
[36] "Leontodon\_helveticus"     
[37] "Leontodon\_autumnalis"     
[38] "Hypochaeris\_radicata"     
[39] "Achillea\_atrata"          
[40] "Achillea\_millefolium"     
[41] "Homogyne\_alpina"          
[42] "Senecio\_doronicum"        
[43] "Adenostyles\_glabra"       
[44] "Arnica\_montana"           
[45] "Aster\_bellidiastrum"      
[46] "Bellis\_perennis"          
[47] "Doronicum\_grandiflorum"   
[48] "Phyteuma\_orbiculare"      
[49] "Phyteuma\_spicatum"        
[50] "Campanula\_rotundifolia" 

\end{Format}
%
\begin{Author}\relax
Charlotte Ndiribe \email{charlotte.ndiribe@unil.ch}, Nicolas Salamin \email{nicolas.salamin@unil.ch} and Antoine Guisan \email{antoine.guisan@unil.ch}

\end{Author}
%
\begin{References}\relax
Ndiribe, C., L. Pellissier, S. Antonelli, A. Dubuis, J. Pottier, P. Vittoz, A. Guisan and N. Salamin. 2013. Phylogenetic plant community structure along elevation is lineage specific. \emph{Ecology and Evolution}, \bold{3}, 4925-4939.
\end{References}
%
\begin{Examples}
\begin{ExampleCode}
fpath <- system.file("extdata", "ecospat.testTree.tre", package="ecospat")
tree <- read.tree(fpath)
plot(tree)
\end{ExampleCode}
\end{Examples}
\inputencoding{utf8}
\HeaderA{ecospat.varpart}{Variation Partitioning For GLM Or GAM}{ecospat.varpart}
%
\begin{Description}\relax
Perform variance partitioning for binomial GLM or GAM based on the deviance of two groups or predicting variables.
\end{Description}
%
\begin{Usage}
\begin{verbatim}
ecospat.varpart (model.1, model.2, model.12)
\end{verbatim}
\end{Usage}
%
\begin{Arguments}
\begin{ldescription}
\item[\code{model.1}] GLM / GAM calibrated on the first group of variables.
\item[\code{model.2}] GLM / GAM calibrated on the second group of variables.
\item[\code{model.12}] GLM / GAM calibrated on all variables from the two groups.
\end{ldescription}
\end{Arguments}
%
\begin{Details}\relax
The deviance is calculated with the adjusted geometric mean squared improvement rescaled for a maximum of 1.
\end{Details}
%
\begin{Value}
Return the four fractions of deviance as in Randin et al. 2009: partial deviance of model 1 and 2, joined deviance and unexplained deviance.
\end{Value}
%
\begin{Author}\relax
Christophe Randin \email{christophe.randin@unibas.ch}, Helene Jaccard and Nigel Gilles Yoccoz
\end{Author}
%
\begin{References}\relax
Randin, C.F., H. Jaccard, P. Vittoz, N.G. Yoccoz and A. Guisan. 2009. Land use improves spatial predictions of mountain plant abundance but not presence-absence. \emph{Journal of Vegetation Science}, \bold{20}, 996-1008.
\end{References}
%
\begin{Examples}
\begin{ExampleCode}
## Not run: 
ecospat.cv.example()
ecospat.varpart (model.1= get ("glm.Achillea_atrata", envir=ecospat.env), 
model.2= get ("glm.Achillea_millefolium", envir=ecospat.env), 
model.12= get ("glm.Achillea_millefolium", envir=ecospat.env))

## End(Not run)
\end{ExampleCode}
\end{Examples}
\printindex{}
\end{document}
